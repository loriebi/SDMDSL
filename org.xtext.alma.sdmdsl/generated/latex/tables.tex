% This is a Generated file ! do not edit !
\subsection{Versioning information for the ASDM.}
\begin{itemize}
\item Version : 
\item CVS revision : -1
\item CVS branch : 
\end{itemize} 
\newpage



\subsection{Main Table}

 
  Contains links to all data subsets. Each data subset is contained in a separate entity, usually a BLOB.


\begingroup
%
% define shortcuts for dimensions

%\newcommand{\numAntenna}{\f$N_{Ante}\f$}
%\newcommand{\numIntegration}{\f$N_{Inte}\f$}

\par\noindent\begin{longtable} {|p{45mm}|p{45mm}|p{65mm}|}
\hline \multicolumn{3}{|c|}{\textbf{Main}} \\
\hline\hline
Name & Type (Shape) & Comment \\
\hline \endfirsthead
\hline \multicolumn{3}{|c|}{\textbf{Main} -- continued from previous page} \\
\hline \hline
Name & Type (Shape) & Comment \\
\hline \endhead
\hline \endfoot


\hline \multicolumn{3}{|l|}{\em Key} \\
\hline 

\texttt{time} & \texttt{ArrayTime} &  mid point of scheduled period. \\
\texttt{configDescriptionId} & \texttt{Tag} &  Configuration description identifier. \\
\texttt{fieldId} & \texttt{Tag} &  Field identifier. \\
\hline \multicolumn{3}{|l|}{\em Required Data} \\
\hline
\texttt{\f$N_{Ante}\f$} (\f$N_{Ante}\f$)& \texttt{int} &
 Number of antennas. \\
\texttt{timeSampling} & \texttt{TimeSampling} &
 time sampling mode. \\
\texttt{interval} & \texttt{Interval} &
 data sampling interval. \\
\texttt{\f$N_{Inte}\f$} (\f$N_{Inte}\f$)& \texttt{int} &
 number of integrations. \\
\texttt{scanNumber} & \texttt{int} &
 scan number. \\
\texttt{subscanNumber} & \texttt{int} &
 subscan number. \\
\texttt{dataSize} & \texttt{int64\_t} &
 size of the binary data , as a number of bytes. \\
\texttt{dataUID} & \texttt{EntityRef} &
 reference to the binary data. \\
\texttt{stateId} & \texttt{Tag [numAntenna] } &
 State identifier. \\
\texttt{execBlockId} & \texttt{Tag} &
 ExecBlock identifier. \\

\hline
\end{longtable}
  
~\par\noindent{\bf Column Descriptions:}

\begin{description}
\item[\texttt{time}]: The mid-point of the scheduled period for the row,   thus not taking into account the effects of data blanking and any   overhead. 
\item[\texttt{configDescriptionId}]: The Configuration     Description Table identifier.  Note that two or more sub-arrays cannot     refer to the same Configuration Description row. The Configuration     Description thus makes possible to identify the various subarrays if     more than one have been used in the same data set. 
\item[\texttt{fieldId}]: The Field Identifier used in the Field Table. 
\item[\texttt{\f$N_{Ante}\f$}]: The number of antennas. Provides the size   of \\texttt{stateId}. 
\item[\texttt{timeSampling}]: This {specifies} whether the sampling   interval is divided into simple integrations, or into integrations further   divided into sub-integrations (for channel averaged correlator data). 
\item[\texttt{interval}]: This is the nominal data interval, as scheduled,   for the whole row. This means that data taking was scheduled to start a   \\texttt{time-interval/2} and end at   \\texttt{time+interval/2}. \\texttt{Interval} corresponds to the sum of all   integrations and does not include the effects of blanking (bad data) or   partial integrations. In ALMA this is the scheduled duration of the   subscan. For the actual subscan start and end times see the Subscan Table. 
\item[\texttt{\f$N_{Inte}\f$}]: The number of integrations in   \\texttt{interval}.  For Alma this is is either true integrations   (for full spectral resolution data), or the total number of   subintegrations in \\texttt{interval} (for channel -averaged   spectral data). 
\item[\texttt{scanNumber}]: In Alma a scan is an amount of data taken to   reach a single result (e.g. a simple calibration). The scan numbers   increment from 1 inside an Execution Block. 
\item[\texttt{subscanNumber}]: In Alma a Subscan is the minimum amount of     data taken by executing a single Control Command Language (CCL)     command. There can be several data cells for each subscan     corresponding to different backends (correlator, total power detectors)     or different results of the same backend (channel averaged or     full-resolution data from a Correlator). In each scan there is     at least one subscan. 
\item[\texttt{dataSize}]: Total size, in bytes, of the binary data file. 
\item[\texttt{dataUID}]: This is  a string that specifies the data object. 
\item[\texttt{stateId}]: The State indentifier used in the State Table. 
\item[\texttt{execBlockId}]: The ExecBlock identifier used in the   ExecBlock Table. For ALMA the ExecBlocks represent each execution of a   Scheduling Block. 
\end{description}
\endgroup

 \newpage

\subsection{AlmaRadiometer Table}

 
 Properties of the Radiometer receiver/backend (used to monitor water vapour  content and correct phases). Note that standard properties (like  spectral coverage) are in the generic tables (like SpectralWindow).

\begingroup
%
% define shortcuts for dimensions

%\newcommand{\numAntenna}{\f$N_{Ante}\f$}

\par\noindent\begin{longtable} {|p{45mm}|p{45mm}|p{65mm}|}
\hline \multicolumn{3}{|c|}{\textbf{AlmaRadiometer}} \\
\hline\hline
Name & Type (Shape) & Comment \\
\hline \endfirsthead
\hline \multicolumn{3}{|c|}{\textbf{AlmaRadiometer} -- continued from previous page} \\
\hline \hline
Name & Type (Shape) & Comment \\
\hline \endhead
\hline \endfoot


\hline \multicolumn{3}{|l|}{\em Key} \\
\hline 

\texttt{almaRadiometerId} & \texttt{Tag} &  identifies a unique row in the table. \\
\hline \multicolumn{3}{|l|}{\em Required Data} \\
\hline

\hline \multicolumn{3}{|l|}{\em Optional Data} \\
\hline
\texttt{\f$N_{Ante}\f$} (\f$N_{Ante}\f$) & \texttt{int} &
 the number of antennas. \\
\texttt{spectralWindowId}  & \texttt{Tag [3*(numAntenna+5)+7] } &
 the references to the actual spectral windows (one spectral window per antenna). \\
\hline
\end{longtable}
  
~\par\noindent{\bf Column Descriptions:}

\begin{description}
\item[\texttt{almaRadiometerId}]: Alma Radiometer Table identifier. 
\item[\texttt{\f$N_{Ante}\f$}]: The number of antennas to which the data refer. 
\item[\texttt{spectralWindowId}]: enter tag descr. here 
\end{description}
\endgroup

 \newpage

\subsection{Annotation Table}

 
 The Annotation Table is intended to offer space for unexpected data to be  added in the software development process at short notice, without  redefining the data model.

\begingroup
%
% define shortcuts for dimensions

%\newcommand{\numAntenna}{\f$N_{Ante}\f$}
%\newcommand{\numBaseband}{\f$N_{Base}\f$}

\par\noindent\begin{longtable} {|p{45mm}|p{45mm}|p{65mm}|}
\hline \multicolumn{3}{|c|}{\textbf{Annotation}} \\
\hline\hline
Name & Type (Shape) & Comment \\
\hline \endfirsthead
\hline \multicolumn{3}{|c|}{\textbf{Annotation} -- continued from previous page} \\
\hline \hline
Name & Type (Shape) & Comment \\
\hline \endhead
\hline \endfoot


\hline \multicolumn{3}{|l|}{\em Key} \\
\hline 

\texttt{annotationId} & \texttt{Tag} &  identifies a unique row in the table. \\
\hline \multicolumn{3}{|l|}{\em Required Data} \\
\hline
\texttt{time} & \texttt{ArrayTime} &
 mid point of the interval of time on which the recorded information is pertinent. \\
\texttt{issue} & \texttt{string} &
 name of this annotation. \\
\texttt{details} & \texttt{string} &
 details of this annotation. \\

\hline \multicolumn{3}{|l|}{\em Optional Data} \\
\hline
\texttt{\f$N_{Ante}\f$} (\f$N_{Ante}\f$) & \texttt{int} &
 number of antennas. \\
\texttt{basebandName}  & \texttt{BasebandName [numBaseband] } &
 an array of numBaseband baseband names. \\
\texttt{\f$N_{Base}\f$} (\f$N_{Base}\f$) & \texttt{int} &
 number of basebands. \\
\texttt{interval}  & \texttt{Interval} &
 time interval \\
\texttt{dValue}  & \texttt{double} &
 scalar data. \\
\texttt{vdValue}  & \texttt{double [] } &
 useful to store an array of double values. \\
\texttt{vvdValues}  & \texttt{double []  [] } &
 useful to store an array of array(s) of double values. \\
\texttt{llValue}  & \texttt{int64\_t} &
 useful to record a long long data. \\
\texttt{vllValue}  & \texttt{int64\_t [] } &
 useful to store an array of array(s) of long long values. \\
\texttt{vvllValue}  & \texttt{int64\_t []  [] } &
 useful to store an array of array(s) long long values. \\
\texttt{antennaId}  & \texttt{Tag [numAntenna] } &
 refers to a collection of rows in the AntennaTable. \\
\hline
\end{longtable}
  
~\par\noindent{\bf Column Descriptions:}

\begin{description}
\item[\texttt{annotationId}]: Annotation Table identifier. 
\item[\texttt{time}]: The midpoint of the time interval the data in this   row are referring to. This is for documentation purposes only. 
\item[\texttt{issue}]: A short (preferably 1-word) string that   identifies the type of annotation. 
\item[\texttt{details}]: Details of this entry: this should   explain the motivation, the dimensionality and contents of the generic   columns: \\texttt{dValue}, \\texttt{llValue} , \\texttt{vdValue},   \\texttt{vllValue}, \\texttt{vvdValues}, \\texttt{vvllValue}. 
\item[\texttt{\f$N_{Ante}\f$}]: The number of antennas to   which the data refer. 
\item[\texttt{basebandName}]: The basebands that the baseband-based   data in this table refer to. 
\item[\texttt{\f$N_{Base}\f$}]: The number of basebands to   which the data refer. 
\item[\texttt{interval}]: Time interval during which the recorded information is pertinent. 
\item[\texttt{dValue}]: space for a scalar floating-point number. 
\item[\texttt{vdValue}]: space for a 1-dimensional array of floating-point data; shape must be made explicit in \\texttt{details}. 
\item[\texttt{vvdValues}]: space for a 2-dimensional array of floating-point data{; shape must be made explicit in \\texttt{details}. 
\item[\texttt{llValue}]: space for a scalar integer. 
\item[\texttt{vllValue}]: space for a 1-dimensional array of integer data; shape must be made explicit in \\texttt{details}. 
\item[\texttt{vvllValue}]: space for a 2-dimensional array of integer data; shape must be made explicit in \\texttt{details}. 
\item[\texttt{antennaId}]: Antenna Table identifier. 
\end{description}
\endgroup

 \newpage

\subsection{Antenna Table}

 
 Antenna characteristics.

\begingroup
%
% define shortcuts for dimensions


\par\noindent\begin{longtable} {|p{45mm}|p{45mm}|p{65mm}|}
\hline \multicolumn{3}{|c|}{\textbf{Antenna}} \\
\hline\hline
Name & Type (Shape) & Comment \\
\hline \endfirsthead
\hline \multicolumn{3}{|c|}{\textbf{Antenna} -- continued from previous page} \\
\hline \hline
Name & Type (Shape) & Comment \\
\hline \endhead
\hline \endfoot


\hline \multicolumn{3}{|l|}{\em Key} \\
\hline 

\texttt{antennaId} & \texttt{Tag} &  identifies a unique row in the table. \\
\hline \multicolumn{3}{|l|}{\em Required Data} \\
\hline
\texttt{name} & \texttt{string} &
 the antenna's name. \\
\texttt{antennaMake} & \texttt{AntennaMake} &
 the antenna's make. \\
\texttt{antennaType} & \texttt{AntennaType} &
 the antenna's type. \\
\texttt{dishDiameter} & \texttt{Length} &
 the diameter of the main reflector. \\
\texttt{position} & \texttt{Length [3] } &
 the antenna's position. \\
\texttt{offset} & \texttt{Length [3] } &
 the position's offset. \\
\texttt{time} & \texttt{ArrayTime} &
 the time of position's measurement. \\
\texttt{stationId} & \texttt{Tag} &
 refers to the station where this antenna is located (i.e. one row in the Station table). \\

\hline \multicolumn{3}{|l|}{\em Optional Data} \\
\hline
\texttt{assocAntennaId}  & \texttt{Tag} &
 refers to an associate antenna (i.e. one row in the Antenna table). \\
\hline
\end{longtable}
  
~\par\noindent{\bf Column Descriptions:}

\begin{description}
\item[\texttt{antennaId}]: Identifies the row in the Antenna Table. 
\item[\texttt{name}]: Provides a unique string identification for the   antenna hardware. \\textit{Examples:} DV01 or DA41 for ALMA antenna prototypes 
\item[\texttt{antennaMake}]: Identifies the antenna manufacturer. Antennas with same optical design may have subtle differences if built according to different designs. 
\item[\texttt{antennaType}]: Generic antenna type; e.g. radio antennas are either for ground use of space use. 
\item[\texttt{dishDiameter}]:  The diameter of the main reflector  (or the largest dimension for non-circular apertures). 
\item[\texttt{position}]: The position of the antenna   pedestal reference point, relative to the station reference point,   measured in the horizon system at the station position. The antenna   pedestal reference point is on the elevation axis, nominally at the same   height as the station reference point (ground level), so that the antenna   position should be always close to zero if the antenna is well positioned   on the station. This is the quantity that has to be re-measured   whenever the antenna is moved to a new station. 
\item[\texttt{offset}]: The position of the antenna phase   reference point in the Yoke, relative to the antenna pedestal reference   point. This is an antenna characteristic that should be unchanged  when the   antenna is moved to a new station.    \\begin{itemize}     \\item The \\$X\\$ component is horizontal along the elevation axis and has no   effect of the interferometer phase; it  can be set arbitrarily to zero.      \\item The \\$Y\\$ component is horizontal and perpendicular to the elevation   axis; it produces an elevation dependent interferometer phase term and has   to be accurately calibrated.    \\item The \\$Z\\$ component is vertical and can be kept equal to the nominal   height of the elevation axis above ground for the antenna's mount. Small   variations from the nominal value have the same phase effect as the \\$Z\\$   component of \\texttt{position}, so they can be ignored.    \\end{itemize}  \\MPositionOffset{YOKE}{Antenna.position}     - The YOKE reference system is defined in ALMA-80.05.00.00-009-B-SPE document;   not known in Measures (CASA) \\\\     - Note - The relevant distance between axes is in the y coordinate, not   x... 
\item[\texttt{time}]: Gives the time at which the positions were measured. 
\item[\texttt{stationId}]: enter tag descr. here 
\item[\texttt{assocAntennaId}]: Identifies an associated antenna in the   Table. This can refer to the same antenna with a position   measured at a different time. 
\end{description}
\endgroup

 \newpage

\subsection{CalAmpli Table}

 
 Amplitude Calibration Result from Telescope Calibration. This calibration checks that observing amplitude calibrators provide reasonable results: From the antenna-based fringe amplitudes rough aperture efficiencies are determined.

\begingroup
%
% define shortcuts for dimensions

%\newcommand{\numReceptor}{\f$N_{Rece}\f$}

\par\noindent\begin{longtable} {|p{45mm}|p{45mm}|p{65mm}|}
\hline \multicolumn{3}{|c|}{\textbf{CalAmpli}} \\
\hline\hline
Name & Type (Shape) & Comment \\
\hline \endfirsthead
\hline \multicolumn{3}{|c|}{\textbf{CalAmpli} -- continued from previous page} \\
\hline \hline
Name & Type (Shape) & Comment \\
\hline \endhead
\hline \endfoot


\hline \multicolumn{3}{|l|}{\em Key} \\
\hline 

\texttt{antennaName} & \texttt{string} &  the antenna's name. \\
\texttt{atmPhaseCorrection} & \texttt{AtmPhaseCorrection} &  qualifies how the atmospheric phase correction has been applied. \\
\texttt{receiverBand} & \texttt{ReceiverBand} &  the name of the receiver band. \\
\texttt{basebandName} & \texttt{BasebandName} &  The name of the 'baseband pair' which is measured. For ALMA a baseband pair is the signal path identified by a second local oscillator and has two polarizations. BB ALL may be used if all basebands are fitted together.  \\
\texttt{calDataId} & \texttt{Tag} &  refers to a unique row in CalData Table. \\
\texttt{calReductionId} & \texttt{Tag} &  refers to a unique row in CalReduction Table. \\
\hline \multicolumn{3}{|l|}{\em Required Data} \\
\hline
\texttt{\f$N_{Rece}\f$} (\f$N_{Rece}\f$)& \texttt{int} &
 the number of receptors. \\
\texttt{polarizationTypes} & \texttt{PolarizationType [numReceptor] } &
 the polarizations of the receptors (an array containing one value per receptor). \\
\texttt{startValidTime} & \texttt{ArrayTime} &
 the start time of result validity period. \\
\texttt{endValidTime} & \texttt{ArrayTime} &
 the end time of result validity period. \\
\texttt{frequencyRange} & \texttt{Frequency [2] } &
 the frequency range over which the result is valid. \\
\texttt{apertureEfficiency} & \texttt{float [numReceptor] } &
 the aperture efficiency without correction. \\
\texttt{apertureEfficiencyError} & \texttt{float [numReceptor] } &
 the aperture efficiency error. \\

\hline \multicolumn{3}{|l|}{\em Optional Data} \\
\hline
\texttt{correctionValidity}  & \texttt{bool} &
 the correction validity. \\
\hline
\end{longtable}
  
~\par\noindent{\bf Column Descriptions:}

\begin{description}
\item[\texttt{antennaName}]: Refers uniquely to the hardware antenna object, as present in the original ASDM Antenna table. 
\item[\texttt{atmPhaseCorrection}]: the atmospheric phase corrections states for which result is given. 
\item[\texttt{receiverBand}]: The name of the front-end frequency band being used. 
\item[\texttt{basebandName}]: {\red long doc missing}
\item[\texttt{calDataId}]: CalData Table identifier. 
\item[\texttt{calReductionId}]: CalReduction Table identifier. 
\item[\texttt{\f$N_{Rece}\f$}]: The number or polarization receptors (one or two) for which the result is given. 
\item[\texttt{polarizationTypes}]: The polarization types of the receptors being used. 
\item[\texttt{startValidTime}]:  The start of the time validity range for the result. 
\item[\texttt{endValidTime}]: The end of the time validity range for the result. 
\item[\texttt{frequencyRange}]: Frequency range over which the result is valid \\MFrequency{TOPO} 
\item[\texttt{apertureEfficiency}]: Antenna aperture efficiency with  and/or without phase correction. 
\item[\texttt{apertureEfficiencyError}]: Error on aperture efficiency measurement. 
\item[\texttt{correctionValidity}]: Deduced validity of atmospheric path length correction (from Water Vapour Radiometers). 
\end{description}
\endgroup

 \newpage

\subsection{CalAppPhase Table}

 
 The CalAppPhase table is relevant to the ALMA observatory when the antennas are being phased to form a coherent sum during the observation. For each scan, the table provides information about which antennas are included in the sum, their relative phase adjustments, the efficiency of the sum (relative to best performance) and the quality of each antenna participating in the system. This data is used in real-time to provide the phased sum signal, and after the observation to analyze the result.

\begingroup
%
% define shortcuts for dimensions

%\newcommand{\numPhasedAntennas}{\f$N_{Phas}\f$}
%\newcommand{\numReceptors}{\f$N_{Rece}\f$}
%\newcommand{\numChannels}{\f$N_{Chan}\f$}
%\newcommand{\numPhaseValues}{\f$N_{Phas}\f$}
%\newcommand{\numCompare}{\f$N_{Comp}\f$}
%\newcommand{\numEfficiencies}{\f$N_{Effi}\f$}
%\newcommand{\numSupports}{\f$N_{Supp}\f$}

\par\noindent\begin{longtable} {|p{45mm}|p{45mm}|p{65mm}|}
\hline \multicolumn{3}{|c|}{\textbf{CalAppPhase}} \\
\hline\hline
Name & Type (Shape) & Comment \\
\hline \endfirsthead
\hline \multicolumn{3}{|c|}{\textbf{CalAppPhase} -- continued from previous page} \\
\hline \hline
Name & Type (Shape) & Comment \\
\hline \endhead
\hline \endfoot


\hline \multicolumn{3}{|l|}{\em Key} \\
\hline 

\texttt{basebandName} & \texttt{BasebandName} &  identifies the baseband. \\
\texttt{scanNumber} & \texttt{int} &  The number of the scan processed by TELCAL. Along with an ExecBlock Id (which should be ExecBlock\\_0 most of the time), the value of scanNumber can be used as the key to retrieve informations related to the scan (e.g. its start time).  \\
\texttt{calDataId} & \texttt{Tag} &  identifies a unique row in the CalData table. \\
\texttt{calReductionId} & \texttt{Tag} &  identifies a unique row in the CalReduction table. \\
\hline \multicolumn{3}{|l|}{\em Required Data} \\
\hline
\texttt{startValidTime} & \texttt{ArrayTime} &
 start of phasing solution validity. \\
\texttt{endValidTime} & \texttt{ArrayTime} &
 end of phasing solution validity. \\
\texttt{adjustTime} & \texttt{ArrayTime} &
 The time of the last adjustment to the phasing analysis via the \\c ParameterTuning  interface. \\
\texttt{adjustToken} & \texttt{string} &
 A parameter supplied via the \\c ParameterTuning interface to indicate the form of adjustment(s) made at adjustTime. Note that TELCAL merely passes this datum and adjustTime through to this table. \\
\texttt{phasingMode} & \texttt{string} &
 The mode in which the phasing system is being operated. \\
\texttt{\f$N_{Phas}\f$} (\f$N_{Phas}\f$)& \texttt{int} &
 the number of antennas in phased sum, \\f\\$N_p\\f\\$. \\
\texttt{phasedAntennas} & \texttt{string [numPhasedAntennas] } &
 the names of the phased antennas. \\
\texttt{refAntennaIndex} & \texttt{int} &
 the index of the reference antenna in the array \\c phasedAntennas . It must be an integer value in the interval \\f\\$ [0, N_p-1]\\f\\$. \\
\texttt{candRefAntennaIndex} & \texttt{int} &
 tne index of a candidate (new) reference antenna in the array phasedAntennas; it must be a integer in the interval \\f\\$[0, N_p-1]\\f\\$. \\
\texttt{phasePacking} & \texttt{string} &
 how to unpack \\c phaseValues. \\
\texttt{\f$N_{Rece}\f$} (\f$N_{Rece}\f$)& \texttt{int} &
 the number of receptors per antenna, \\f\\$N_r\\f\\$.The number (\\f\\$N_r \\le 2 \\f\\$) of receptors per antenna, usually two (polarizations), but it might be one in special cases. \\
\texttt{\f$N_{Chan}\f$} (\f$N_{Chan}\f$)& \texttt{int} &
 the number of data channels, \\f\\$N_d\\f\\$.  \\
\texttt{\f$N_{Phas}\f$} (\f$N_{Phas}\f$)& \texttt{int} &
 The number  of phase data values present in the table, \\f\\$N_v\\f\\$. \\
\texttt{phaseValues} & \texttt{float [numPhaseValues] } &
 the array of phase data values. \\
\texttt{\f$N_{Comp}\f$} (\f$N_{Comp}\f$)& \texttt{int} &
 the number of comparison antennas, \\f\\$N_c\\f\\$. \\
\texttt{\f$N_{Effi}\f$} (\f$N_{Effi}\f$)& \texttt{int} &
 the number of efficiencies, \\f\\$N_e\\f\\$. \\
\texttt{compareArray} & \texttt{string [numCompare] } &
 the names of the comparison antennas. \\
\texttt{efficiencyIndices} & \texttt{int [numEfficiencies] } &
 indices of the antenna(s) in \\c compareArray used to calculate \\c efficiencies; they must be distinct integers in the interval \\f\\$[0, N_c]\\f\\$. \\
\texttt{efficiencies} & \texttt{float [numEfficiencies]  [numChannels] } &
 an array of efficiencies of phased sum. \\
\texttt{quality} & \texttt{float [numPhasedAntennas+numCompare] } &
 quality of phased antennas. \\
\texttt{phasedSumAntenna} & \texttt{string} &
 the name of the phased sum antenna. \\

\hline \multicolumn{3}{|l|}{\em Optional Data} \\
\hline
\texttt{typeSupports}  & \texttt{string} &
 encoding of supporting data values. \\
\texttt{\f$N_{Supp}\f$} (\f$N_{Supp}\f$) & \texttt{int} &
 the number of supporting data values, \\f\\$N_s\\f\\$. \\
\texttt{phaseSupports}  & \texttt{float [numSupports] } &
 an array of supporting data values. \\
\hline
\end{longtable}
  
~\par\noindent{\bf Column Descriptions:}

\begin{description}
\item[\texttt{basebandName}]: identifies the baseband. 
\item[\texttt{scanNumber}]: The number of the scan processed by TELCAL. Along with an ExecBlock Id (which should be
ExecBlock\\_0 most of thetime), thevalue of scanNumber can be used as the key to retrieve informations related to the scan (e.g. its start time). 
\item[\texttt{calDataId}]: identifies a unique row in the CalData table. 
\item[\texttt{calReductionId}]: identifies a unique row in the CalReduction table. 
\item[\texttt{startValidTime}]: The start of the interval in which the phase solution was calculated. Normally the first few seconds of each scan include data before the previous slow phasing solution can be applied, so the valid interval corresponds to the last phasing correction. 
\item[\texttt{endValidTime}]: The end of the interval in which the phase solution was calculated. Note that \\f[ startTime < startValidTime < endValidTime \\le endTime \\f]. 
\item[\texttt{adjustTime}]:  Usually, this is the timestamp of the commanding of the last slow phasing correction. However, other adjustments might also have been made (e.g. \\c phasedArray membership changed in the correlator hardware).
\item[\texttt{adjustToken}]:  A parameter supplied via the \\c ParameterTuning interface to indicate the form of adjustment(s) made at \\c adjustTime . Note that TELCAL merely passes this datum and adjustTime through to this table. 
\item[\texttt{phasingMode}]: The mode in which the phasing system is being operated. 
\item[\texttt{\f$N_{Phas}\f$}]: The number of antennas included in the phased sum. 
\item[\texttt{phasedAntennas}]: The names of the \\f\\$ N_p \\f\\$ antennas contributing to the phased sum. 
\item[\texttt{refAntennaIndex}]: the index of the reference antenna in the array \\c phasedAntennas. It must be an integer value in the array \\c phasedAntennas. 
\item[\texttt{candRefAntennaIndex}]: TELCAL may recommend the adoption of a candidate (new) \\c refAntenna  with this entry (index in \\c phasedAntennas ). This recommendation is always available (in case the current reference antenna becomes unsuitable for some reason), but the VOM is not obliged to adopt the recommendation. It must be an integer in the interval \\f\\$[0, Np-1]\\f\\$. 
\item[\texttt{phasePacking}]:  Indicates one of several possibilities for converting the phase data into TFB commands. 
\item[\texttt{\f$N_{Rece}\f$}]: the number of receptors per antenna, \\f\\$N_r\\f\\$.The number (\\f\\$N_r \\le 2 \\f\\$) of receptors per antenna usually two (polarizations), but it might be one in special cases. 
\item[\texttt{\f$N_{Chan}\f$}]: The number of data channels for which efficiency data is presented, \\f\\$N_d\\f\\$. 
\item[\texttt{\f$N_{Phas}\f$}]: The number of phase data values present in the table, \\f\\$N_v\\f\\$. 
\item[\texttt{phaseValues}]: An array containing the \\f\\$N_v\\f\\$ phase data values. 
\item[\texttt{\f$N_{Comp}\f$}]:  The number  of antennas not included in the phased sum, \\f\\$N_c\\f\\$. 
\item[\texttt{\f$N_{Effi}\f$}]:  The number \\f\\$N_e\\f\\$ of antennas in the array \\c compareArray  used to calculate efficiencies. 
\item[\texttt{compareArray}]: The names of the antennas not in the phased sum, which could be used as comparison antenna. The array of available antennas (to the observation) has \\f\\$(N_p + 1 + N_c)\\f\\$ members; \\f\\$N_p\\f\\$ are in the phase-sum, one is the phased-sum, and \\f\\$N_c\\f\\$ are not. 
\item[\texttt{efficiencyIndices}]: A list of \\f\\$N_e\\f\\$  indices in \\c compareArray  for which efficiencies are calculated. The first index in the list refers to the nominal comparison antenna, the second index refers to a candidate replacement (should the first become unusable), and others may be listed. 
\item[\texttt{efficiencies}]: An array of normalized efficiencies for the phased sum for each data channel. Those for the \\c compAntenna are to be used for decisions; the other values are advisory. The efficiencies are provided per channel for each antenna of \\c compareArray  mentioned in the list \\c efficiencyIndices . 
\item[\texttt{quality}]: A normalized figure of merit (\\f\\$ 0.0 \\le q \\le 1.0\\f\\$) expressing the quality of the solution for every antenna. 
\item[\texttt{phasedSumAntenna}]:  The name of the antenna whose data is discarded in favor of the phased sum. The antenna is also known as \\c cai63Antenna . The efficiency is calculated through the correlation of this antenna with antennas referenced by \\c efficiencyIndices. 
\item[\texttt{typeSupports}]: An indicator of which supporting data is being provided. 
\item[\texttt{\f$N_{Supp}\f$}]:  The number of supporting data values present, \\f\\$N_s\\f\\$. 
\item[\texttt{phaseSupports}]: An array of \\f\\$N_\\f\\$s supporting data values. The presence and use of this array is unspecified; but might include channel average frequencies or supplementary quality data as an assist in the implementation. (Indeed, there is a long list of such items that TelCal could compute.) 
\end{description}
\endgroup

 \newpage

\subsection{CalAtmosphere Table}

 
 Results of atmosphere calibration by TelCal. This calibration determines the system temperatures corrected for atmospheric absorption. Ionospheric effects are not dealt with in the Table.

\begingroup
%
% define shortcuts for dimensions

%\newcommand{\numFreq}{\f$N_{Freq}\f$}
%\newcommand{\numLoad}{\f$N_{Load}\f$}
%\newcommand{\numReceptor}{\f$N_{Rece}\f$}

\par\noindent\begin{longtable} {|p{45mm}|p{45mm}|p{65mm}|}
\hline \multicolumn{3}{|c|}{\textbf{CalAtmosphere}} \\
\hline\hline
Name & Type (Shape) & Comment \\
\hline \endfirsthead
\hline \multicolumn{3}{|c|}{\textbf{CalAtmosphere} -- continued from previous page} \\
\hline \hline
Name & Type (Shape) & Comment \\
\hline \endhead
\hline \endfoot


\hline \multicolumn{3}{|l|}{\em Key} \\
\hline 

\texttt{antennaName} & \texttt{string} &  the name of the antenna. \\
\texttt{receiverBand} & \texttt{ReceiverBand} &  identifies the receiver band. \\
\texttt{basebandName} & \texttt{BasebandName} &  identifies the baseband.  \\
\texttt{calDataId} & \texttt{Tag} &  refers to a unique row in CalData Table. \\
\texttt{calReductionId} & \texttt{Tag} &  refers to a unique row in CalReduction Table. \\
\hline \multicolumn{3}{|l|}{\em Required Data} \\
\hline
\texttt{startValidTime} & \texttt{ArrayTime} &
 the start time of result validity period. \\
\texttt{endValidTime} & \texttt{ArrayTime} &
 the end time of result validity period. \\
\texttt{\f$N_{Freq}\f$} (\f$N_{Freq}\f$)& \texttt{int} &
 the number of frequency points. \\
\texttt{\f$N_{Load}\f$} (\f$N_{Load}\f$)& \texttt{int} &
 the number of loads. \\
\texttt{\f$N_{Rece}\f$} (\f$N_{Rece}\f$)& \texttt{int} &
 the number of receptors. \\
\texttt{forwardEffSpectrum} & \texttt{float [numReceptor]  [numFreq] } &
 the spectra of forward efficiencies (one value per receptor, per frequency). \\
\texttt{frequencyRange} & \texttt{Frequency [2] } &
 the frequency range. \\
\texttt{groundPressure} & \texttt{Pressure} &
 the ground pressure. \\
\texttt{groundRelHumidity} & \texttt{Humidity} &
 the ground relative humidity. \\
\texttt{frequencySpectrum} & \texttt{Frequency [numFreq] } &
 the frequencies. \\
\texttt{groundTemperature} & \texttt{Temperature} &
 the ground temperature. \\
\texttt{polarizationTypes} & \texttt{PolarizationType [numReceptor] } &
 the polarizations of the receptors (an array with one value per receptor). \\
\texttt{powerSkySpectrum} & \texttt{float [numReceptor]  [numFreq] } &
 the powers on the sky (one value per receptor per frequency). \\
\texttt{powerLoadSpectrum} & \texttt{float [numLoad]  [numReceptor]  [numFreq] } &
 the powers on the loads (one value per load per receptor per frequency). \\
\texttt{syscalType} & \texttt{SyscalMethod} &
 the type of calibration used. \\
\texttt{tAtmSpectrum} & \texttt{Temperature [numReceptor]  [numFreq] } &
 the spectra of atmosphere physical  temperatures (one value per receptor per frequency). \\
\texttt{tRecSpectrum} & \texttt{Temperature [numReceptor]  [numFreq] } &
 the spectra of the receptors temperatures (one value  per receptor per frequency). \\
\texttt{tSysSpectrum} & \texttt{Temperature [numReceptor]  [numFreq] } &
 the spectra of system temperatures (one value  per receptor per frequency). \\
\texttt{tauSpectrum} & \texttt{float [numReceptor]  [numFreq] } &
 the spectra of atmosheric optical depths (one value  per receptor per frequency). \\
\texttt{tAtm} & \texttt{Temperature [numReceptor] } &
 the atmosphere physical temperatures (one value per receptor). \\
\texttt{tRec} & \texttt{Temperature [numReceptor] } &
 the receptors temperatures (one value per receptor). \\
\texttt{tSys} & \texttt{Temperature [numReceptor] } &
 the system temperatures (one value per receptor). \\
\texttt{tau} & \texttt{float [numReceptor] } &
 the atmospheric optical depths (one value per receptor). \\
\texttt{water} & \texttt{Length [numReceptor] } &
 the water vapor path lengths (one value per receptor). \\
\texttt{waterError} & \texttt{Length [numReceptor] } &
 the uncertainties of water vapor contents (one value per receptor). \\

\hline \multicolumn{3}{|l|}{\em Optional Data} \\
\hline
\texttt{alphaSpectrum}  & \texttt{float [numReceptor]  [numFreq] } &
 the alpha coefficients, two loads only (one value per receptor per frequency). \\
\texttt{forwardEfficiency}  & \texttt{float [numReceptor] } &
 the forward efficiencies (one value per receptor). \\
\texttt{forwardEfficiencyError}  & \texttt{double [numReceptor] } &
 the uncertainties on forwardEfficiency (one value per receptor). \\
\texttt{sbGain}  & \texttt{float [numReceptor] } &
 the relative gains of LO1 sideband (one value per receptor). \\
\texttt{sbGainError}  & \texttt{float [numReceptor] } &
 the uncertainties on the relative gains of LO1 sideband (one value per receptor). \\
\texttt{sbGainSpectrum}  & \texttt{float [numReceptor]  [numFreq] } &
 the spectra of relative sideband gains (one value  per receptor per frequency). \\
\hline
\end{longtable}
  
~\par\noindent{\bf Column Descriptions:}

\begin{description}
\item[\texttt{antennaName}]: Refers uniquely to the hardware antenna object, as present in the original ASDM Antenna table. 
\item[\texttt{receiverBand}]: The name of the front-end frequency band being used. 
\item[\texttt{basebandName}]: {\red long doc missing}
\item[\texttt{calDataId}]: CalData Table identifier. 
\item[\texttt{calReductionId}]: CalReduction Table identifier. 
\item[\texttt{startValidTime}]: The start of the time validity range for the result. 
\item[\texttt{endValidTime}]: The end of the time validity range for the result. 
\item[\texttt{\f$N_{Freq}\f$}]: Number of frequency points for which the results are given. 
\item[\texttt{\f$N_{Load}\f$}]: \\numLoad\\ Number of loads used in calibration. 
\item[\texttt{\f$N_{Rece}\f$}]: The number or polarization receptors (one or two) for which the result is given. 
\item[\texttt{forwardEffSpectrum}]: The value of the forward efficiency for each frequency point. 
\item[\texttt{frequencyRange}]: Frequency range over which the result is valid   {\\MFrequency{TOPO}} 
\item[\texttt{groundPressure}]: The atmospheric pressure at the altitude of the observatory. 
\item[\texttt{groundRelHumidity}]: The relative atmospheric humidity (\\%) at the altitude of the observatory. 
\item[\texttt{frequencySpectrum}]: The frequency values for which the results are given. 
\item[\texttt{groundTemperature}]: The ambient temperature at the observatory. 
\item[\texttt{polarizationTypes}]: The polarization types of the receptors being used. 
\item[\texttt{powerSkySpectrum}]: Observed power on sky. 
\item[\texttt{powerLoadSpectrum}]: Observed power on loads. 
\item[\texttt{syscalType}]: The type of calibration used: a single-direction measurement, or  a series of measurements at different elevations ('SkyDip') 
\item[\texttt{tAtmSpectrum}]: The value of atmosphere physical temperature for each frequency point. 
\item[\texttt{tRecSpectrum}]: The value of the receiver  temperature for each frequency point. 
\item[\texttt{tSysSpectrum}]: The value of the system temperature for each frequency point. 
\item[\texttt{tauSpectrum}]: The value of the optical depth for each frequency point. 
\item[\texttt{tAtm}]: The physical temperature of the atmosphere absorbing layers. 
\item[\texttt{tRec}]: The receiver noise temperature (the reference plane is at the level where the calibration loads are inserted in the signal path). 
\item[\texttt{tSys}]: The system temperature (corrected for atmospheric absorption and antenna losses). 
\item[\texttt{tau}]: The optical depth of the atmosphere along the line of sight. 
\item[\texttt{water}]: The amount of precipitable water vapour in the atmosphere. 
\item[\texttt{waterError}]: The uncertainty of the amount of precipitable water vapour in the atmosphere. 
\item[\texttt{alphaSpectrum}]:  alpha coefficient   (two-load only); see \\href{https://wikio.nrao.edu/pub/ALMA/CalExamples/AmpliCalStepByStep.pdf}{\\em   Amplitude Calibration steps} memo. 
\item[\texttt{forwardEfficiency}]: This is the coupling factor to the sky,   that is the fraction of the antenna beam that sees the emission from the   atmosphere. 
\item[\texttt{forwardEfficiencyError}]: The uncertainty of \\texttt{forwardEfficiency} 
\item[\texttt{sbGain}]: The relative gain of the side band. This is the   ratio of the gain {in the first LO sideband used to obtain the   \\texttt{frequencyRange} relative to the total (dual-sideband) gain. 
\item[\texttt{sbGainError}]: Uncertainty on \\texttt{sbGain}. 
\item[\texttt{sbGainSpectrum}]: The value of the relative side band gain   for each spectral point. Optional for EVLA, mandatory for ALMA. 
\end{description}
\endgroup

 \newpage

\subsection{CalBandpass Table}

 
 Result of passband calibration performed on-line by TelCal.

\begingroup
%
% define shortcuts for dimensions

%\newcommand{\numAntenna}{\f$N_{Ante}\f$}
%\newcommand{\numPoly}{\f$N_{Poly}\f$}
%\newcommand{\numReceptor}{\f$N_{Rece}\f$}
%\newcommand{\numBaseline}{\f$N_{Base}\f$}

\par\noindent\begin{longtable} {|p{45mm}|p{45mm}|p{65mm}|}
\hline \multicolumn{3}{|c|}{\textbf{CalBandpass}} \\
\hline\hline
Name & Type (Shape) & Comment \\
\hline \endfirsthead
\hline \multicolumn{3}{|c|}{\textbf{CalBandpass} -- continued from previous page} \\
\hline \hline
Name & Type (Shape) & Comment \\
\hline \endhead
\hline \endfoot


\hline \multicolumn{3}{|l|}{\em Key} \\
\hline 

\texttt{basebandName} & \texttt{BasebandName} &  identifies the baseband. \\
\texttt{sideband} & \texttt{NetSideband} &  identifies the first LO sideband. \\
\texttt{atmPhaseCorrection} & \texttt{AtmPhaseCorrection} &  qualifies how the atmospheric phase correction has been applied. \\
\texttt{typeCurve} & \texttt{CalCurveType} &  identifies the type of curve. \\
\texttt{receiverBand} & \texttt{ReceiverBand} &  identifies the receiver band. \\
\texttt{calDataId} & \texttt{Tag} &  refers to a unique row in CalData Table. \\
\texttt{calReductionId} & \texttt{Tag} &  refers to a unique row in CalReduction Table. \\
\hline \multicolumn{3}{|l|}{\em Required Data} \\
\hline
\texttt{startValidTime} & \texttt{ArrayTime} &
 the start time of result validity period. \\
\texttt{endValidTime} & \texttt{ArrayTime} &
 the end time of result validity period. \\
\texttt{\f$N_{Ante}\f$} (\f$N_{Ante}\f$)& \texttt{int} &
 the number of antennas. \\
\texttt{\f$N_{Poly}\f$} (\f$N_{Poly}\f$)& \texttt{int} &
 the number of coefficients of the polynomial. \\
\texttt{\f$N_{Rece}\f$} (\f$N_{Rece}\f$)& \texttt{int} &
 the number of receptors. \\
\texttt{antennaNames} & \texttt{string [numAntenna] } &
 the names of the antennas. \\
\texttt{refAntennaName} & \texttt{string} &
 the name of the reference antenna. \\
\texttt{freqLimits} & \texttt{Frequency [2] } &
 the frequency range for the polynomial description of the passband. \\
\texttt{polarizationTypes} & \texttt{PolarizationType [numReceptor] } &
 the polarizations of the receptors (one value per receptor). \\
\texttt{curve} & \texttt{float [numAntenna]  [numReceptor]  [numPoly] } &
 the amplitude or phase coefficients, depending on the value of typeCurve (one array of numPoly values per antenna per receptor). \\
\texttt{reducedChiSquared} & \texttt{double [numReceptor] } &
 measures the quality of the least squares fits (one value per receptor). \\

\hline \multicolumn{3}{|l|}{\em Optional Data} \\
\hline
\texttt{\f$N_{Base}\f$} (\f$N_{Base}\f$) & \texttt{int} &
 the number of baselines. \\
\texttt{rms}  & \texttt{float [numReceptor]  [numBaseline] } &
 the amplitude or phase residuals ( one array of numBaseline values per receptor). \\
\hline
\end{longtable}
  
~\par\noindent{\bf Column Descriptions:}

\begin{description}
\item[\texttt{basebandName}]: The name of the 'baseband pair' which is   measured. For ALMA a baseband pair is the signal path identified by a   second local oscillator and has two polarizations.   \\texttt{BB\\_ALL} may be used if all basebands are fitted together. 
\item[\texttt{sideband}]: First LO Sideband: For sideband-separated      spectra one must use different bandpasses for each individual      sideband. 
\item[\texttt{atmPhaseCorrection}]: the atmospheric phase corrections   states for which result is given. 
\item[\texttt{typeCurve}]: Defines the quantity which is fitted:      amplitude ot phase. 
\item[\texttt{receiverBand}]: The name of the front-end frequency band being used. 
\item[\texttt{calDataId}]: CalData Table identifier. 
\item[\texttt{calReductionId}]: CalReduction Table identifier. 
\item[\texttt{startValidTime}]: The start of the time validity range for the result. 
\item[\texttt{endValidTime}]: The end of the time validity range for the result. 
\item[\texttt{\f$N_{Ante}\f$}]: Number of antennas for which the result is valid. 
\item[\texttt{\f$N_{Poly}\f$}]: Number of polynomial coefficients. 
\item[\texttt{\f$N_{Rece}\f$}]: The number or polarization receptors (one or two) for which the result is given. 
\item[\texttt{antennaNames}]: Refer uniquely to the hardware antenna   object, as present in the original ASDM Antenna table. 
\item[\texttt{refAntennaName}]: The name of the antenna used as reference      to get the antenna-based phases. 
\item[\texttt{freqLimits}]: The frequency limits for the polynomial description of the passband. This frequency interval is reduced to the (-1, 1) interval over which the Chebychev polynomials are defined. 
\item[\texttt{polarizationTypes}]: The polarization types of the receptors being used. 
\item[\texttt{curve}]: The amplitude or phase coefficients. 
\item[\texttt{reducedChiSquared}]: Reduced \\$\\chi^2\\$ indicating the quality      of the least-squares fit. This is a single number for each polarization      as the baselines are to be fitted consistently. 
\item[\texttt{\f$N_{Base}\f$}]: Number of baselines for which the result is given 
\item[\texttt{rms}]:  The rms of the amplitude/phase residuals for each      baseline/polarisation. 
\end{description}
\endgroup

 \newpage

\subsection{CalCurve Table}

 
 Result of time-dependent calibration performed on-line by TelCal

\begingroup
%
% define shortcuts for dimensions

%\newcommand{\numAntenna}{\f$N_{Ante}\f$}
%\newcommand{\numPoly}{\f$N_{Poly}\f$}
%\newcommand{\numReceptor}{\f$N_{Rece}\f$}
%\newcommand{\numBaseline}{\f$N_{Base}\f$}

\par\noindent\begin{longtable} {|p{45mm}|p{45mm}|p{65mm}|}
\hline \multicolumn{3}{|c|}{\textbf{CalCurve}} \\
\hline\hline
Name & Type (Shape) & Comment \\
\hline \endfirsthead
\hline \multicolumn{3}{|c|}{\textbf{CalCurve} -- continued from previous page} \\
\hline \hline
Name & Type (Shape) & Comment \\
\hline \endhead
\hline \endfoot


\hline \multicolumn{3}{|l|}{\em Key} \\
\hline 

\texttt{atmPhaseCorrection} & \texttt{AtmPhaseCorrection} &  qualifies how the atmospheric phase correction has been applied. \\
\texttt{typeCurve} & \texttt{CalCurveType} &  identifies the type of curve. \\
\texttt{receiverBand} & \texttt{ReceiverBand} &  identifies the receiver band. \\
\texttt{calDataId} & \texttt{Tag} &  refers to a unique row in CalData Table. \\
\texttt{calReductionId} & \texttt{Tag} &  refers to a unique row in CalReduction Table. \\
\hline \multicolumn{3}{|l|}{\em Required Data} \\
\hline
\texttt{startValidTime} & \texttt{ArrayTime} &
 the start time of result validity period. \\
\texttt{endValidTime} & \texttt{ArrayTime} &
  the end time of result validity period. \\
\texttt{frequencyRange} & \texttt{Frequency [2] } &
 the range of frequencies over which the result is valid. \\
\texttt{\f$N_{Ante}\f$} (\f$N_{Ante}\f$)& \texttt{int} &
 the number of antennas. \\
\texttt{\f$N_{Poly}\f$} (\f$N_{Poly}\f$)& \texttt{int} &
 the number of coefficients of the polynomials. \\
\texttt{\f$N_{Rece}\f$} (\f$N_{Rece}\f$)& \texttt{int} &
 the number of receptors. \\
\texttt{antennaNames} & \texttt{string [numAntenna] } &
 the names of the antennas. \\
\texttt{refAntennaName} & \texttt{string} &
 the name of the reference antenna. \\
\texttt{polarizationTypes} & \texttt{PolarizationType [numReceptor] } &
 identifies the polarizations of the receptors (one value per receptor). \\
\texttt{curve} & \texttt{float [numAntenna]  [numReceptor]  [numPoly] } &
 the coefficients of the polynomials (one array of numPoly coefficients per receptor per antenna). \\
\texttt{reducedChiSquared} & \texttt{double [numReceptor] } &
 measures the quality of the least squares fits (one value per receptor). \\

\hline \multicolumn{3}{|l|}{\em Optional Data} \\
\hline
\texttt{\f$N_{Base}\f$} (\f$N_{Base}\f$) & \texttt{int} &
 the number of baselines. \\
\texttt{rms}  & \texttt{float [numReceptor]  [numBaseline] } &
 the amplitude or phase residuals (one array of numBaselines values per receptor). \\
\hline
\end{longtable}
  
~\par\noindent{\bf Column Descriptions:}

\begin{description}
\item[\texttt{atmPhaseCorrection}]: the atmospheric phase correction state      for which result is given. 
\item[\texttt{typeCurve}]: Defines the quantity which is fitted: amplitude or phase. 
\item[\texttt{receiverBand}]: The name of the front-end frequency band being used. 
\item[\texttt{calDataId}]: CalData Table identifier. 
\item[\texttt{calReductionId}]: CalReduction Table identifier. 
\item[\texttt{startValidTime}]: The start of the time validity range for the result. 
\item[\texttt{endValidTime}]: The end of the time validity range for the result. 
\item[\texttt{frequencyRange}]: Frequency range over which the result is valid. {\\MFrequency{TOPO}} 
\item[\texttt{\f$N_{Ante}\f$}]: Number of antennas for which the result is valid. 
\item[\texttt{\f$N_{Poly}\f$}]: The number of coefficients in the Chebichev polynomials       used to fit the data. 
\item[\texttt{\f$N_{Rece}\f$}]: The number or polarization receptors (one or two)       for which the result is given. 
\item[\texttt{antennaNames}]: Refer uniquely to the hardware antenna object,      as present in the original ASDM Antenna table. 
\item[\texttt{refAntennaName}]: The name of the antenna used as reference to get       the antenna-based phases. 
\item[\texttt{polarizationTypes}]: The polarization types of the receptors being used. 
\item[\texttt{curve}]:  These are Chebichev polynomial   coefficients. The interval between startValidTime and endValidTime is   reduced to the \\$\\[-1,1\\]\\$ interval over which the Chebychev coefficients are   defined.  For interferometer amplitude, data is expressed in   terms of correlation coefficient; for interferomete phase, coefficients   are given in radians. 
\item[\texttt{reducedChiSquared}]:  Reduced \\$\\chi^2\\$ indicating the      quality of the least-squares fit. 
\item[\texttt{\f$N_{Base}\f$}]: Number of baselines ffor which the result is given. 
\item[\texttt{rms}]: The root means square deviations of fit residuals. 
\end{description}
\endgroup

 \newpage

\subsection{CalData Table}

 
 This table describes the data used to derive the calibration results.

\begingroup
%
% define shortcuts for dimensions

%\newcommand{\numScan}{\f$N_{Scan}\f$}

\par\noindent\begin{longtable} {|p{45mm}|p{45mm}|p{65mm}|}
\hline \multicolumn{3}{|c|}{\textbf{CalData}} \\
\hline\hline
Name & Type (Shape) & Comment \\
\hline \endfirsthead
\hline \multicolumn{3}{|c|}{\textbf{CalData} -- continued from previous page} \\
\hline \hline
Name & Type (Shape) & Comment \\
\hline \endhead
\hline \endfoot


\hline \multicolumn{3}{|l|}{\em Key} \\
\hline 

\texttt{calDataId} & \texttt{Tag} &  identifies a unique row in the table. \\
\hline \multicolumn{3}{|l|}{\em Required Data} \\
\hline
\texttt{startTimeObserved} & \texttt{ArrayTime} &
 the start time of observation. \\
\texttt{endTimeObserved} & \texttt{ArrayTime} &
 the end time of observation. \\
\texttt{execBlockUID} & \texttt{EntityRef} &
 the UID of the Execution Block. \\
\texttt{calDataType} & \texttt{CalDataOrigin} &
 identifies the origin of the data used for the calibration. \\
\texttt{calType} & \texttt{CalType} &
 identifies the type of performed calibration. \\
\texttt{\f$N_{Scan}\f$} (\f$N_{Scan}\f$)& \texttt{int} &
 the number of scans (in this Execution Block). \\
\texttt{scanSet} & \texttt{int [numScan] } &
 the set of scan numbers. \\

\hline \multicolumn{3}{|l|}{\em Optional Data} \\
\hline
\texttt{assocCalDataId}  & \texttt{Tag} &
 refers to an associate row in CalDataTable. \\
\texttt{assocCalNature}  & \texttt{AssociatedCalNature} &
 identifies the nature of the relation with the associate row in CalDataTable. \\
\texttt{fieldName}  & \texttt{string [numScan] } &
 the names of the fields (one name per scan). \\
\texttt{sourceName}  & \texttt{string [numScan] } &
 the names of the sources as given during observations (one source name per scan). \\
\texttt{sourceCode}  & \texttt{string [numScan] } &
 the special characteristics of sources expressed in a textual form (one string per scan). \\
\texttt{scanIntent}  & \texttt{ScanIntent [numScan] } &
 identifies the intents of  the scans (one value per scan). \\
\hline
\end{longtable}
  
~\par\noindent{\bf Column Descriptions:}

\begin{description}
\item[\texttt{calDataId}]: Row identifier. 
\item[\texttt{startTimeObserved}]: The start time of the data set used. 
\item[\texttt{endTimeObserved}]: The end time of the data set used. 
\item[\texttt{execBlockUID}]: Archive UID of the ExecBlock. 
\item[\texttt{calDataType}]:  The origin of the data used: Correlator (full     resolution or channel averaged data), Total power detectors, WVR     receivers, etc. This information is added as e.g. a pointing scan     may have been reduced using either total power or interferometry, or the     atmosphere may be calibrated using autocorrelations or total power     detectors. 
\item[\texttt{calType}]: This enumerated item tells in which calibration     table the results is. 
\item[\texttt{\f$N_{Scan}\f$}]: The number of scans in the scan set  used. 
\item[\texttt{scanSet}]: The list of scan numbers in the set of scans used. 
\item[\texttt{assocCalDataId}]: This is used to chain data sets obtained     in different ExecBlocks, for which scan numbers are re-used. 
\item[\texttt{assocCalNature}]: Nature of the association established by \\texttt{assocCalDataId}. Normally this is used to refer to data in different     execution blocks, for which scan numbers may be re-used. 
\item[\texttt{fieldName}]: The name of the field the array was aimed at. 
\item[\texttt{sourceName}]: Names of the sources as given during observations. 
\item[\texttt{sourceCode}]: Special characteristics of source, e.g. passband calibrator, phase calibrator, flux calibrator. 
\item[\texttt{scanIntent}]: The list of the intents associated with each      scan in the data set. 
\end{description}
\endgroup

 \newpage

\subsection{CalDelay Table}

 
 Result of delay offset calibration performed on-line by  TelCal. This calibration determines the delay offsets to be added in the  correlator to compensate for residual cable delays.   Delays are entered in seconds but represented as double precision floating point numbers.

\begingroup
%
% define shortcuts for dimensions

%\newcommand{\numReceptor}{\f$N_{Rece}\f$}
%\newcommand{\numSideband}{\f$N_{Side}\f$}

\par\noindent\begin{longtable} {|p{45mm}|p{45mm}|p{65mm}|}
\hline \multicolumn{3}{|c|}{\textbf{CalDelay}} \\
\hline\hline
Name & Type (Shape) & Comment \\
\hline \endfirsthead
\hline \multicolumn{3}{|c|}{\textbf{CalDelay} -- continued from previous page} \\
\hline \hline
Name & Type (Shape) & Comment \\
\hline \endhead
\hline \endfoot


\hline \multicolumn{3}{|l|}{\em Key} \\
\hline 

\texttt{antennaName} & \texttt{string} &  the name of the antenna. \\
\texttt{atmPhaseCorrection} & \texttt{AtmPhaseCorrection} &  qualifies how the atmospheric phase correction has been applied. \\
\texttt{basebandName} & \texttt{BasebandName} &  Name of the Baseband \\
\texttt{receiverBand} & \texttt{ReceiverBand} &   identifies the receiver band. \\
\texttt{calDataId} & \texttt{Tag} &  refers to a unique row in CalData Table. \\
\texttt{calReductionId} & \texttt{Tag} &  refers to a unique row in CalReduction Table. \\
\hline \multicolumn{3}{|l|}{\em Required Data} \\
\hline
\texttt{startValidTime} & \texttt{ArrayTime} &
 the start time of the result validity period. \\
\texttt{endValidTime} & \texttt{ArrayTime} &
 the end time of the result validity period. \\
\texttt{refAntennaName} & \texttt{string} &
 the name of the reference antenna. \\
\texttt{\f$N_{Rece}\f$} (\f$N_{Rece}\f$)& \texttt{int} &
 the number of receptors. \\
\texttt{delayError} & \texttt{double [numReceptor] } &
 the uncertainties on the measured delay offsets (one value per receptor). \\
\texttt{delayOffset} & \texttt{double [numReceptor] } &
 the measured delay offsets (one value per receptor). \\
\texttt{polarizationTypes} & \texttt{PolarizationType [numReceptor] } &
 identifies the polarizations of the receptors (one value per receptor). \\
\texttt{reducedChiSquared} & \texttt{double [numReceptor] } &
 measure of the quality of the fit (one value per receptor). \\
\texttt{appliedDelay} & \texttt{double [numReceptor] } &
 the delay that was applied (one value per receptor).  \\

\hline \multicolumn{3}{|l|}{\em Optional Data} \\
\hline
\texttt{crossDelayOffset}  & \texttt{double} &
 the measured cross delay offset (reference antenna only). \\
\texttt{crossDelayOffsetError}  & \texttt{double} &
 the uncertainty for the cross delay offset. \\
\texttt{\f$N_{Side}\f$} (\f$N_{Side}\f$) & \texttt{int} &
 the number of sideband. \\
\texttt{refFreq}  & \texttt{Frequency [numSideband] } &
 the reference frequencies (one value per sideband). \\
\texttt{refFreqPhase}  & \texttt{Angle [numSideband] } &
 the phases at reference frequencies (one value per sideband). \\
\texttt{sidebands}  & \texttt{ReceiverSideband [numSideband] } &
 identifies the receiver's sidebands (one value per sideband). \\
\hline
\end{longtable}
  
~\par\noindent{\bf Column Descriptions:}

\begin{description}
\item[\texttt{antennaName}]: Refers uniquely to the hardware antenna     object, as present in the original ASDM Antenna table. 
\item[\texttt{atmPhaseCorrection}]: The atmospheric phase corrections states      for which result is given. 
\item[\texttt{basebandName}]: The name of the 'baseband pair' which is     measured. For ALMA a baseband pair is the signal path identified by a     second local oscillator and has two polarizations.    \\texttt{BB\\_ALL} may be used if all basebands are fitted together. 
\item[\texttt{receiverBand}]: The name of the front-end frequency band being used. 
\item[\texttt{calDataId}]: CalData Table identifier. 
\item[\texttt{calReductionId}]: CalReduction Table identifier. 
\item[\texttt{startValidTime}]: The start of the time validity range for the result. 
\item[\texttt{endValidTime}]: The end of the time validity range for the result. 
\item[\texttt{refAntennaName}]: The name of the antenna used as reference     to get the antenna-based phases. 
\item[\texttt{\f$N_{Rece}\f$}]: The number or polarization receptors (one or     two) for which the result is given. 
\item[\texttt{delayError}]: The statistical uncertainty on the delay     offset found by TelCal for the specified antenna, receiver band, and     baseband. 
\item[\texttt{delayOffset}]: The delay offset found by TelCal for the     specified antenna, receiver band, and baseband. 
\item[\texttt{polarizationTypes}]: The nominal polarization     types of the receptors being used. 
\item[\texttt{reducedChiSquared}]: Reduced \\$\\chi^2\\$      indicating the quality of the least-squares fit. 
\item[\texttt{appliedDelay}]: {\red long doc missing}
\item[\texttt{crossDelayOffset}]: The cross-polarization delay offset     found by TelCal for the specified receiver band, and baseband . Note :     this must be the same for all antennas; this is the delay to be added     to Y signals relative to X signals to get a flat frequency dependence of     phases for a polarized point source. 
\item[\texttt{crossDelayOffsetError}]: The uncertainty on the     cross-polarization delay offset found by TelCal for the specified     receiver band, and baseband. 
\item[\texttt{\f$N_{Side}\f$}]: \\numSideband\\ Number of Sidebands: in the     side-band separated case, data from both sidebands are available for a     given baseband. The delay offset should be the same for both sidebands,     but the phase in \\texttt{refFreqPhase} should be sideband-dependent. 
\item[\texttt{refFreq}]: A reference frequency within the band. 
\item[\texttt{refFreqPhase}]: Phase fitted at the frequency {\\texttt refFreq}. 
\item[\texttt{sidebands}]: Receiver side bands of the     reference frequencies given in \\texttt{refFreq}. 
\end{description}
\endgroup

 \newpage

\subsection{CalDevice Table}

 
 Calibration device characteristics. This table is not part of the   Calibration Data Model but describes the actual observations; it refers to   the amplitude calibration device which includes the hot loads.     Calibration device properties are assumed independent of frequency   throughout a spectral window.

\begingroup
%
% define shortcuts for dimensions

%\newcommand{\numCalload}{\f$N_{Call}\f$}
%\newcommand{\numReceptor}{\f$N_{Rece}\f$}

\par\noindent\begin{longtable} {|p{45mm}|p{45mm}|p{65mm}|}
\hline \multicolumn{3}{|c|}{\textbf{CalDevice}} \\
\hline\hline
Name & Type (Shape) & Comment \\
\hline \endfirsthead
\hline \multicolumn{3}{|c|}{\textbf{CalDevice} -- continued from previous page} \\
\hline \hline
Name & Type (Shape) & Comment \\
\hline \endhead
\hline \endfoot


\hline \multicolumn{3}{|l|}{\em Key} \\
\hline 

\texttt{antennaId} & \texttt{Tag} &  refers to a unique row  in AntennaTable. \\
\texttt{spectralWindowId} & \texttt{Tag} &  refers to a unique row in SpectralWindow Table. \\
\texttt{timeInterval} & \texttt{ArrayTimeInterval} &  the period of validity of the data recorded in this row. \\
\texttt{feedId} & \texttt{int} &  refers to the collection of rows in FeedTable having this value of feedId in their key. \\
\hline \multicolumn{3}{|l|}{\em Required Data} \\
\hline
\texttt{\f$N_{Call}\f$} (\f$N_{Call}\f$)& \texttt{int} &
 the number of calibration loads. \\
\texttt{calLoadNames} & \texttt{CalibrationDevice [numCalload] } &
 identifies the calibration loads (an array with one value per load). \\

\hline \multicolumn{3}{|l|}{\em Optional Data} \\
\hline
\texttt{\f$N_{Rece}\f$} (\f$N_{Rece}\f$) & \texttt{int} &
 the number of receptors. \\
\texttt{calEff}  & \texttt{float [numReceptor]  [numCalload] } &
 the calibration efficiencies (one value per receptor per load). \\
\texttt{noiseCal}  & \texttt{double [numCalload] } &
 the equivalent temperatures of the of the noise sources used (one value per load). \\
\texttt{coupledNoiseCal}  & \texttt{float [numReceptor]  [numCalload] } &
   \\
\texttt{temperatureLoad}  & \texttt{Temperature [numCalload] } &
 the physical temperatures of the loads for a black body calibration source (one value per load). \\
\hline
\end{longtable}
  
~\par\noindent{\bf Column Descriptions:}

\begin{description}
\item[\texttt{antennaId}]: Antenna Table identifier. 
\item[\texttt{spectralWindowId}]: SpectralWindow Table identifier. 
\item[\texttt{timeInterval}]: Validity time interval for the data in the row. 
\item[\texttt{feedId}]: Specifies which feed was used in the Feed Table. 
\item[\texttt{\f$N_{Call}\f$}]: The number of calibration loads for     which data are  given. 
\item[\texttt{calLoadNames}]: The names of the calibration loads for which     data are provided. 
\item[\texttt{\f$N_{Rece}\f$}]: The number of receptors. 
\item[\texttt{calEff}]: The coupling factor of the calibration source to     the receiver beam. 
\item[\texttt{noiseCal}]: The equivalent  temperature of the noise source used. 
\item[\texttt{coupledNoiseCal}]: {\red long doc missing}
\item[\texttt{temperatureLoad}]: The physical temperature of the load (for     a black-body calibration source). 
\end{description}
\endgroup

 \newpage

\subsection{CalFlux Table}

 
 Result of flux calibration performed on-line by TelCal. Atmospheric absorption is corrected for. No ionosphere correction has been applied.

\begingroup
%
% define shortcuts for dimensions

%\newcommand{\numFrequencyRanges}{\f$N_{Freq}\f$}
%\newcommand{\numStokes}{\f$N_{Stok}\f$}

\par\noindent\begin{longtable} {|p{45mm}|p{45mm}|p{65mm}|}
\hline \multicolumn{3}{|c|}{\textbf{CalFlux}} \\
\hline\hline
Name & Type (Shape) & Comment \\
\hline \endfirsthead
\hline \multicolumn{3}{|c|}{\textbf{CalFlux} -- continued from previous page} \\
\hline \hline
Name & Type (Shape) & Comment \\
\hline \endhead
\hline \endfoot


\hline \multicolumn{3}{|l|}{\em Key} \\
\hline 

\texttt{sourceName} & \texttt{string} &  the name of the source. \\
\texttt{calDataId} & \texttt{Tag} &  refers to a unique row in CalData Table. \\
\texttt{calReductionId} & \texttt{Tag} &  refers to a unique row in CalReduction Table. \\
\hline \multicolumn{3}{|l|}{\em Required Data} \\
\hline
\texttt{startValidTime} & \texttt{ArrayTime} &
 the start time of result validity period. \\
\texttt{endValidTime} & \texttt{ArrayTime} &
 the end time of result validity period. \\
\texttt{\f$N_{Freq}\f$} (\f$N_{Freq}\f$)& \texttt{int} &
 the number of frequency ranges. \\
\texttt{\f$N_{Stok}\f$} (\f$N_{Stok}\f$)& \texttt{int} &
 the number of Stokes parameters. \\
\texttt{frequencyRanges} & \texttt{Frequency [numFrequencyRanges]  [2] } &
 the frequency ranges (one pair of values per range). \\
\texttt{fluxMethod} & \texttt{FluxCalibrationMethod} &
 identifies the flux determination method. \\
\texttt{flux} & \texttt{double [numStokes]  [numFrequencyRanges] } &
 the flux densities (one value par Stokes parameter per frequency range) expressed in Jansky (Jy). \\
\texttt{fluxError} & \texttt{double [numStokes]  [numFrequencyRanges] } &
 the uncertainties on the flux densities (one value per Stokes parameter per frequency range). \\
\texttt{stokes} & \texttt{StokesParameter [numStokes] } &
 the Stokes parameter. \\

\hline \multicolumn{3}{|l|}{\em Optional Data} \\
\hline
\texttt{direction}  & \texttt{Angle [2] } &
 the direction of the source. \\
\texttt{directionCode}  & \texttt{DirectionReferenceCode} &
 identifies the reference frame of the source's direction. \\
\texttt{directionEquinox}  & \texttt{Angle} &
 equinox associated with the reference frame of the source's direction. \\
\texttt{PA}  & \texttt{Angle [numStokes]  [numFrequencyRanges] } &
 the position's angles for the source model (one value per Stokes parameter per frequency range). \\
\texttt{PAError}  & \texttt{Angle [numStokes]  [numFrequencyRanges] } &
 the uncertainties on the position's angles (one value per Stokes parameter per frequency range). \\
\texttt{size}  & \texttt{Angle [numStokes]  [numFrequencyRanges]  [2] } &
 the sizes of the source (one pair of angles per Stokes parameter per frequency range). \\
\texttt{sizeError}  & \texttt{Angle [numStokes]  [numFrequencyRanges]  [2] } &
 the uncertainties of the sizes of the source (one pair of angles per Stokes parameter per frequency range). \\
\texttt{sourceModel}  & \texttt{SourceModel} &
 identifies the source model. \\
\hline
\end{longtable}
  
~\par\noindent{\bf Column Descriptions:}

\begin{description}
\item[\texttt{sourceName}]: The name of the source for which flux density     information was derived. 
\item[\texttt{calDataId}]: CalData Table identifier. 
\item[\texttt{calReductionId}]: CalReductionTable identifier. 
\item[\texttt{startValidTime}]: The start of the time validity range for     the result. 
\item[\texttt{endValidTime}]: The end of the time validity range for the result. 
\item[\texttt{\f$N_{Freq}\f$}]:  The number of frequency ranges for which     flux density information was derived 
\item[\texttt{\f$N_{Stok}\f$}]: The number of Stokes parameters which were     measured for this source. 
\item[\texttt{frequencyRanges}]: Frequency ranges over which the result is valid. {\\MFrequency{TOPO}} 
\item[\texttt{fluxMethod}]: The method which was used to derive flux     densities. 
\item[\texttt{flux}]: The derived flux density values expressed in Jansky (Jy). 
\item[\texttt{fluxError}]:  The statistical uncertainties of the flux     densities which were derived. 
\item[\texttt{stokes}]: The names of the Stokes parameters which were derived. 
\item[\texttt{direction}]: The reference code for    \\texttt{direction},   if not \\texttt{J2000}. 
\item[\texttt{directionCode}]: The direction to the source in celestial coordinates. 
\item[\texttt{directionEquinox}]: The reference equinox for    \\texttt{direction},   if required by \\texttt{directionCode} 
\item[\texttt{PA}]: Position angle for source model. 
\item[\texttt{PAError}]: Uncertainty on position angle for source model. 
\item[\texttt{size}]: Half power sizes of source (main axes of ellipse). 
\item[\texttt{sizeError}]: Uncertainties on half power size of source      (main axes of ellipse) 
\item[\texttt{sourceModel}]: Model used for source, e.g., point-like or Gaussian. 
\end{description}
\endgroup

 \newpage

\subsection{CalFocus Table}

 
 Result of focus calibration performed on-line by TelCal.

\begingroup
%
% define shortcuts for dimensions

%\newcommand{\numReceptor}{\f$N_{Rece}\f$}

\par\noindent\begin{longtable} {|p{45mm}|p{45mm}|p{65mm}|}
\hline \multicolumn{3}{|c|}{\textbf{CalFocus}} \\
\hline\hline
Name & Type (Shape) & Comment \\
\hline \endfirsthead
\hline \multicolumn{3}{|c|}{\textbf{CalFocus} -- continued from previous page} \\
\hline \hline
Name & Type (Shape) & Comment \\
\hline \endhead
\hline \endfoot


\hline \multicolumn{3}{|l|}{\em Key} \\
\hline 

\texttt{antennaName} & \texttt{string} &  the name of the antenna. \\
\texttt{receiverBand} & \texttt{ReceiverBand} &  identifies the receiver band. \\
\texttt{calDataId} & \texttt{Tag} &  refers to a unique row in CalData Table. \\
\texttt{calReductionId} & \texttt{Tag} &  refers to a unique row in CalReduction Table. \\
\hline \multicolumn{3}{|l|}{\em Required Data} \\
\hline
\texttt{startValidTime} & \texttt{ArrayTime} &
 the start time of the result validity period. \\
\texttt{endValidTime} & \texttt{ArrayTime} &
 the end time of the result validity period. \\
\texttt{ambientTemperature} & \texttt{Temperature} &
 the ambient temperature. \\
\texttt{atmPhaseCorrection} & \texttt{AtmPhaseCorrection} &
 qualifies how the atmospheric phase correction has been applied. \\
\texttt{focusMethod} & \texttt{FocusMethod} &
 identifies the method used during the calibration. \\
\texttt{frequencyRange} & \texttt{Frequency [2] } &
 the frequency range over which the result is valid. \\
\texttt{pointingDirection} & \texttt{Angle [2] } &
 the antenna pointing direction (horizontal coordinates). \\
\texttt{\f$N_{Rece}\f$} (\f$N_{Rece}\f$)& \texttt{int} &
 the number of receptors. \\
\texttt{polarizationTypes} & \texttt{PolarizationType [numReceptor] } &
 identifies the polarization types (one value per receptor). \\
\texttt{wereFixed} & \texttt{bool [3] } &
 coordinates were fixed (true) or not fixed (false) (one value per individual coordinate). \\
\texttt{offset} & \texttt{Length [numReceptor]  [3] } &
 the measured focus offsets in X,Y,Z (one triple of values per receptor). \\
\texttt{offsetError} & \texttt{Length [numReceptor]  [3] } &
 the statistical uncertainties on measured focus offsets (one triple per receptor). \\
\texttt{offsetWasTied} & \texttt{bool [numReceptor]  [3] } &
 focus was tied (true) or not tied (false) (one value per receptor and focus individual coordinate). \\
\texttt{reducedChiSquared} & \texttt{double [numReceptor]  [3] } &
 a measure of the quality of the fit (one triple per receptor). \\
\texttt{position} & \texttt{Length [numReceptor]  [3] } &
 the absolute focus position in X,Y,Z (one triple of values per receptor).  \\

\hline \multicolumn{3}{|l|}{\em Optional Data} \\
\hline
\texttt{polarizationsAveraged}  & \texttt{bool} &
 Polarizations were averaged. \\
\texttt{focusCurveWidth}  & \texttt{Length [numReceptor]  [3] } &
 half power width of fitted focus curve (one triple per receptor). \\
\texttt{focusCurveWidthError}  & \texttt{Length [numReceptor]  [3] } &
 Uncertainty of the focus curve width. \\
\texttt{focusCurveWasFixed}  & \texttt{bool [3] } &
 each coordinate of the focus curve width was set (true) or not set (false) to an assumed value. \\
\texttt{offIntensity}  & \texttt{Temperature [numReceptor] } &
 the off intensity levels (one value per receptor). \\
\texttt{offIntensityError}  & \texttt{Temperature [numReceptor] } &
 the uncertainties on the off intensity levels (one value per receptor). \\
\texttt{offIntensityWasFixed}  & \texttt{bool} &
 the off intensity level was fixed (true) or not fixed (false). \\
\texttt{peakIntensity}  & \texttt{Temperature [numReceptor] } &
 the maximum intensities (one value per receptor). \\
\texttt{peakIntensityError}  & \texttt{Temperature [numReceptor] } &
 the uncertainties on the maximum intensities (one value per receptor). \\
\texttt{peakIntensityWasFixed}  & \texttt{bool} &
 the maximum intensity was fixed (true) or not fixed (false). \\
\texttt{astigmPlus}  & \texttt{Length [numReceptor] } &
 the astigmatism component with 0 degree symmetry axis.  \\
\texttt{astigmPlusError}  & \texttt{Length [numReceptor] } &
 the statistical error on astigmPlus  \\
\texttt{astigmMult}  & \texttt{Length [numReceptor] } &
 the astigmatism component with 45 degrees symmetry axis.  \\
\texttt{astigmMultError}  & \texttt{Length [numReceptor] } &
 the statistical error on astigmMult  \\
\texttt{illumOffset}  & \texttt{Length [numReceptor]  [2] } &
 the illumination offset of the primary reflector expressed as a pair of values.  \\
\texttt{illumOffsetError}  & \texttt{Length [numReceptor]  [2] } &
 the statistical error on illumOffset.  \\
\texttt{fitRMS}  & \texttt{Length [numReceptor] } &
 The RMS of the half path length after removing the best fit parabola.  \\
\hline
\end{longtable}
  
~\par\noindent{\bf Column Descriptions:}

\begin{description}
\item[\texttt{antennaName}]: Refers uniquely to the hardware antenna object, as present in the original ASDM Antenna table. 
\item[\texttt{receiverBand}]: The name of the front-end frequency band being used. 
\item[\texttt{calDataId}]: CalData Table identifier. 
\item[\texttt{calReductionId}]: CalReduction Table identifier. 
\item[\texttt{startValidTime}]: The start of the time validity range for   the result. 
\item[\texttt{endValidTime}]: The end of the time validity range for the result. 
\item[\texttt{ambientTemperature}]: Ambient temperature at the   time of measurement. For mm-wave  antennas a temperature dependence of the   focus correction is expected. 
\item[\texttt{atmPhaseCorrection}]: The atmospheric phase correction states for which result is given. 
\item[\texttt{focusMethod}]: Method used, e.g., 'Interferometry' or '5 points' 
\item[\texttt{frequencyRange}]: Frequency range over which the result is valid.  {\\MFrequency{TOPO}} 
\item[\texttt{pointingDirection}]: The antenna pointing direction (horizontal    coordinates). For mm-wave  antennas an elevation dependence of the   focus correction is expected.   \\MDirection{AZEL}{NOW}{Antenna.position} 
\item[\texttt{\f$N_{Rece}\f$}]: Number of receptors. 
\item[\texttt{polarizationTypes}]: The relevant polarizations for the measured focus parameters. 
\item[\texttt{wereFixed}]: Indicates which focus coordinates were kept   fixed during measursment (and thus were not measured). 
\item[\texttt{offset}]: The measured focus offsets in X, Y, Z.   This offset is relative to the nominal position of the focus, once the   focus model has been applied. 
\item[\texttt{offsetError}]: Uncertainty of \\texttt{offset}. 
\item[\texttt{offsetWasTied}]: True for a polarization and focus     coordinate when this quantity was assumed fixed relative to the     corresponding coordinate in the other polarization. 
\item[\texttt{reducedChiSquared}]: Reduced \\$\\chi^2\\$      indicating the quality of the least-squares fit. 
\item[\texttt{position}]: {\red long doc missing}
\item[\texttt{polarizationsAveraged}]: Set when polarizations were averaged over to improve sensitivity. 
\item[\texttt{focusCurveWidth}]: Half-power width of   fitted focus curve. 
\item[\texttt{focusCurveWidthError}]: Statistical uncertainty of the half-power width of the fitted focus curve. 
\item[\texttt{focusCurveWasFixed}]: Indicates that   the half-power width of the fitted focus curvewas fixed to an assumed   value. 
\item[\texttt{offIntensity}]: Off intensity level. This is   needed for completeness to define the fitted beam function whenever  the off   level is non-zero (single-dish pointing). 
\item[\texttt{offIntensityError}]: Off intensity level   uncertainty 
\item[\texttt{offIntensityWasFixed}]: Off intensity level   was fixed. 
\item[\texttt{peakIntensity}]: Fitted maximum intensity   of signal. 
\item[\texttt{peakIntensityError}]: Statistical   uncertainty of the fitted maximum signal intensity. 
\item[\texttt{peakIntensityWasFixed}]: Indicates that the   maximal signal intensity was fixed to an assumed value. 
\item[\texttt{astigmPlus}]: {\red long doc missing}
\item[\texttt{astigmPlusError}]: {\red long doc missing}
\item[\texttt{astigmMult}]: {\red long doc missing}
\item[\texttt{astigmMultError}]: {\red long doc missing}
\item[\texttt{illumOffset}]: {\red long doc missing}
\item[\texttt{illumOffsetError}]: {\red long doc missing}
\item[\texttt{fitRMS}]: {\red long doc missing}
\end{description}
\endgroup

 \newpage

\subsection{CalFocusModel Table}

 
 Result of focus model calibration performed by TelCal.

\begingroup
%
% define shortcuts for dimensions

%\newcommand{\numCoeff}{\f$N_{Coef}\f$}
%\newcommand{\numSourceObs}{\f$N_{Sour}\f$}

\par\noindent\begin{longtable} {|p{45mm}|p{45mm}|p{65mm}|}
\hline \multicolumn{3}{|c|}{\textbf{CalFocusModel}} \\
\hline\hline
Name & Type (Shape) & Comment \\
\hline \endfirsthead
\hline \multicolumn{3}{|c|}{\textbf{CalFocusModel} -- continued from previous page} \\
\hline \hline
Name & Type (Shape) & Comment \\
\hline \endhead
\hline \endfoot


\hline \multicolumn{3}{|l|}{\em Key} \\
\hline 

\texttt{antennaName} & \texttt{string} &  the name of the antenna. \\
\texttt{receiverBand} & \texttt{ReceiverBand} &  identifies the receiver band. \\
\texttt{polarizationType} & \texttt{PolarizationType} &  identifies the polarization type for which this focus model is valid. \\
\texttt{calDataId} & \texttt{Tag} &  refers to a unique row in CalData Table. \\
\texttt{calReductionId} & \texttt{Tag} &  refers to a unique row in CalReduction Table. \\
\hline \multicolumn{3}{|l|}{\em Required Data} \\
\hline
\texttt{startValidTime} & \texttt{ArrayTime} &
 the start time of result validity period. \\
\texttt{endValidTime} & \texttt{ArrayTime} &
 the end time of result validity period. \\
\texttt{antennaMake} & \texttt{AntennaMake} &
 identifies the antenna make. \\
\texttt{\f$N_{Coef}\f$} (\f$N_{Coef}\f$)& \texttt{int} &
 the number of coefficients. \\
\texttt{\f$N_{Sour}\f$} (\f$N_{Sour}\f$)& \texttt{int} &
 the number of source directions observed to derive the model. \\
\texttt{coeffName} & \texttt{string [numCoeff] } &
 the names given to  the coefficients in the model. \\
\texttt{coeffFormula} & \texttt{string [numCoeff] } &
 the coefficients formula (one string per coefficient). \\
\texttt{coeffValue} & \texttt{float [numCoeff] } &
 the fitted values of the coefficients. \\
\texttt{coeffError} & \texttt{float [numCoeff] } &
 the statistical uncertainties on the derived coefficients (one value per coefficient). \\
\texttt{coeffFixed} & \texttt{bool [numCoeff] } &
 one coefficient was fixed (true) or not fixed (false) (one boolean value per coefficient). \\
\texttt{focusModel} & \texttt{string} &
 the name of this focus model. \\
\texttt{focusRMS} & \texttt{Length [3] } &
 the RMS deviations of residuals of focus coordinates. \\
\texttt{reducedChiSquared} & \texttt{double} &
 a measure of the quality of the least-square fit. \\

\hline
\end{longtable}
  
~\par\noindent{\bf Column Descriptions:}

\begin{description}
\item[\texttt{antennaName}]:  Refers uniquely to the hardware antenna object, as present in the original ASDM Antenna table. 
\item[\texttt{receiverBand}]: The name of the front-end frequency band being used. 
\item[\texttt{polarizationType}]: Polarization component for which the focus model is valid. 
\item[\texttt{calDataId}]: CalData Table identifier. 
\item[\texttt{calReductionId}]: CalReduction Table identifier. 
\item[\texttt{startValidTime}]: The start of the time validity range for the result. 
\item[\texttt{endValidTime}]: The end of the time validity range for the result. 
\item[\texttt{antennaMake}]: The antenna make (e.g., for ALMA, the manufacturer name such as AEC, Vertex, or Melco). 
\item[\texttt{\f$N_{Coef}\f$}]: Number of coefficients in the focus model. 
\item[\texttt{\f$N_{Sour}\f$}]: Number of source directions   observed to derive the model. 
\item[\texttt{coeffName}]: The {given} names of the coefficients   in the model. 
\item[\texttt{coeffFormula}]: The formula describing the fitted functional   dependence for the focus coordinate. 
\item[\texttt{coeffValue}]: The fitted value for the coefficient. 
\item[\texttt{coeffError}]: The statistical uncertainty on the derived coefficients. 
\item[\texttt{coeffFixed}]: A boolean specifying that the coefficient was fixed to an assumed value. 
\item[\texttt{focusModel}]: Name of this focus model. 
\item[\texttt{focusRMS}]: The root mean square deviation of residuals in focus coordinates.
\item[\texttt{reducedChiSquared}]: Reduced \\$\\chi^2\\$      indicating the quality of the least-squares fit. 
\end{description}
\endgroup

 \newpage

\subsection{CalGain Table}

 
 This Table is a placeholder to be used to wrap up casa gain tables produced  in the Science Pipeline and Offline so that they can be archived in the  ALMA Calibration Data Base.

\begingroup
%
% define shortcuts for dimensions


\par\noindent\begin{longtable} {|p{45mm}|p{45mm}|p{65mm}|}
\hline \multicolumn{3}{|c|}{\textbf{CalGain}} \\
\hline\hline
Name & Type (Shape) & Comment \\
\hline \endfirsthead
\hline \multicolumn{3}{|c|}{\textbf{CalGain} -- continued from previous page} \\
\hline \hline
Name & Type (Shape) & Comment \\
\hline \endhead
\hline \endfoot


\hline \multicolumn{3}{|l|}{\em Key} \\
\hline 

\texttt{calDataId} & \texttt{Tag} &  refers to a unique row in CalData Table. \\
\texttt{calReductionId} & \texttt{Tag} &  refers to a unique row  in CalReductionTable. \\
\hline \multicolumn{3}{|l|}{\em Required Data} \\
\hline
\texttt{startValidTime} & \texttt{ArrayTime} &
  the start time of result validity period. \\
\texttt{endValidTime} & \texttt{ArrayTime} &
  the end time of result validity period. \\
\texttt{gain} & \texttt{float} &
 TBD \\
\texttt{gainValid} & \texttt{bool} &
 TBD \\
\texttt{fit} & \texttt{float} &
 TBD \\
\texttt{fitWeight} & \texttt{float} &
 TBD \\
\texttt{totalGainValid} & \texttt{bool} &
 TBD \\
\texttt{totalFit} & \texttt{float} &
 TBD \\
\texttt{totalFitWeight} & \texttt{float} &
 TBD \\

\hline
\end{longtable}
  
~\par\noindent{\bf Column Descriptions:}

\begin{description}
\item[\texttt{calDataId}]: CalData Table identifier. 
\item[\texttt{calReductionId}]: CalReduction Table identifier. 
\item[\texttt{startValidTime}]: The start of the time validity range for the result. 
\item[\texttt{endValidTime}]: The end of the time validity range for the result. 
\item[\texttt{gain}]: TBD 
\item[\texttt{gainValid}]: TBD 
\item[\texttt{fit}]: TBD 
\item[\texttt{fitWeight}]: TBD 
\item[\texttt{totalGainValid}]: TBD 
\item[\texttt{totalFit}]: TBD 
\item[\texttt{totalFitWeight}]: TBD 
\end{description}
\endgroup

 \newpage

\subsection{CalHolography Table}

 
 Result of holography calibration performed by TelCal.

\begingroup
%
% define shortcuts for dimensions

%\newcommand{\numReceptor}{\f$N_{Rece}\f$}
%\newcommand{\numPanelModes}{\f$N_{Pane}\f$}
%\newcommand{\numScrew}{\f$N_{Scre}\f$}

\par\noindent\begin{longtable} {|p{45mm}|p{45mm}|p{65mm}|}
\hline \multicolumn{3}{|c|}{\textbf{CalHolography}} \\
\hline\hline
Name & Type (Shape) & Comment \\
\hline \endfirsthead
\hline \multicolumn{3}{|c|}{\textbf{CalHolography} -- continued from previous page} \\
\hline \hline
Name & Type (Shape) & Comment \\
\hline \endhead
\hline \endfoot


\hline \multicolumn{3}{|l|}{\em Key} \\
\hline 

\texttt{antennaName} & \texttt{string} &  the name of the antenna. \\
\texttt{calDataId} & \texttt{Tag} &  refers to a unique row in CalData Table. \\
\texttt{calReductionId} & \texttt{Tag} &  refers to a unique row in CalReduction Table. \\
\hline \multicolumn{3}{|l|}{\em Required Data} \\
\hline
\texttt{antennaMake} & \texttt{AntennaMake} &
 identifies the antenna make. \\
\texttt{startValidTime} & \texttt{ArrayTime} &
 Start time of result validity period \\
\texttt{endValidTime} & \texttt{ArrayTime} &
 the end time of result validity period. \\
\texttt{ambientTemperature} & \texttt{Temperature} &
 the ambient temperature. \\
\texttt{focusPosition} & \texttt{Length [3] } &
 the focus position. \\
\texttt{frequencyRange} & \texttt{Frequency [2] } &
 the range of frequencies for which the measurement is valid. \\
\texttt{illuminationTaper} & \texttt{double} &
 the amplitude illumination taper. \\
\texttt{\f$N_{Rece}\f$} (\f$N_{Rece}\f$)& \texttt{int} &
 the number of receptors. \\
\texttt{polarizationTypes} & \texttt{PolarizationType [numReceptor] } &
 identifies the polarization types (one value per receptor). \\
\texttt{\f$N_{Pane}\f$} (\f$N_{Pane}\f$)& \texttt{int} &
 the number panel modes fitted. \\
\texttt{receiverBand} & \texttt{ReceiverBand} &
 identifies the receiver band. \\
\texttt{beamMapUID} & \texttt{EntityRef} &
 refers to the beam map image. \\
\texttt{rawRMS} & \texttt{Length} &
 the RMS of the pathlength residuals. \\
\texttt{weightedRMS} & \texttt{Length} &
 the weigthted RMS of the pathlength residuals. \\
\texttt{surfaceMapUID} & \texttt{EntityRef} &
 refers to the resulting antenna surface map image. \\
\texttt{direction} & \texttt{Angle [2] } &
 the direction of the source. \\

\hline \multicolumn{3}{|l|}{\em Optional Data} \\
\hline
\texttt{\f$N_{Scre}\f$} (\f$N_{Scre}\f$) & \texttt{int} &
 the number of screws. \\
\texttt{screwName}  & \texttt{string [numScrew] } &
 the names of the screws (one value per screw). \\
\texttt{screwMotion}  & \texttt{Length [numScrew] } &
 the prescribed screw motions (one value per screw). \\
\texttt{screwMotionError}  & \texttt{Length [numScrew] } &
 the uncertainties on the prescribed screw  motions (one value per screw). \\
\texttt{gravCorrection}  & \texttt{bool} &
 indicates if a gravitational correction was applied (true) or not (false). \\
\texttt{gravOptRange}  & \texttt{Angle [2] } &
 the range of gravitational optimization. \\
\texttt{tempCorrection}  & \texttt{bool} &
 indicates if a temperature correction was applied (true) or not (false). \\
\texttt{tempOptRange}  & \texttt{Temperature [2] } &
 the range of temperature optimization. \\
\hline
\end{longtable}
  
~\par\noindent{\bf Column Descriptions:}

\begin{description}
\item[\texttt{antennaName}]: Refers uniquely to the hardware antenna   object, as present in the original ASDM Antenna table. 
\item[\texttt{calDataId}]: CalData Table identifier. 
\item[\texttt{calReductionId}]: CalReduction Table identifier. 
\item[\texttt{antennaMake}]: The antenna make (e.g., for ALMA, the manufacturer name such as AEC, Vertex, or Melco). 
\item[\texttt{startValidTime}]: The start of the time validity range for   the result. 
\item[\texttt{endValidTime}]: The end of the time validity range for the result. 
\item[\texttt{ambientTemperature}]: Ambient temperature at the   time of measurement. The surface deformations are  expected to depend   on temperature. 
\item[\texttt{focusPosition}]: The optimal focus position (in XYZ) as   derived from the aperture map phases.   \\MPosition{REFLECTOR}  
\item[\texttt{frequencyRange}]: Frequency range over which the result is valid.   \\MFrequency{TOPO} 
\item[\texttt{illuminationTaper}]: Power illumination     taper assumed to calculate weighted rms. 
\item[\texttt{\f$N_{Rece}\f$}]: The number or polarization receptors (one or two) for which the result is given. 
\item[\texttt{polarizationTypes}]: The polarization types of the receptors being used. 
\item[\texttt{\f$N_{Pane}\f$}]: The number of panel independent position/deformation modes that have been fitted. 
\item[\texttt{receiverBand}]: The name of the front-end frequency band being used. 
\item[\texttt{beamMapUID}]: The beam map UID provides a link to the     resulting beam map image either as a disk file or in the ALMA     Archive. The disk file name is built from the UID string by replacing     all colons and slashes by underscores. 
\item[\texttt{rawRMS}]: The root mean square of the pathlength residuals,   measured along Z, that is perpendicular to the aperture plane   when looking a source at infinite distance. 
\item[\texttt{weightedRMS}]: The root mean square of pathlength residuals     (along Z); weighted assuming a primary beam illumination as specified by     \\texttt{illuminationTaper}. 
\item[\texttt{surfaceMapUID}]: The surface map UID provides a link to the     resulting antenna surface map image either as a disk file or in the ALMA     Archive. The disk file name is built from the UID string by replacing     all colons and slashes by underscores. 
\item[\texttt{direction}]: The antenna pointing direction (horizontal coordinates)   \\MDirection{AZEL}{NOW}{Antenna..position} 
\item[\texttt{\f$N_{Scre}\f$}]: Number of screws to be adjusted using surface map data. 
\item[\texttt{screwName}]: The string identification of the panel screws. 
\item[\texttt{screwMotion}]: The prescribed panel screw adjustments   derived from the panel fit to the aperture map phases. 
\item[\texttt{screwMotionError}]:  The statistical uncertainties on the   prescribed panel screw adjustments derived from the panel fit to the   aperture map phases. 
\item[\texttt{gravCorrection}]: Optimization target elevation     range for the gravitaionnal correction applied. 
\item[\texttt{gravOptRange}]: Optimization target elevation     range for the gravitaionnal correction applied. 
\item[\texttt{tempCorrection}]: A temperature correction was     applied in the screw motion data. 
\item[\texttt{tempOptRange}]: Optimization target temperature  range. 
\end{description}
\endgroup

 \newpage

\subsection{CalPhase Table}

 
 Result of the phase calibration performed by TelCal.

\begingroup
%
% define shortcuts for dimensions

%\newcommand{\numBaseline}{\f$N_{Base}\f$}
%\newcommand{\numReceptor}{\f$N_{Rece}\f$}

\par\noindent\begin{longtable} {|p{45mm}|p{45mm}|p{65mm}|}
\hline \multicolumn{3}{|c|}{\textbf{CalPhase}} \\
\hline\hline
Name & Type (Shape) & Comment \\
\hline \endfirsthead
\hline \multicolumn{3}{|c|}{\textbf{CalPhase} -- continued from previous page} \\
\hline \hline
Name & Type (Shape) & Comment \\
\hline \endhead
\hline \endfoot


\hline \multicolumn{3}{|l|}{\em Key} \\
\hline 

\texttt{basebandName} & \texttt{BasebandName} &  identifies the baseband. \\
\texttt{receiverBand} & \texttt{ReceiverBand} &  identifies the receiver band. \\
\texttt{atmPhaseCorrection} & \texttt{AtmPhaseCorrection} &   describes how the atmospheric phase correction has been applied. \\
\texttt{calDataId} & \texttt{Tag} &  refers to a unique row in CalData Table. \\
\texttt{calReductionId} & \texttt{Tag} &  refers to a unique row in CalReduction Table. \\
\hline \multicolumn{3}{|l|}{\em Required Data} \\
\hline
\texttt{startValidTime} & \texttt{ArrayTime} &
 the start time of result validity period. \\
\texttt{endValidTime} & \texttt{ArrayTime} &
 the end time of result validity period. \\
\texttt{\f$N_{Base}\f$} (\f$N_{Base}\f$)& \texttt{int} &
 the number of baselines. \\
\texttt{\f$N_{Rece}\f$} (\f$N_{Rece}\f$)& \texttt{int} &
 the number of receptors. \\
\texttt{ampli} & \texttt{float [numReceptor]  [numBaseline] } &
 the amplitudes (one value per receptor per baseline). \\
\texttt{antennaNames} & \texttt{string [numBaseline]  [2] } &
 the names of the antennas (one pair of strings per baseline). \\
\texttt{baselineLengths} & \texttt{Length [numBaseline] } &
 the physical lengths of the baselines (one value per baseline). \\
\texttt{decorrelationFactor} & \texttt{float [numReceptor]  [numBaseline] } &
 the decorrelation factors (one value per receptor per baseline). \\
\texttt{direction} & \texttt{Angle [2] } &
 the direction of the source. \\
\texttt{frequencyRange} & \texttt{Frequency [2] } &
 the frequency range over which the result is valid. \\
\texttt{integrationTime} & \texttt{Interval} &
 the integration duration for a data point. \\
\texttt{phase} & \texttt{float [numReceptor]  [numBaseline] } &
 the phases of the averaged interferometer signal (one value per receptor per baseline). \\
\texttt{polarizationTypes} & \texttt{PolarizationType [numReceptor] } &
 identifies the polarization types of the receptors (one value per receptor). \\
\texttt{phaseRMS} & \texttt{float [numReceptor]  [numBaseline] } &
 the RMS of phase fluctuations relative to the average signal (one value per receptor per baseline). \\
\texttt{statPhaseRMS} & \texttt{float [numReceptor]  [numBaseline] } &
 the RMS of phase deviations expected from the thermal fluctuations (one value per receptor per baseline). \\

\hline \multicolumn{3}{|l|}{\em Optional Data} \\
\hline
\texttt{correctionValidity}  & \texttt{bool [numBaseline] } &
 the deduced validity of atmospheric path length correction (from water vapor radiometers). \\
\hline
\end{longtable}
  
~\par\noindent{\bf Column Descriptions:}

\begin{description}
\item[\texttt{basebandName}]: The name of the 'baseband pair' which is measured. For ALMA, a baseband pair is the signal path identified by a second local oscillator and has two polarizations. 
\item[\texttt{receiverBand}]: The name of the front-end frequency band being used. 
\item[\texttt{atmPhaseCorrection}]: The atmospheric phase corrections states for which result is given. 
\item[\texttt{calDataId}]: CalData Table identifier. 
\item[\texttt{calReductionId}]: CalReduction Table identifier. 
\item[\texttt{startValidTime}]: The start of the time validity range for the result. 
\item[\texttt{endValidTime}]: The end of the time validity range for the result. 
\item[\texttt{\f$N_{Base}\f$}]: Number of baselines for which the result is given. 
\item[\texttt{\f$N_{Rece}\f$}]: The number or polarization receptors (one or two) for which the result is given. 
\item[\texttt{ampli}]: Amplitude of averaged signal. 
\item[\texttt{antennaNames}]: Refer uniquely to the hardware antenna object, as present in the original ASDM Antenna table. 
\item[\texttt{baselineLengths}]: The physical length of each baseline. 
\item[\texttt{decorrelationFactor}]: The calculated decorrelation factor (amplitude loss) due to non-thermal phase fluctuations. 
\item[\texttt{direction}]: The antenna pointing direction in horizontal coordinates.   \\MDirection{AZEL}{NOW}{Antenna.position} 
\item[\texttt{frequencyRange}]: Frequency range over which the result is valid   \\MFrequency{TOPO} 
\item[\texttt{integrationTime}]: Integration time on a   data point, to calculate rms. 
\item[\texttt{phase}]: The phase of the averaged interferometer signal. 
\item[\texttt{polarizationTypes}]: The polarization types of the receptors being used. 
\item[\texttt{phaseRMS}]: The root mean square of phase fluctuations relative to the average signal. 
\item[\texttt{statPhaseRMS}]: The root mean square of phase deviations expected from thermal fluctuations. 
\item[\texttt{correctionValidity}]: Deduced validity of atmospheric path length correction (from Water Vapour Radiometers; remark: It is not clear that correctionValidity is really an array. What about its size?). 
\end{description}
\endgroup

 \newpage

\subsection{CalPointing Table}

 
 Result of the pointing calibration performed on-line by TelCal.

\begingroup
%
% define shortcuts for dimensions

%\newcommand{\numReceptor}{\f$N_{Rece}\f$}

\par\noindent\begin{longtable} {|p{45mm}|p{45mm}|p{65mm}|}
\hline \multicolumn{3}{|c|}{\textbf{CalPointing}} \\
\hline\hline
Name & Type (Shape) & Comment \\
\hline \endfirsthead
\hline \multicolumn{3}{|c|}{\textbf{CalPointing} -- continued from previous page} \\
\hline \hline
Name & Type (Shape) & Comment \\
\hline \endhead
\hline \endfoot


\hline \multicolumn{3}{|l|}{\em Key} \\
\hline 

\texttt{antennaName} & \texttt{string} &  Antenna Name \\
\texttt{receiverBand} & \texttt{ReceiverBand} &  identifies the receiver band. \\
\texttt{calDataId} & \texttt{Tag} &  refers to a unique row in CalData Table. \\
\texttt{calReductionId} & \texttt{Tag} &  refers to a unique row in CalReduction Table. \\
\hline \multicolumn{3}{|l|}{\em Required Data} \\
\hline
\texttt{startValidTime} & \texttt{ArrayTime} &
 the start time of result validity period. \\
\texttt{endValidTime} & \texttt{ArrayTime} &
 the end time of result validity period. \\
\texttt{ambientTemperature} & \texttt{Temperature} &
 the ambient temperature. \\
\texttt{antennaMake} & \texttt{AntennaMake} &
 identifies the antenna make. \\
\texttt{atmPhaseCorrection} & \texttt{AtmPhaseCorrection} &
 describes how the atmospheric phase correction has been applied. \\
\texttt{direction} & \texttt{Angle [2] } &
 the antenna pointing direction. \\
\texttt{frequencyRange} & \texttt{Frequency [2] } &
 the frequency range over which the result is valid. \\
\texttt{pointingModelMode} & \texttt{PointingModelMode} &
 identifies the pointing model mode. \\
\texttt{pointingMethod} & \texttt{PointingMethod} &
 identifies the pointing method. \\
\texttt{\f$N_{Rece}\f$} (\f$N_{Rece}\f$)& \texttt{int} &
 the number of receptors. \\
\texttt{polarizationTypes} & \texttt{PolarizationType [numReceptor] } &
 identifies the polarizations types (one value per receptor). \\
\texttt{collOffsetRelative} & \texttt{Angle [numReceptor]  [2] } &
 the collimation offsets (relative) (one pair of angles  per receptor). \\
\texttt{collOffsetAbsolute} & \texttt{Angle [numReceptor]  [2] } &
 the collimation offsets (absolute) (one pair of angles per receptor). \\
\texttt{collError} & \texttt{Angle [numReceptor]  [2] } &
 the uncertainties on collimation (one pair of angles per receptor) \\
\texttt{collOffsetTied} & \texttt{bool [numReceptor]  [2] } &
 indicates if a collimation offset was tied (true) or not tied (false) to another polar (one pair of boolean values per receptor). \\
\texttt{reducedChiSquared} & \texttt{double [numReceptor] } &
 a measure of the quality of the least square fit. \\

\hline \multicolumn{3}{|l|}{\em Optional Data} \\
\hline
\texttt{averagedPolarizations}  & \texttt{bool} &
 true when the polarizations were averaged together to improve sensitivity. \\
\texttt{beamPA}  & \texttt{Angle [numReceptor] } &
 the fitted beam position angles (one value per receptor). \\
\texttt{beamPAError}  & \texttt{Angle [numReceptor] } &
 the uncertaintes on the fitted beam position angles (one value per receptor). \\
\texttt{beamPAWasFixed}  & \texttt{bool} &
 indicates if the beam position was fixed (true) or not fixed (false). \\
\texttt{beamWidth}  & \texttt{Angle [numReceptor]  [2] } &
 the fitted beam widths (one pair of angles per receptor). \\
\texttt{beamWidthError}  & \texttt{Angle [numReceptor]  [2] } &
 the uncertainties on the fitted beam widths (one pair of angles per receptor). \\
\texttt{beamWidthWasFixed}  & \texttt{bool [2] } &
 indicates if the beam width was fixed (true) or not fixed (true) (one pair of booleans). \\
\texttt{offIntensity}  & \texttt{Temperature [numReceptor] } &
 the off intensity levels (one value per receptor). \\
\texttt{offIntensityError}  & \texttt{Temperature [numReceptor] } &
 the uncertainties on the off intensity levels (one value per receptor). \\
\texttt{offIntensityWasFixed}  & \texttt{bool} &
 indicates if the off intensity level was fixed (true) or not fixed (false). \\
\texttt{peakIntensity}  & \texttt{Temperature [numReceptor] } &
 the maximum intensities (one value per receptor). \\
\texttt{peakIntensityError}  & \texttt{Temperature [numReceptor] } &
 the uncertainties on the maximum intensities (one value per receptor). \\
\texttt{peakIntensityWasFixed}  & \texttt{bool} &
 the maximum intensity was fixed. \\
\hline
\end{longtable}
  
~\par\noindent{\bf Column Descriptions:}

\begin{description}
\item[\texttt{antennaName}]: Refers uniquely to the hardware antenna object as present in the original ASDM Antenna table. 
\item[\texttt{receiverBand}]: The name of the front-end frequency band being used. 
\item[\texttt{calDataId}]: CalData Table identifier. 
\item[\texttt{calReductionId}]: CalReduction Table identifier. 
\item[\texttt{startValidTime}]: The start of the time validity range for the result. 
\item[\texttt{endValidTime}]: The end of the time validity range for the   result. 
\item[\texttt{ambientTemperature}]: Ambient temperature at the   time of measurement. For mm-wave antennas, a temperature   dependence of the pointing correction may be  expected. 
\item[\texttt{antennaMake}]: The antenna make (e.g., for ALMA, the antenna manufacturer name such as AEC, Vertex, or Melco). 
\item[\texttt{atmPhaseCorrection}]: The atmospheric phase correction states for which result is given. 
\item[\texttt{direction}]: The antenna pointing direction (horizontal coordinates)   \\MDirection{AZEL}{NOW}{Antenna.position} 
\item[\texttt{frequencyRange}]: Frequency range over which the result is valid {\\MFrequency{TOPO}} 
\item[\texttt{pointingModelMode}]: Radio pointing or Optical pointing. 
\item[\texttt{pointingMethod}]: Observing method used to determine the collimation offsets. 
\item[\texttt{\f$N_{Rece}\f$}]: Number of receptors. 
\item[\texttt{polarizationTypes}]: The relevant polarizations for the measured pointing  parameters. 
\item[\texttt{collOffsetRelative}]: The collimation offsets found required   to center the source, relative to the expected direction of the source,   using a predetermined pointing model. These are the collimation offsets to   be applied for reference pointing, or for a more refined local pointing   model.     \\MDirectionOffset{AZEL}{NOW}{Antenna.position}{[virtual]} 
\item[\texttt{collOffsetAbsolute}]: The collimation offsets found required   to center the source, relative to the expected direction of the source   assuming a perfect antenna mount. These are collimation offsets to be used   for determination of the pointing model.   \\MDirectionOffset{AZEL}{NOW}{Antenna.position}{target} 
\item[\texttt{collError}]: Statistical uncertainties in the determination of azimuth and elevation collimations. 
\item[\texttt{collOffsetTied}]: True for a polarization     coordinate when this quantity was assumed fixed relative to the     corresponding coordinate in the other polarization. 
\item[\texttt{reducedChiSquared}]: Reduced \\$\\chi^2\\$      indicating the quality of the least-squares fit. 
\item[\texttt{averagedPolarizations}]: Set when   polarizations were averaged together to improve sensitivity. 
\item[\texttt{beamPA}]: Position angle of fitted antenna beam. 
\item[\texttt{beamPAError}]: Statistical uncertainty of position angle of fitted antenna beam. 
\item[\texttt{beamPAWasFixed}]: Indicates that the  position angle of the fitted antenna beam was fixed to an assumed value. 
\item[\texttt{beamWidth}]: Half-power width of fitted antenna beam. 
\item[\texttt{beamWidthError}]: Statistical uncertainty of the half-power width of the fitted antenna beam. 
\item[\texttt{beamWidthWasFixed}]: Indicates that the  half-power width of antenna beam was fixed to an assumed value. 
\item[\texttt{offIntensity}]: Off intensity level. This is   needed for completeness to define the fitted beam function whenever  the off   level is non-zero (single-dish pointing). 
\item[\texttt{offIntensityError}]: Off intensity level   uncertainty. 
\item[\texttt{offIntensityWasFixed}]: Off intensity level   was fixed. 
\item[\texttt{peakIntensity}]: Fitted maximum intensity of signal. 
\item[\texttt{peakIntensityError}]: Statistical uncertainty of the fitted maximum signal intensity. 
\item[\texttt{peakIntensityWasFixed}]: Indicates that the  maximal signal intensity  was fixed to an assumed value. 
\end{description}
\endgroup

 \newpage

\subsection{CalPointingModel Table}

 
 Result of pointing model calibration performed by TelCal.

\begingroup
%
% define shortcuts for dimensions

%\newcommand{\numCoeff}{\f$N_{Coef}\f$}
%\newcommand{\numObs}{\f$N_{Obs}\f$}

\par\noindent\begin{longtable} {|p{45mm}|p{45mm}|p{65mm}|}
\hline \multicolumn{3}{|c|}{\textbf{CalPointingModel}} \\
\hline\hline
Name & Type (Shape) & Comment \\
\hline \endfirsthead
\hline \multicolumn{3}{|c|}{\textbf{CalPointingModel} -- continued from previous page} \\
\hline \hline
Name & Type (Shape) & Comment \\
\hline \endhead
\hline \endfoot


\hline \multicolumn{3}{|l|}{\em Key} \\
\hline 

\texttt{antennaName} & \texttt{string} &  the name of the antenna. \\
\texttt{receiverBand} & \texttt{ReceiverBand} &  identifies the receiver band. \\
\texttt{calDataId} & \texttt{Tag} &  refers to a unique row in CalData Table. \\
\texttt{calReductionId} & \texttt{Tag} &  refers to a unique row in CalReduction Table. \\
\hline \multicolumn{3}{|l|}{\em Required Data} \\
\hline
\texttt{startValidTime} & \texttt{ArrayTime} &
 the start time of result validity period. \\
\texttt{endValidTime} & \texttt{ArrayTime} &
 the end time of result validity period. \\
\texttt{antennaMake} & \texttt{AntennaMake} &
 the antenna make. \\
\texttt{pointingModelMode} & \texttt{PointingModelMode} &
 identifies the pointing model mode. \\
\texttt{polarizationType} & \texttt{PolarizationType} &
 identifies the polarization type. \\
\texttt{\f$N_{Coef}\f$} (\f$N_{Coef}\f$)& \texttt{int} &
 the number of coefficients in the pointing model. \\
\texttt{coeffName} & \texttt{string [numCoeff] } &
 the names of the coefficients (one string per coefficient). \\
\texttt{coeffVal} & \texttt{float [numCoeff] } &
 the values of the coefficients resulting from the pointing model fitting (one value per coefficient). \\
\texttt{coeffError} & \texttt{float [numCoeff] } &
 the uncertainties on the pointing model coefficients (one value per coefficient). \\
\texttt{coeffFixed} & \texttt{bool [numCoeff] } &
 indicates if one coefficient was fixed (true) or not fixed (false) (one boolean per coefficient). \\
\texttt{azimuthRMS} & \texttt{Angle} &
 Azimuth RMS (on Sky) \\
\texttt{elevationRms} & \texttt{Angle} &
 Elevation rms (on Sky) \\
\texttt{skyRMS} & \texttt{Angle} &
 rms on sky \\
\texttt{reducedChiSquared} & \texttt{double} &
 measures the quality of the least square fit. \\

\hline \multicolumn{3}{|l|}{\em Optional Data} \\
\hline
\texttt{\f$N_{Obs}\f$} (\f$N_{Obs}\f$) & \texttt{int} &
 the number of source directions observed to derive the pointing model. \\
\texttt{coeffFormula}  & \texttt{string [numCoeff] } &
 formulas used for the fitting (one string per coefficient). \\
\hline
\end{longtable}
  
~\par\noindent{\bf Column Descriptions:}

\begin{description}
\item[\texttt{antennaName}]: Refers uniquely to the hardware antenna object, as present in the original ASDM Antenna table. 
\item[\texttt{receiverBand}]: The name of the front-end frequency band being used. 
\item[\texttt{calDataId}]: CalData Table identifier. 
\item[\texttt{calReductionId}]: CalReduction Table identifier. 
\item[\texttt{startValidTime}]: The start of the time validity range for the result. 
\item[\texttt{endValidTime}]: The end of the time validity range for the result. 
\item[\texttt{antennaMake}]: The antenna make (e.g., for ALMA, the manufaturer name such as AEC, Vertex, or Melco). 
\item[\texttt{pointingModelMode}]: Pointing Model mode (Radio or optical) 
\item[\texttt{polarizationType}]: Polarization component for which the pointing model is valid. 
\item[\texttt{\f$N_{Coef}\f$}]: The number of coefficients in the pointing model. 
\item[\texttt{coeffName}]: The names of the coefficients, following   \\texttt{tpoint} software conventions (generic    functions, see \\texttt{tpoint} software documentation by P. Wallace). 
\item[\texttt{coeffVal}]: The fitted pointing model coefficients. 
\item[\texttt{coeffError}]: Statistical uncertainties of pointing model coefficients. 
\item[\texttt{coeffFixed}]: Indicates that the coefficient was kept fixed   to an assumed value. 
\item[\texttt{azimuthRMS}]: Root mean squared of azimuth residuals (as a true angle on the sky). 
\item[\texttt{elevationRms}]: Root mean of squared elevation residuals (as a true angle on the sky). 
\item[\texttt{skyRMS}]: Root mean squared of angular distance deviations. 
\item[\texttt{reducedChiSquared}]: Reduced \\$\\chi^2\\$      indicating the quality of the least-squares fit. 
\item[\texttt{\f$N_{Obs}\f$}]: The number of source directions   observed used to derive the pointing model. 
\item[\texttt{coeffFormula}]: Formula used. This   describes the functions fitted, for the corresponding coefficient. This is   useful when \\texttt{tpoint} software has not been used. 
\end{description}
\endgroup

 \newpage

\subsection{CalPosition Table}

 
 Result of antenna positions calibration performed by TelCal.

\begingroup
%
% define shortcuts for dimensions

%\newcommand{\numAntenna}{\f$N_{Ante}\f$}

\par\noindent\begin{longtable} {|p{45mm}|p{45mm}|p{65mm}|}
\hline \multicolumn{3}{|c|}{\textbf{CalPosition}} \\
\hline\hline
Name & Type (Shape) & Comment \\
\hline \endfirsthead
\hline \multicolumn{3}{|c|}{\textbf{CalPosition} -- continued from previous page} \\
\hline \hline
Name & Type (Shape) & Comment \\
\hline \endhead
\hline \endfoot


\hline \multicolumn{3}{|l|}{\em Key} \\
\hline 

\texttt{antennaName} & \texttt{string} &  the name of the antenna. \\
\texttt{atmPhaseCorrection} & \texttt{AtmPhaseCorrection} &  describes how the atmospheric phase correction has been applied. \\
\texttt{calDataId} & \texttt{Tag} &  refers to a unique row in CalData Table. \\
\texttt{calReductionId} & \texttt{Tag} &  refers to a unique row in CalReduction Table. \\
\hline \multicolumn{3}{|l|}{\em Required Data} \\
\hline
\texttt{startValidTime} & \texttt{ArrayTime} &
 the start time of result validity period. \\
\texttt{endValidTime} & \texttt{ArrayTime} &
 the end time of result validity period. \\
\texttt{antennaPosition} & \texttt{Length [3] } &
 the position of the antenna. \\
\texttt{stationName} & \texttt{string} &
 the name of the station. \\
\texttt{stationPosition} & \texttt{Length [3] } &
 the position of the station. \\
\texttt{positionMethod} & \texttt{PositionMethod} &
 identifies the method used for the position calibration. \\
\texttt{receiverBand} & \texttt{ReceiverBand} &
 identifies the receiver band. \\
\texttt{\f$N_{Ante}\f$} (\f$N_{Ante}\f$)& \texttt{int} &
 the number of antennas of reference. \\
\texttt{refAntennaNames} & \texttt{string [numAntenna] } &
 the names of the antennas of reference (one string per antenna). \\
\texttt{axesOffset} & \texttt{Length} &
 the measured axe's offset. \\
\texttt{axesOffsetErr} & \texttt{Length} &
 the uncertainty on the determination of the axe's offset. \\
\texttt{axesOffsetFixed} & \texttt{bool} &
 the axe's offset was fixed (true) or not fixed (false). \\
\texttt{positionOffset} & \texttt{Length [3] } &
 the measured position offsets (a triple). \\
\texttt{positionErr} & \texttt{Length [3] } &
 the uncertainties on the measured position offsets (a triple). \\
\texttt{reducedChiSquared} & \texttt{double} &
 measures the quality of the fit. \\

\hline \multicolumn{3}{|l|}{\em Optional Data} \\
\hline
\texttt{delayRms}  & \texttt{double} &
 the RMS deviation for the observed delays. \\
\texttt{phaseRms}  & \texttt{Angle} &
 the RMS deviation for the observed phases. \\
\hline
\end{longtable}
  
~\par\noindent{\bf Column Descriptions:}

\begin{description}
\item[\texttt{antennaName}]: Refers uniquely to the hardware antenna object, as present in the original ASDM Antenna table. 
\item[\texttt{atmPhaseCorrection}]: The atmospheric phase correction states for which result is given. 
\item[\texttt{calDataId}]: CalData Table identifier. 
\item[\texttt{calReductionId}]: CalReduction Table identifier. 
\item[\texttt{startValidTime}]: The start of the time validity range for the result. 
\item[\texttt{endValidTime}]: The end of the time validity range for the   result. 
\item[\texttt{antennaPosition}]: The antenna position measured values in the X, Y, Z   horizontal system, relative to the station.    \\MPositionOffset{AZEL}{Station.position} 
\item[\texttt{stationName}]: The name of the station where the antenna was set. 
\item[\texttt{stationPosition}]: The station position in the X, Y, Z   geocentric system. These are included as references for    \\texttt{stationPosition}. 
\item[\texttt{positionMethod}]: Position measurement method used (fit to   delays or fit to phases). 
\item[\texttt{receiverBand}]: The name of the front-end frequency band being used. 
\item[\texttt{\f$N_{Ante}\f$}]: The number of antennas used as reference for   the antenna with unknown position. 
\item[\texttt{refAntennaNames}]: The names of the antennas used as reference to get the antenna unknown position. 
\item[\texttt{axesOffset}]: Measured offsets between azimuth and elevation   axes. This is the horizontal component perpendicular to the elevation   axis, counted positive in the direction where the antenna is pointed at,   when horizon-looking. 
\item[\texttt{axesOffsetErr}]: Statistical uncertainties of   measured offsets between azimuth and elevation axes. 
\item[\texttt{axesOffsetFixed}]: The offsets between azimuth and elevation   axes were held fixed at an assumed value. 
\item[\texttt{positionOffset}]: The measured position offsets in the X, Y,   Z horizontal system, relative to the values assumed at the time   of observing and used to track the phases.   \\MPositionOffset{AZEL}{stationPosition} 
\item[\texttt{positionErr}]: The statistical uncertainties of the measured   position offsets in the X, Y, Z horizontal system.     \\MPositionOffset{AZEL}{stationPosition} 
\item[\texttt{reducedChiSquared}]: Reduced \\$\\chi^2\\$      indicating the quality of the least-squares fit. 
\item[\texttt{delayRms}]: The root mean squared deviations for the   observed delays. 
\item[\texttt{phaseRms}]: The root mean squared deviations for the   observed phases. 
\end{description}
\endgroup

 \newpage

\subsection{CalPrimaryBeam Table}

 
 Result of Primary Beam Map measurement.

\begingroup
%
% define shortcuts for dimensions

%\newcommand{\numSubband}{\f$N_{Subb}\f$}
%\newcommand{\numReceptor}{\f$N_{Rece}\f$}

\par\noindent\begin{longtable} {|p{45mm}|p{45mm}|p{65mm}|}
\hline \multicolumn{3}{|c|}{\textbf{CalPrimaryBeam}} \\
\hline\hline
Name & Type (Shape) & Comment \\
\hline \endfirsthead
\hline \multicolumn{3}{|c|}{\textbf{CalPrimaryBeam} -- continued from previous page} \\
\hline \hline
Name & Type (Shape) & Comment \\
\hline \endhead
\hline \endfoot


\hline \multicolumn{3}{|l|}{\em Key} \\
\hline 

\texttt{antennaName} & \texttt{string} &  the name of the antenna. \\
\texttt{receiverBand} & \texttt{ReceiverBand} &  identifies the receiver band. \\
\texttt{calDataId} & \texttt{Tag} &  refers to a unique row in CalData Table. \\
\texttt{calReductionId} & \texttt{Tag} &  refers to a unique row in CalReduction Table. \\
\hline \multicolumn{3}{|l|}{\em Required Data} \\
\hline
\texttt{startValidTime} & \texttt{ArrayTime} &
 the start time of result validity period. \\
\texttt{endValidTime} & \texttt{ArrayTime} &
 the end time of result validity period. \\
\texttt{antennaMake} & \texttt{AntennaMake} &
 the antenna make. \\
\texttt{\f$N_{Subb}\f$} (\f$N_{Subb}\f$)& \texttt{int} &
 the number of subband images (frequency ranges simultaneously measured ).  \\
\texttt{frequencyRange} & \texttt{Frequency [numSubband]  [2] } &
 the range of frequencies over which the result is valid. \\
\texttt{\f$N_{Rece}\f$} (\f$N_{Rece}\f$)& \texttt{int} &
 the number of receptors. \\
\texttt{polarizationTypes} & \texttt{PolarizationType [numReceptor] } &
 identifies the polarizations types of the receptors (one value per receptor). \\
\texttt{mainBeamEfficiency} & \texttt{double [numReceptor] } &
 the main beam efficiency as derived from the beam map. \\
\texttt{beamDescriptionUID} & \texttt{EntityRef} &
 refers to the beam description image.  \\
\texttt{relativeAmplitudeRms} & \texttt{float} &
 the RMS fluctuations in terms of the relative beam amplitude. \\
\texttt{direction} & \texttt{Angle [2] } &
 the center direction.  \\
\texttt{minValidDirection} & \texttt{Angle [2] } &
 the minimum center direction of validity.  \\
\texttt{maxValidDirection} & \texttt{Angle [2] } &
 the maximum center direction of validity.  \\
\texttt{descriptionType} & \texttt{PrimaryBeamDescription} &
 quantity used to describe beam.  \\
\texttt{imageChannelNumber} & \texttt{int [numSubband] } &
 channel number in image for each subband.  \\
\texttt{imageNominalFrequency} & \texttt{Frequency [numSubband] } &
 nominal frequency for subband.  \\

\hline
\end{longtable}
  
~\par\noindent{\bf Column Descriptions:}

\begin{description}
\item[\texttt{antennaName}]: Refers uniquely to the hardware antenna object, as present in the original ASDM Antenna table. 
\item[\texttt{receiverBand}]: The name of the front-end frequency band being used. 
\item[\texttt{calDataId}]: CalData Table identifier. 
\item[\texttt{calReductionId}]: CalReduction Table identifier. 
\item[\texttt{startValidTime}]: The start of the time validity range for the result. 
\item[\texttt{endValidTime}]: The start of the time validity range for the result. 
\item[\texttt{antennaMake}]: The antenna make (e.g., for ALMA, the manufacturer name such as AEC, Vertex, or Melco). 
\item[\texttt{\f$N_{Subb}\f$}]: {\red long doc missing}
\item[\texttt{frequencyRange}]: Frequency range over which the result is valid. \\MFrequency{TOPO} 
\item[\texttt{\f$N_{Rece}\f$}]: The number or polarization receptors (one or two) for which the result is given. 
\item[\texttt{polarizationTypes}]: The polarization types of the receptors being used. 
\item[\texttt{mainBeamEfficiency}]: The main beam efficiency as     derived for the beam map. 
\item[\texttt{beamDescriptionUID}]: {\red long doc missing}
\item[\texttt{relativeAmplitudeRms}]: The root mean square   fluctuations in terms of relative beam amplitude, i.e. the antenna gain   scaled by its maximal value (on axis). 
\item[\texttt{direction}]: {\red long doc missing}
\item[\texttt{minValidDirection}]: {\red long doc missing}
\item[\texttt{maxValidDirection}]: {\red long doc missing}
\item[\texttt{descriptionType}]: {\red long doc missing}
\item[\texttt{imageChannelNumber}]: {\red long doc missing}
\item[\texttt{imageNominalFrequency}]: {\red long doc missing}
\end{description}
\endgroup

 \newpage

\subsection{CalReduction Table}

 
 Generic items describing the data reduction process.

\begingroup
%
% define shortcuts for dimensions

%\newcommand{\numApplied}{\f$N_{Appl}\f$}
%\newcommand{\numParam}{\f$N_{Para}\f$}
%\newcommand{\numInvalidConditions}{\f$N_{Inva}\f$}

\par\noindent\begin{longtable} {|p{45mm}|p{45mm}|p{65mm}|}
\hline \multicolumn{3}{|c|}{\textbf{CalReduction}} \\
\hline\hline
Name & Type (Shape) & Comment \\
\hline \endfirsthead
\hline \multicolumn{3}{|c|}{\textbf{CalReduction} -- continued from previous page} \\
\hline \hline
Name & Type (Shape) & Comment \\
\hline \endhead
\hline \endfoot


\hline \multicolumn{3}{|l|}{\em Key} \\
\hline 

\texttt{calReductionId} & \texttt{Tag} &  identifies a unique row in the table. \\
\hline \multicolumn{3}{|l|}{\em Required Data} \\
\hline
\texttt{\f$N_{Appl}\f$} (\f$N_{Appl}\f$)& \texttt{int} &
 the number of applied calibrations prior the reduction. \\
\texttt{appliedCalibrations} & \texttt{string [numApplied] } &
 the list of applied calibrations (one string per calibration). \\
\texttt{\f$N_{Para}\f$} (\f$N_{Para}\f$)& \texttt{int} &
 the number of listed parameters used for calibration. \\
\texttt{paramSet} & \texttt{string [numParam] } &
 the input parameters expressed as (keyword,value) pairs (one string per parameter). \\
\texttt{\f$N_{Inva}\f$} (\f$N_{Inva}\f$)& \texttt{int} &
 the number of invalidating conditions. \\
\texttt{invalidConditions} & \texttt{InvalidatingCondition [numInvalidConditions] } &
 invalidating use cases (one string per case). \\
\texttt{timeReduced} & \texttt{ArrayTime} &
 the epoch at which the data reduction was finished. \\
\texttt{messages} & \texttt{string} &
 messages issued by the data reduction software. \\
\texttt{software} & \texttt{string} &
 the name of the data reduction software reduction used. \\
\texttt{softwareVersion} & \texttt{string} &
 version information about the data reduction software used. \\

\hline
\end{longtable}
  
~\par\noindent{\bf Column Descriptions:}

\begin{description}
\item[\texttt{calReductionId}]: CalReduction row identifier. 
\item[\texttt{\f$N_{Appl}\f$}]: The number of calibrations applied to data before solving for the      result. 
\item[\texttt{appliedCalibrations}]: List of calibrations applied before solving for the result. 
\item[\texttt{\f$N_{Para}\f$}]: The number of listed parameters as used for this calibration. 
\item[\texttt{paramSet}]: The list of parameters needed to specfy the calibration applied given      as (keyword,value) pairs. 
\item[\texttt{\f$N_{Inva}\f$}]: The number of use cases that may invalidate the result. 
\item[\texttt{invalidConditions}]: The list of use cases that may invalidate the result. 
\item[\texttt{timeReduced}]: The epoch at which the data reduction was finished. 
\item[\texttt{messages}]: Messages issued by the data reduction software. 
\item[\texttt{software}]: The name of the data reduction software used to   derive the result. 
\item[\texttt{softwareVersion}]: The version of the data reduction spftware used to derive the     result. 
\end{description}
\endgroup

 \newpage

\subsection{CalSeeing Table}

 
 The seeing parameters deduced from TelCal calibrations.

\begingroup
%
% define shortcuts for dimensions

%\newcommand{\numBaseLengths}{\f$N_{Base}\f$}

\par\noindent\begin{longtable} {|p{45mm}|p{45mm}|p{65mm}|}
\hline \multicolumn{3}{|c|}{\textbf{CalSeeing}} \\
\hline\hline
Name & Type (Shape) & Comment \\
\hline \endfirsthead
\hline \multicolumn{3}{|c|}{\textbf{CalSeeing} -- continued from previous page} \\
\hline \hline
Name & Type (Shape) & Comment \\
\hline \endhead
\hline \endfoot


\hline \multicolumn{3}{|l|}{\em Key} \\
\hline 

\texttt{atmPhaseCorrection} & \texttt{AtmPhaseCorrection} &  describes how the atmospheric phase correction has been applied. \\
\texttt{calDataId} & \texttt{Tag} &  refers to a unique row in CalData Table. \\
\texttt{calReductionId} & \texttt{Tag} &  refers to a unique row in CalReduction Table. \\
\hline \multicolumn{3}{|l|}{\em Required Data} \\
\hline
\texttt{startValidTime} & \texttt{ArrayTime} &
 the start time of result validity period. \\
\texttt{endValidTime} & \texttt{ArrayTime} &
 the end time of result validity period. \\
\texttt{frequencyRange} & \texttt{Frequency [2] } &
 the range of frequencies over which this result is valid. \\
\texttt{integrationTime} & \texttt{Interval} &
 the duration of averaging for the evaluation of the RMS. \\
\texttt{\f$N_{Base}\f$} (\f$N_{Base}\f$)& \texttt{int} &
 the number of baselines for which the the RMS phase data is evaluated. \\
\texttt{baselineLengths} & \texttt{Length [numBaseLengths] } &
 the lengths of the baselines (one value per baseline). \\
\texttt{phaseRMS} & \texttt{Angle [numBaseLengths] } &
 the RMS of phase fluctuations (one value per baseline). \\
\texttt{seeing} & \texttt{Angle} &
 the seeing parameter, deduced for the LO1. \\
\texttt{seeingError} & \texttt{Angle} &
 the uncertainty on the seeing parameter. \\

\hline \multicolumn{3}{|l|}{\em Optional Data} \\
\hline
\texttt{exponent}  & \texttt{float} &
 the exponent of the spatial structure function. \\
\texttt{outerScale}  & \texttt{Length} &
 the outer scale. \\
\texttt{outerScaleRMS}  & \texttt{Angle} &
 the RMS of phase fluctuations at scale length outerScale. \\
\hline
\end{longtable}
  
~\par\noindent{\bf Column Descriptions:}

\begin{description}
\item[\texttt{atmPhaseCorrection}]: The atmospheric phase correction   states for which result is given. 
\item[\texttt{calDataId}]: CalData Table identifier. 
\item[\texttt{calReductionId}]: CalReduction Table identifier. 
\item[\texttt{startValidTime}]: The start of the time validity range for the result. 
\item[\texttt{endValidTime}]: The end of the time validity range for the result. 
\item[\texttt{frequencyRange}]: Frequency range over which the result is valid.\\MFrequency{TOPO} 
\item[\texttt{integrationTime}]: Averaging time for evaluation of rms fluctuations. 
\item[\texttt{\f$N_{Base}\f$}]: The number of baselines for which the rms phase data is evaluated. 
\item[\texttt{baselineLengths}]: The baseline lengths at which the phase rms is evaluated. 
\item[\texttt{phaseRMS}]: Root mean squared phase fluctuations for  each baseline length. 
\item[\texttt{seeing}]: The seeing parameter deduced for the calculated   rms phases. This is the half-power width of the beam that would   be synthesized if those phase fluctuations were not corrected. 
\item[\texttt{seeingError}]: The uncertainty on \\texttt{seeing}. 
\item[\texttt{exponent}]: The exponents of the fitted power laws   in the phase rms data. There may be two values (inner, outer) if an outer   scale is given. 
\item[\texttt{outerScale}]: The outer scale of turbulence (validity limit      of power law in phase structure function). 
\item[\texttt{outerScaleRMS}]: Root mean squared phase   fluctuations at scale length \\texttt{outerScale}. This number allows to   calculate the modelled phase structure function at any scale (above and   below \\texttt{outerScale}). 
\end{description}
\endgroup

 \newpage

\subsection{CalWVR Table}

 
 Result of the water vapour radiometric  calibration performed by TelCal. 

\begingroup
%
% define shortcuts for dimensions

%\newcommand{\numInputAntennas}{\f$N_{Inpu}\f$}
%\newcommand{\numChan}{\f$N_{Chan}\f$}
%\newcommand{\numPoly}{\f$N_{Poly}\f$}

\par\noindent\begin{longtable} {|p{45mm}|p{45mm}|p{65mm}|}
\hline \multicolumn{3}{|c|}{\textbf{CalWVR}} \\
\hline\hline
Name & Type (Shape) & Comment \\
\hline \endfirsthead
\hline \multicolumn{3}{|c|}{\textbf{CalWVR} -- continued from previous page} \\
\hline \hline
Name & Type (Shape) & Comment \\
\hline \endhead
\hline \endfoot


\hline \multicolumn{3}{|l|}{\em Key} \\
\hline 

\texttt{antennaName} & \texttt{string} &  the name of the antenna. \\
\texttt{calDataId} & \texttt{Tag} &  refers to a unique row in CalData Table. \\
\texttt{calReductionId} & \texttt{Tag} &  refers to unique row  in CalReductionTable. \\
\hline \multicolumn{3}{|l|}{\em Required Data} \\
\hline
\texttt{startValidTime} & \texttt{ArrayTime} &
 the start time of result validity period. \\
\texttt{endValidTime} & \texttt{ArrayTime} &
 the end time of result validity period. \\
\texttt{wvrMethod} & \texttt{WVRMethod} &
 identifies the method used for the calibration. \\
\texttt{\f$N_{Inpu}\f$} (\f$N_{Inpu}\f$)& \texttt{int} &
 the number of input antennas (i.e. equiped with functional WVRs). \\
\texttt{inputAntennaNames} & \texttt{string [numInputAntennas] } &
 the names of the input antennas (one string per antenna). \\
\texttt{\f$N_{Chan}\f$} (\f$N_{Chan}\f$)& \texttt{int} &
 the number of frequency channels in the WVR receiver. \\
\texttt{chanFreq} & \texttt{Frequency [numChan] } &
 the channel frequencies (one value per channel). \\
\texttt{chanWidth} & \texttt{Frequency [numChan] } &
 the widths of the channels (one value per channel). \\
\texttt{refTemp} & \texttt{Temperature [numInputAntennas]  [numChan] } &
 the reference temperatures (one value per input antenna per channel). \\
\texttt{\f$N_{Poly}\f$} (\f$N_{Poly}\f$)& \texttt{int} &
 the number of polynomial coefficients. \\
\texttt{pathCoeff} & \texttt{float [numInputAntennas]  [numChan]  [numPoly] } &
 the path length coefficients (one value per input antenna per channel per polynomial coefficient). \\
\texttt{polyFreqLimits} & \texttt{Frequency [2] } &
 the limits of the interval of frequencies for which the path length coefficients are computed. \\
\texttt{wetPath} & \texttt{float [numPoly] } &
 The wet path as a function frequency (expressed as a polynomial). \\
\texttt{dryPath} & \texttt{float [numPoly] } &
 The dry path as a function frequency (expressed as a polynomial). \\
\texttt{water} & \texttt{Length} &
 The precipitable water vapor corresponding to the reference model.  \\

\hline
\end{longtable}
  
~\par\noindent{\bf Column Descriptions:}

\begin{description}
\item[\texttt{antennaName}]: Refers uniquely to the hardware antenna object, as present in the original ASDM Antenna table. 
\item[\texttt{calDataId}]: CalData Table identifier. 
\item[\texttt{calReductionId}]: CalReduction Table identifier. 
\item[\texttt{startValidTime}]: The start of the time validity range for the result. 
\item[\texttt{endValidTime}]: The end of the time validity range for the result. 
\item[\texttt{wvrMethod}]: Method used, e.g, ab initio, Empirical. 
\item[\texttt{\f$N_{Inpu}\f$}]: The number of antennas equipped with     functional WVRs, from which the data should be interpolated, using the     path length coefficients calculated, to obtain the pathlength correction     to be applied to the antenna given as 'antennaName'. 
\item[\texttt{inputAntennaNames}]:    \\item[\\texttt{inputAntennaNames}]: The names of the antennas equipped with     functional WVRs, from which the data should be interpolated, using the     path length coefficients calculated, to obtain the pathlength correction     to be applied to the antenna given as 'antennaName'. 
\item[\texttt{\f$N_{Chan}\f$}]: Number of frequency channels in the WVR receiver. 
\item[\texttt{chanFreq}]: The center frequency of the WVR channels.   {\\MFrequency{TOPO}} 
\item[\texttt{chanWidth}]: The frequency width of the WVR channels. 
\item[\texttt{refTemp}]: The reference temperature   \\$T_{Rcj}\\$ for each WVR channel to be used in the path length formula. 
\item[\texttt{\f$N_{Poly}\f$}]: The number of polynomial coefficients given, to     obtain the frequency dependence of the pathlength correction. 
\item[\texttt{pathCoeff}]: The path length coefficients to be used to obtain      the pathlength correction to be applied to the antenna given as 'antennaName'.       These are the coefficients \\$C_{kcj}\\$ (\\$k=1, \\$\\numPoly )     used to obtain the      path length as a linear combinations of the data from    the \\numChan\\ WVR      channels (in temperature units). Each coefficient is     a term of a polynomial      expansion (\\numPoly\\ coefficients) of {the} predicted path    length as      a function of observing frequency in the astronomical band    (frequency limits      in PolyFreqLimits). These polynomials are Chebichev         polynomials in this frequency interval reduced to \\$[-1.,1.]\\$).         For the given Antenna (antennaName), the {path length} correction to be applied is:     \\$\\$ \\sum_{j=1,\\numInputAntenna} \\big[\\sum_{c=1,\\numChan} P_{c j}(\\nu) (T_{cj} - T_{Rcj})\\big] \\$\\$     where       \\begin{itemize}     \\item \\$T_{Rcj}\\$ is the reference WVR temperature for channel \\$c\\$ of antenna \\$j\\$      \\item \\$T_{cj}\\$ is the observed WVR temperature for channel \\$c\\$ of antenna \\$j\\$      \\item \\$ P_{c j}(\\nu) \\$ is the value of the Chebichev polynomial with     coefficients evaluated at sky frequency \\$\\nu\\$ with coefficents \\$C_{kcj}\\$ (\\$k=1, \\numPoly\\$)         \\end{itemize}      In the most frequent case there is a single input antenna     (AntennaName). If the WVR device on a given antenna is not available or     fails, the above formula gives the desired way to interpolate the     correction for that antenna, based of other antennas (close neighbours). 
\item[\texttt{polyFreqLimits}]: Frequency limits of the frequency interval   for which the path length coefficients are computed. 
\item[\texttt{wetPath}]:  The wet path as a function of frequency expressed as Chebichev polynomial in the frequency range reduced to \\f\\$[-1,1]\\f\\$. This corresponds to the reference model that reproduces the average line shape and is used in the delay server to track the phases and delays, while the departures from the average water line shape are used for the pathlength correction applied to the correlator. 
\item[\texttt{dryPath}]: The dry path as a function of frequency expressed as a Chebicehv polynomial in the frequency range reduced to \\f\\$[-1,1]\\f\\$. This corresponds to the same reference model as wetPath. 
\item[\texttt{water}]: {\red long doc missing}
\end{description}
\endgroup

 \newpage

\subsection{ConfigDescription Table}

 
 Defines the hardware configuration used to  obtain the science data.

\begingroup
%
% define shortcuts for dimensions

%\newcommand{\numAntenna}{\f$N_{Ante}\f$}
%\newcommand{\numDataDescription}{\f$N_{Data}\f$}
%\newcommand{\numFeed}{\f$N_{Feed}\f$}
%\newcommand{\numAtmPhaseCorrection}{\f$N_{AtmP}\f$}
%\newcommand{\numAssocValues}{\f$N_{Asso}\f$}

\par\noindent\begin{longtable} {|p{45mm}|p{45mm}|p{65mm}|}
\hline \multicolumn{3}{|c|}{\textbf{ConfigDescription}} \\
\hline\hline
Name & Type (Shape) & Comment \\
\hline \endfirsthead
\hline \multicolumn{3}{|c|}{\textbf{ConfigDescription} -- continued from previous page} \\
\hline \hline
Name & Type (Shape) & Comment \\
\hline \endhead
\hline \endfoot


\hline \multicolumn{3}{|l|}{\em Key} \\
\hline 

\texttt{configDescriptionId} & \texttt{Tag} &  identifies a unique row in the table. \\
\hline \multicolumn{3}{|l|}{\em Required Data} \\
\hline
\texttt{\f$N_{Ante}\f$} (\f$N_{Ante}\f$)& \texttt{int} &
 the number of antennas. \\
\texttt{\f$N_{Data}\f$} (\f$N_{Data}\f$)& \texttt{int} &
 the number of data descriptions. \\
\texttt{\f$N_{Feed}\f$} (\f$N_{Feed}\f$)& \texttt{int} &
 the number of feeds. \\
\texttt{correlationMode} & \texttt{CorrelationMode} &
 identifies the correlation mode. \\
\texttt{\f$N_{AtmP}\f$} (\f$N_{AtmP}\f$)& \texttt{int} &
 the number of descriptions of the atmospheric phase correction. \\
\texttt{atmPhaseCorrection} & \texttt{AtmPhaseCorrection [numAtmPhaseCorrection] } &
 describe how the atmospheric phase corrections have been applied (one value per correction). \\
\texttt{processorType} & \texttt{ProcessorType} &
 identifies the generic processor's type. \\
\texttt{spectralType} & \texttt{SpectralResolutionType} &
 identifies the spectral type of the data. \\
\texttt{antennaId} & \texttt{Tag [numAntenna] } &
 identifies numAntenna rows in AntennaTable. \\
\texttt{feedId} & \texttt{int [numAntenna*numFeed] } &
 refers to many collections of rows in the Feed Table. \\
\texttt{switchCycleId} & \texttt{Tag [numDataDescription] } &
 refers to a unique row in the SwitchCycle Table. \\
\texttt{dataDescriptionId} & \texttt{Tag [numDataDescription] } &
 refers to one or more rows in DataDescriptionTable. \\
\texttt{processorId} & \texttt{Tag} &
 refers to a unique row in ProcessorTable. \\

\hline \multicolumn{3}{|l|}{\em Optional Data} \\
\hline
\texttt{phasedArrayList}  & \texttt{int [numAntenna] } &
 phased array identifiers. \\
\texttt{\f$N_{Asso}\f$} (\f$N_{Asso}\f$) & \texttt{int} &
 the number of associated config descriptions. \\
\texttt{assocNature}  & \texttt{SpectralResolutionType [numAssocValues] } &
 the natures of the associations with other config descriptions (one value per association). \\
\texttt{assocConfigDescriptionId}  & \texttt{Tag [numAssocValues] } &
 refers to one or more rows in ConfigDescriptionTable. \\
\hline
\end{longtable}
  
~\par\noindent{\bf Column Descriptions:}

\begin{description}
\item[\texttt{configDescriptionId}]: Identifies the row in the Configuration Description Table. 
\item[\texttt{\f$N_{Ante}\f$}]: The number of antennas used (given by the \\texttt{antennaId} array). 
\item[\texttt{\f$N_{Data}\f$}]: Number of data descriptions for this     row. This is equal to the number of spectral windows. 
\item[\texttt{\f$N_{Feed}\f$}]: Number of feeds (given by the \\texttt{feedId} array). For     ALMA, \\texttt{numFeed} is always one and \\texttt{feedId} is zero. 
\item[\texttt{correlationMode}]: The correlation mode used; for ALMA this is either      Autocorrelation only, or Correlation and Autocorrelation. 
\item[\texttt{\f$N_{AtmP}\f$}]: Number of Atmospheric Phase Corrections. 
\item[\texttt{atmPhaseCorrection}]: The atmospheric phase correction states of data given      (corrected, uncorrected, or both). 
\item[\texttt{processorType}]: The generic processor type, such as, e.g., CORRELATOR, SPECTROMETER, or RADIOMETER. 
\item[\texttt{spectralType}]: The spectral type of this     data. It may be e.g direct spectral processor data (at full resolution),     or channel averaged spectral procesor data, or total power detector     data. 
\item[\texttt{antennaId}]: The number of antennas used (given by the \\texttt{antennaId} table). 
\item[\texttt{feedId}]: Specifies which feed was used in the Feed Table. 
\item[\texttt{switchCycleId}]: SwitchCycle Table identifier. 
\item[\texttt{dataDescriptionId}]: DataDescription  Table identifier. 
\item[\texttt{processorId}]: The Processor Identifier provides a direct     link to a row in the Processor Table. 
\item[\texttt{phasedArrayList}]: List of phased array     identifiers; normally not used for ALMA. 
\item[\texttt{\f$N_{Asso}\f$}]: The number of associated config descriptions. 
\item[\texttt{assocNature}]: The nature of the associations established by the \\texttt{assocConfigDescriptionId} array. 
\item[\texttt{assocConfigDescriptionId}]: ConfigDescription Table identifier 
\end{description}
\endgroup

 \newpage

\subsection{CorrelatorMode Table}

 
 Contains information on a Correlator processor.

\begingroup
%
% define shortcuts for dimensions

%\newcommand{\numBaseband}{\f$N_{Base}\f$}
%\newcommand{\numAxes}{\f$N_{Axes}\f$}

\par\noindent\begin{longtable} {|p{45mm}|p{45mm}|p{65mm}|}
\hline \multicolumn{3}{|c|}{\textbf{CorrelatorMode}} \\
\hline\hline
Name & Type (Shape) & Comment \\
\hline \endfirsthead
\hline \multicolumn{3}{|c|}{\textbf{CorrelatorMode} -- continued from previous page} \\
\hline \hline
Name & Type (Shape) & Comment \\
\hline \endhead
\hline \endfoot


\hline \multicolumn{3}{|l|}{\em Key} \\
\hline 

\texttt{correlatorModeId} & \texttt{Tag} &  refers to a unique row in the table. \\
\hline \multicolumn{3}{|l|}{\em Required Data} \\
\hline
\texttt{\f$N_{Base}\f$} (\f$N_{Base}\f$)& \texttt{int} &
 the number of basebands. \\
\texttt{basebandNames} & \texttt{BasebandName [numBaseband] } &
 identifies the basebands (one value per basebands). \\
\texttt{basebandConfig} & \texttt{int [numBaseband] } &
 encodes the basebands configurations (one value per baseband). \\
\texttt{accumMode} & \texttt{AccumMode} &
 identifies the accumulation mode.  \\
\texttt{binMode} & \texttt{int} &
 the binning mode. \\
\texttt{\f$N_{Axes}\f$} (\f$N_{Axes}\f$)& \texttt{int} &
 the number of axes in the binary data blocks. \\
\texttt{axesOrderArray} & \texttt{AxisName [numAxes] } &
 the order of axes in the binary data blocks. \\
\texttt{filterMode} & \texttt{FilterMode [numBaseband] } &
 identifies the filters modes (one value per baseband). \\
\texttt{correlatorName} & \texttt{CorrelatorName} &
 identifies the correlator's name. \\

\hline
\end{longtable}
  
~\par\noindent{\bf Column Descriptions:}

\begin{description}
\item[\texttt{correlatorModeId}]: Identifies the row in the Correlator Mode  Table. 
\item[\texttt{\f$N_{Base}\f$}]: The number of baseband pairs used; this may   be up to four for ALMA. A pair has two orthogonal polarization channels. 
\item[\texttt{basebandNames}]: Baseband names, that is the baseband pairs that are used ; there are four for ALMA. 
\item[\texttt{basebandConfig}]: Baseband Configuration; for ALMA currently   expressed as a number like '103' for Time division mode, or '1' for   tunable filter mode; there is one of these for each baseband pair. 
\item[\texttt{accumMode}]: The accumulation mode: for ALMA this   is either FAST (For 1ms dumps, autocorrelation only) or NORMAL (for 16ms   minimum dumps, and simultaneous correlation and autocorrelation). 
\item[\texttt{binMode}]: The number of data bins: data bins are used   together for switch cycles. For instance we have two for frequency   switching of nutator switching, but one only for interferometry. 
\item[\texttt{\f$N_{Axes}\f$}]: The number of axes in the binary data blocks. 
\item[\texttt{axesOrderArray}]: The standard order of axes in the binary  data blocks. Axes may be omitted (See the BDF documentation). 
\item[\texttt{filterMode}]: The mode of operation of the digital filters   used at the input of the {ALMA} correlator. These are the TDM   (Time Division Mode) or TFB (Tunable Filter Bank) modes. 
\item[\texttt{correlatorName}]: The name of the correlator; in ALMA we   have the 'baseline' correlator and the 'ACA' correlator. 
\end{description}
\endgroup

 \newpage

\subsection{DataDescription Table}

 
 Spectro-polarization description.

\begingroup
%
% define shortcuts for dimensions


\par\noindent\begin{longtable} {|p{45mm}|p{45mm}|p{65mm}|}
\hline \multicolumn{3}{|c|}{\textbf{DataDescription}} \\
\hline\hline
Name & Type (Shape) & Comment \\
\hline \endfirsthead
\hline \multicolumn{3}{|c|}{\textbf{DataDescription} -- continued from previous page} \\
\hline \hline
Name & Type (Shape) & Comment \\
\hline \endhead
\hline \endfoot


\hline \multicolumn{3}{|l|}{\em Key} \\
\hline 

\texttt{dataDescriptionId} & \texttt{Tag} &  identifies a unique row in the table. \\
\hline \multicolumn{3}{|l|}{\em Required Data} \\
\hline
\texttt{polOrHoloId} & \texttt{Tag} &
 refers to a unique row in PolarizationTable or HolograpyTable. \\
\texttt{spectralWindowId} & \texttt{Tag} &
 refers to a unique row in SpectralWindowTable. \\

\hline
\end{longtable}
  
~\par\noindent{\bf Column Descriptions:}

\begin{description}
\item[\texttt{dataDescriptionId}]: DataDescription  Table identifier. 
\item[\texttt{polOrHoloId}]: Polarization Table identifier or Holography   Table identifier. 
\item[\texttt{spectralWindowId}]: SpectralWindow Table identifier. 
\end{description}
\endgroup

 \newpage

\subsection{DelayModel Table}

 
 Contains the delay model components. For ALMA this includes all TMCDB delay model components.

\begingroup
%
% define shortcuts for dimensions

%\newcommand{\numPoly}{\f$N_{Poly}\f$}
%\newcommand{\numLO}{\f$N_{LO}\f$}
%\newcommand{\numReceptor}{\f$N_{Rece}\f$}

\par\noindent\begin{longtable} {|p{45mm}|p{45mm}|p{65mm}|}
\hline \multicolumn{3}{|c|}{\textbf{DelayModel}} \\
\hline\hline
Name & Type (Shape) & Comment \\
\hline \endfirsthead
\hline \multicolumn{3}{|c|}{\textbf{DelayModel} -- continued from previous page} \\
\hline \hline
Name & Type (Shape) & Comment \\
\hline \endhead
\hline \endfoot


\hline \multicolumn{3}{|l|}{\em Key} \\
\hline 

\texttt{antennaId} & \texttt{Tag} &  refers to a unique row in AntennaTable. \\
\texttt{spectralWindowId} & \texttt{Tag} &  refers to a unique row in  SpectraWindowTable.  \\
\texttt{timeInterval} & \texttt{ArrayTimeInterval} &  time interval for which the row's content is valid. \\
\hline \multicolumn{3}{|l|}{\em Required Data} \\
\hline
\texttt{\f$N_{Poly}\f$} (\f$N_{Poly}\f$)& \texttt{int} &
 the number of coefficients of the polynomials. \\
\texttt{phaseDelay} & \texttt{double [numPoly] } &
 the phase delay polynomial (rad).  \\
\texttt{phaseDelayRate} & \texttt{double [numPoly] } &
 Phase delay rate polynomial (rad/s).  \\
\texttt{groupDelay} & \texttt{double [numPoly] } &
 Group delay polynomial (s).  \\
\texttt{groupDelayRate} & \texttt{double [numPoly] } &
 Group delay rate polynomial (s/s)  \\
\texttt{fieldId} & \texttt{Tag} &
   \\

\hline \multicolumn{3}{|l|}{\em Optional Data} \\
\hline
\texttt{timeOrigin}  & \texttt{ArrayTime} &
 value used as the origin for the evaluation of the polynomials.  \\
\texttt{atmosphericGroupDelay}  & \texttt{double} &
 Atmosphere group delay.  \\
\texttt{atmosphericGroupDelayRate}  & \texttt{double} &
 Atmosphere group delay rate.  \\
\texttt{geometricDelay}  & \texttt{double} &
 Geometric delay.  \\
\texttt{geometricDelayRate}  & \texttt{double} &
 Geometric delay.  \\
\texttt{\f$N_{LO}\f$} (\f$N_{LO}\f$) & \texttt{int} &
 the number of local oscillators.  \\
\texttt{LOOffset}  & \texttt{Frequency [numLO] } &
 Local oscillator offset.  \\
\texttt{LOOffsetRate}  & \texttt{Frequency [numLO] } &
 Local oscillator offset rate.  \\
\texttt{dispersiveDelay}  & \texttt{double} &
 Dispersive delay.  \\
\texttt{dispersiveDelayRate}  & \texttt{double} &
 Dispersive delay rate.  \\
\texttt{atmosphericDryDelay}  & \texttt{double} &
 the dry atmospheric delay component. \\
\texttt{atmosphericWetDelay}  & \texttt{double} &
 the wet atmospheric delay. \\
\texttt{padDelay}  & \texttt{double} &
 Pad delay.  \\
\texttt{antennaDelay}  & \texttt{double} &
 Antenna delay.  \\
\texttt{\f$N_{Rece}\f$} (\f$N_{Rece}\f$) & \texttt{int} &
   \\
\texttt{polarizationType}  & \texttt{PolarizationType [numReceptor] } &
 describes the polarizations of the receptors (one value per receptor).  \\
\texttt{electronicDelay}  & \texttt{double [numReceptor] } &
 the electronic delay.  \\
\texttt{electronicDelayRate}  & \texttt{double [numReceptor] } &
 the electronic delay rate.  \\
\texttt{receiverDelay}  & \texttt{double [numReceptor] } &
 the receiver delay.  \\
\texttt{IFDelay}  & \texttt{double [numReceptor] } &
 the intermediate frequency delay.  \\
\texttt{LODelay}  & \texttt{double [numReceptor] } &
 the local oscillator delay.  \\
\texttt{crossPolarizationDelay}  & \texttt{double} &
 the cross polarization delay.  \\
\hline
\end{longtable}
  
~\par\noindent{\bf Column Descriptions:}

\begin{description}
\item[\texttt{antennaId}]: Antenna identifier, as indexed from an element   in the antennaArray collection in the configDescription table. 
\item[\texttt{spectralWindowId}]: {\red long doc missing}
\item[\texttt{timeInterval}]: Time interval for which the parameters in   the row are valid. The same reference used for the Time column in the Main Table   must be used. 
\item[\texttt{\f$N_{Poly}\f$}]: Series order for the delay time polynomial expansions. 
\item[\texttt{phaseDelay}]: {\red long doc missing}
\item[\texttt{phaseDelayRate}]: {\red long doc missing}
\item[\texttt{groupDelay}]: {\red long doc missing}
\item[\texttt{groupDelayRate}]: {\red long doc missing}
\item[\texttt{fieldId}]: {\red long doc missing}
\item[\texttt{timeOrigin}]: {\red long doc missing}
\item[\texttt{atmosphericGroupDelay}]: {\red long doc missing}
\item[\texttt{atmosphericGroupDelayRate}]: {\red long doc missing}
\item[\texttt{geometricDelay}]: {\red long doc missing}
\item[\texttt{geometricDelayRate}]: {\red long doc missing}
\item[\texttt{\f$N_{LO}\f$}]: {\red long doc missing}
\item[\texttt{LOOffset}]: {\red long doc missing}
\item[\texttt{LOOffsetRate}]: {\red long doc missing}
\item[\texttt{dispersiveDelay}]: {\red long doc missing}
\item[\texttt{dispersiveDelayRate}]: {\red long doc missing}
\item[\texttt{atmosphericDryDelay}]: Dry atmosphere delay component. 
\item[\texttt{atmosphericWetDelay}]: Wet atmosphere delay component. 
\item[\texttt{padDelay}]: {\red long doc missing}
\item[\texttt{antennaDelay}]: {\red long doc missing}
\item[\texttt{\f$N_{Rece}\f$}]: {\red long doc missing}
\item[\texttt{polarizationType}]: {\red long doc missing}
\item[\texttt{electronicDelay}]: {\red long doc missing}
\item[\texttt{electronicDelayRate}]: {\red long doc missing}
\item[\texttt{receiverDelay}]: {\red long doc missing}
\item[\texttt{IFDelay}]: {\red long doc missing}
\item[\texttt{LODelay}]: {\red long doc missing}
\item[\texttt{crossPolarizationDelay}]: {\red long doc missing}
\end{description}
\endgroup

 \newpage

\subsection{DelayModelFixedParameters Table}

 
 

\begingroup
%
% define shortcuts for dimensions


\par\noindent\begin{longtable} {|p{45mm}|p{45mm}|p{65mm}|}
\hline \multicolumn{3}{|c|}{\textbf{DelayModelFixedParameters}} \\
\hline\hline
Name & Type (Shape) & Comment \\
\hline \endfirsthead
\hline \multicolumn{3}{|c|}{\textbf{DelayModelFixedParameters} -- continued from previous page} \\
\hline \hline
Name & Type (Shape) & Comment \\
\hline \endhead
\hline \endfoot


\hline \multicolumn{3}{|l|}{\em Key} \\
\hline 

\texttt{delayModelFixedParametersId} & \texttt{Tag} &  identifies a unique row in the table.  \\
\hline \multicolumn{3}{|l|}{\em Required Data} \\
\hline
\texttt{delayModelVersion} & \texttt{string} &
  should include the name of the software and its version.  Something like  "CALC v11" or "VDT v1.0" or "MODEST v2.1".   \\
\texttt{execBlockId} & \texttt{Tag} &
 refers to a unique row of the ExecBlock table.  \\

\hline \multicolumn{3}{|l|}{\em Optional Data} \\
\hline
\texttt{gaussConstant}  & \texttt{AngularRate} &
 the Gauss gravitational constant (should be of order \\f\\$ 1.720209895.10^{-2} rad/d \\f\\$ but in SI units of \\f\\$ rad s^{-1} \\f\\$).   \\
\texttt{newtonianConstant}  & \texttt{double} &
 the newtonian constant of gravitation (should be of order \\f\\$ 6.67259.10^{-11} m^3  kg^{-1}  s^2 \\f\\$).   \\
\texttt{gravity}  & \texttt{double} &
 the gravity acceleration in \\f\\$ m s^{-2} \\f\\$.  \\
\texttt{earthFlattening}  & \texttt{double} &
 the ratio of equatorial to polar radii.  \\
\texttt{earthRadius}  & \texttt{Length} &
 the earth equatorial radius in \\f\\$ m \\f\\$.  \\
\texttt{moonEarthMassRatio}  & \texttt{double} &
   \\
\texttt{ephemerisEpoch}  & \texttt{string} &
 should always be 'J2000'.  \\
\texttt{earthTideLag}  & \texttt{double} &
   \\
\texttt{earthGM}  & \texttt{double} &
 the earth gravitation constant in \\f\\$ m^3 s^{-2} \\f\\$.  \\
\texttt{moonGM}  & \texttt{double} &
 the moon gravitation constant in \\f\\$ m^3 s^{-2} \\f\\$.  \\
\texttt{sunGM}  & \texttt{double} &
 the sun gravitation constant in \\f\\$ m^3 s^{-2} \\f\\$.  \\
\texttt{loveNumberH}  & \texttt{double} &
 the earth global Love number H.  \\
\texttt{loveNumberL}  & \texttt{double} &
 the earth global Love number L.  \\
\texttt{precessionConstant}  & \texttt{AngularRate} &
 the general precession constant in \\f\\$ arcsec \\  s^{-1} \\f\\$.  \\
\texttt{lightTime1AU}  & \texttt{double} &
 the light time for 1 AU in seconds.  \\
\texttt{speedOfLight}  & \texttt{Speed} &
 the speed of light in \\f\\$ m s^{-1} \\f\\$.
  \\
\texttt{delayModelFlags}  & \texttt{string} &
 the delay model switches.  \\
\hline
\end{longtable}
  
~\par\noindent{\bf Column Descriptions:}

\begin{description}
\item[\texttt{delayModelFixedParametersId}]: {\red long doc missing}
\item[\texttt{delayModelVersion}]: {\red long doc missing}
\item[\texttt{execBlockId}]: {\red long doc missing}
\item[\texttt{gaussConstant}]: {\red long doc missing}
\item[\texttt{newtonianConstant}]: {\red long doc missing}
\item[\texttt{gravity}]: {\red long doc missing}
\item[\texttt{earthFlattening}]: {\red long doc missing}
\item[\texttt{earthRadius}]: {\red long doc missing}
\item[\texttt{moonEarthMassRatio}]: {\red long doc missing}
\item[\texttt{ephemerisEpoch}]: {\red long doc missing}
\item[\texttt{earthTideLag}]: {\red long doc missing}
\item[\texttt{earthGM}]: {\red long doc missing}
\item[\texttt{moonGM}]: {\red long doc missing}
\item[\texttt{sunGM}]: {\red long doc missing}
\item[\texttt{loveNumberH}]: {\red long doc missing}
\item[\texttt{loveNumberL}]: {\red long doc missing}
\item[\texttt{precessionConstant}]: {\red long doc missing}
\item[\texttt{lightTime1AU}]: {\red long doc missing}
\item[\texttt{speedOfLight}]: {\red long doc missing}
\item[\texttt{delayModelFlags}]: {\red long doc missing}
\end{description}
\endgroup

 \newpage

\subsection{DelayModelVariableParameters Table}

 
 

\begingroup
%
% define shortcuts for dimensions


\par\noindent\begin{longtable} {|p{45mm}|p{45mm}|p{65mm}|}
\hline \multicolumn{3}{|c|}{\textbf{DelayModelVariableParameters}} \\
\hline\hline
Name & Type (Shape) & Comment \\
\hline \endfirsthead
\hline \multicolumn{3}{|c|}{\textbf{DelayModelVariableParameters} -- continued from previous page} \\
\hline \hline
Name & Type (Shape) & Comment \\
\hline \endhead
\hline \endfoot


\hline \multicolumn{3}{|l|}{\em Key} \\
\hline 

\texttt{delayModelVariableParametersId} & \texttt{Tag} &  identifies a unique row in the table.  \\
\hline \multicolumn{3}{|l|}{\em Required Data} \\
\hline
\texttt{time} & \texttt{ArrayTime} &
 the day and time relevant for the data in this row.  \\
\texttt{ut1\_utc} & \texttt{double} &
 \\f\\$ UT1 - UTC \\f\\$ in \\f\\$ second \\f\\$.  \\
\texttt{iat\_utc} & \texttt{double} &
 \\f\\$ IAT - UTC \\f\\$ in \\f\\$ second \\f\\$.  \\
\texttt{timeType} & \texttt{DifferenceType} &
 the type of the two time differences expressed in ut1_utc and iat_utc  \\
\texttt{gstAtUt0} & \texttt{Angle} &
 in \\f\\$ radian \\f\\$.  \\
\texttt{earthRotationRate} & \texttt{AngularRate} &
 in \\f\\$ radian \\ s^{-1} \\f\\$ (the seconds are in \\f\\$ IAT \\f\\$).    \\
\texttt{polarOffsets} & \texttt{double [2] } &
 the \\f\\$ X, Y \\f\\$ polar offsets in \\f\\$ arcsec \\f\\$.  \\
\texttt{polarOffsetsType} & \texttt{DifferenceType} &
 the type of the polar offsets (values found in polarOffsets).  \\
\texttt{delayModelFixedParametersId} & \texttt{Tag} &
 refers to a unique row of the DelayModelFixedParameters table.  \\

\hline \multicolumn{3}{|l|}{\em Optional Data} \\
\hline
\texttt{nutationInLongitude}  & \texttt{Angle} &
 the nutation in longitude ( the part parallel to the ecliptic) in \\f\\$ radian \\f\\$.  \\
\texttt{nutationInLongitudeRate}  & \texttt{AngularRate} &
 the rate of nutation in longitude in \\f\\$ radian \\ s^{-1} \\f\\$.  \\
\texttt{nutationInObliquity}  & \texttt{Angle} &
 the nutation in obliquity (the part perpendicular to the ecliptic) in \\f\\$ radian \\f\\$.  \\
\texttt{nutationInObliquityRate}  & \texttt{AngularRate} &
 the rate of nutation in obliquity in \\f\\$ radian \\ s^{-1} \\f\\$.  \\
\hline
\end{longtable}
  
~\par\noindent{\bf Column Descriptions:}

\begin{description}
\item[\texttt{delayModelVariableParametersId}]: {\red long doc missing}
\item[\texttt{time}]: {\red long doc missing}
\item[\texttt{ut1\_utc}]: {\red long doc missing}
\item[\texttt{iat\_utc}]: {\red long doc missing}
\item[\texttt{timeType}]: {\red long doc missing}
\item[\texttt{gstAtUt0}]: {\red long doc missing}
\item[\texttt{earthRotationRate}]: {\red long doc missing}
\item[\texttt{polarOffsets}]: {\red long doc missing}
\item[\texttt{polarOffsetsType}]: {\red long doc missing}
\item[\texttt{delayModelFixedParametersId}]: {\red long doc missing}
\item[\texttt{nutationInLongitude}]: {\red long doc missing}
\item[\texttt{nutationInLongitudeRate}]: {\red long doc missing}
\item[\texttt{nutationInObliquity}]: {\red long doc missing}
\item[\texttt{nutationInObliquityRate}]: {\red long doc missing}
\end{description}
\endgroup

 \newpage

\subsection{Doppler Table}

 
 Doppler tracking information. This table defines how velocity  information is converted into a frequency offset to compensate in real time  for the Doppler effect. This table may be omitted for ALMA when the  Doppler tracking is not corrected.

\begingroup
%
% define shortcuts for dimensions


\par\noindent\begin{longtable} {|p{45mm}|p{45mm}|p{65mm}|}
\hline \multicolumn{3}{|c|}{\textbf{Doppler}} \\
\hline\hline
Name & Type (Shape) & Comment \\
\hline \endfirsthead
\hline \multicolumn{3}{|c|}{\textbf{Doppler} -- continued from previous page} \\
\hline \hline
Name & Type (Shape) & Comment \\
\hline \endhead
\hline \endfoot


\hline \multicolumn{3}{|l|}{\em Key} \\
\hline 

\texttt{dopplerId} & \texttt{int} &  identifies a collection of rows in the table. \\
\texttt{sourceId} & \texttt{int} &  refers to a collection of rows in SourceTable. \\
\hline \multicolumn{3}{|l|}{\em Required Data} \\
\hline
\texttt{transitionIndex} & \texttt{int} &
 selects the transition in the source table for which the doppler tracking is done. \\
\texttt{velDef} & \texttt{DopplerReferenceCode} &
 identifies the definition of the velocity. \\

\hline
\end{longtable}
  
~\par\noindent{\bf Column Descriptions:}

\begin{description}
\item[\texttt{dopplerId}]: Identifies the row in the Doppler Table. 
\item[\texttt{sourceId}]: Identifies a source in the Source table. 
\item[\texttt{transitionIndex}]: Identifies a particular spectral     transition (for a source in the Source Table). 
\item[\texttt{velDef}]: Velocity definition of the Doppler Shift,     e.g. RADIO or OPTICAL. The value of the velocity is found in     the Source Table as \\texttt{sysVel[transitionIndex]}.  \\MDoppler{} 
\end{description}
\endgroup

 \newpage

\subsection{Ephemeris Table}

 
 

\begingroup
%
% define shortcuts for dimensions

%\newcommand{\numPolyDir}{\f$N_{Poly}\f$}
%\newcommand{\numPolyDist}{\f$N_{Poly}\f$}
%\newcommand{\numPolyRadVel}{\f$N_{Poly}\f$}

\par\noindent\begin{longtable} {|p{45mm}|p{45mm}|p{65mm}|}
\hline \multicolumn{3}{|c|}{\textbf{Ephemeris}} \\
\hline\hline
Name & Type (Shape) & Comment \\
\hline \endfirsthead
\hline \multicolumn{3}{|c|}{\textbf{Ephemeris} -- continued from previous page} \\
\hline \hline
Name & Type (Shape) & Comment \\
\hline \endhead
\hline \endfoot


\hline \multicolumn{3}{|l|}{\em Key} \\
\hline 

\texttt{timeInterval} & \texttt{ArrayTimeInterval} &  interval of time during which the data are relevant.   \\
\texttt{ephemerisId} & \texttt{int} &  identifies a collection of rows in the table.   \\
\hline \multicolumn{3}{|l|}{\em Required Data} \\
\hline
\texttt{observerLocation} & \texttt{double [3] } &
 a triple of double precision values defining the observer location. This triple contains in that order the longitude, the latitude and the altitude of the observer :
<ul>
<li> the longitude is expressed in radian. An east (resp. west) longitude is denoted as a positive (resp. negative) quantity.</li>
<li> the latitude is expressed in radian. A north (resp. south) latitude is denoted as a positive (resp. negative) quantity. </li>
<li> the altitude is expressed in meter. It's the altitude above the reference ellipsoid. </li>
</ul>
A triple with all its elements equal to 0.0 will mean that a geocentric coordinate system is in use instead of a topocentric one.  \\
\texttt{equinoxEquator} & \texttt{double} &
 epoch at which equator and equinox were calculated for ephemeris. Expresses a year as a decimal value (J2000 would be represented as 2000.0).  \\
\texttt{\f$N_{Poly}\f$} (\f$N_{Poly}\f$)& \texttt{int} &
 the number of coefficients of the polynomial stored in phaseDir. It has to be \\f\\$ \\ge 1 \\f\\$.   \\
\texttt{dir} & \texttt{double [numPolyDir]  [2] } &
 the ephemeris direction expressed in radian. The nominal entry in the phaseDir, delayDir, or ReferenceDir in the Field table serves as additional offset to the direction described by "dir". The actual direction is obtained by composition, e.g. actual phase direction = [phasDir value from Field table] + [dir].

The direction described by dir  is the result of the sum

\\f[ dir_{0,i} + dir_{1,i}*dt + dir_{2,i}*dt^2 + ... + dir_{numPolyDir-1,i}*dt^{numPolyDir-1}, \\forall i \\in \\{0,1\\} \\f]

where

\\f[ dt = t - timeOrigin \\f] 
  \\
\texttt{\f$N_{Poly}\f$} (\f$N_{Poly}\f$)& \texttt{int} &
 the number of coefficients of the polynomial stored in distance. It has to be \\f\\$ \\ge 1 \\f\\$.   \\
\texttt{distance} & \texttt{double [numPolyDist] } &
 the coefficiens of the polynomial used to calculate the distance, expressed in meter,  to the object from the position of the antenna along the given direction. This distance is the result of the sum :

\\f[ distance_0 + distance_1*dt + distance_2*dt^2 + ... + distance_{numPolyDist-1}*dt^{numPolyDist-1} \\f]

where

\\f[ dt = t - timeOrigin \\f].
  \\
\texttt{timeOrigin} & \texttt{ArrayTime} &
 the time origin used in the evaluation of the polynomial expressions.   \\
\texttt{origin} & \texttt{string} &
 the origin of the ephemeris information. \\

\hline \multicolumn{3}{|l|}{\em Optional Data} \\
\hline
\texttt{\f$N_{Poly}\f$} (\f$N_{Poly}\f$) & \texttt{int} &
 the number of coefficients of the polynomial stored in radVel . It has to be \\f\\$ \\ge 1 \\f\\$.   \\
\texttt{radVel}  & \texttt{double [numPolyRadVel] } &
  the coefficients of a polynomial expressing a radial velocity as a function of the time expressed in m/s. The time origin used to tabulate the polynomial is stored in timeOrigin.    \\
\hline
\end{longtable}
  
~\par\noindent{\bf Column Descriptions:}

\begin{description}
\item[\texttt{timeInterval}]: {\red long doc missing}
\item[\texttt{ephemerisId}]: {\red long doc missing}
\item[\texttt{observerLocation}]: {\red long doc missing}
\item[\texttt{equinoxEquator}]: {\red long doc missing}
\item[\texttt{\f$N_{Poly}\f$}]: {\red long doc missing}
\item[\texttt{dir}]: {\red long doc missing}
\item[\texttt{\f$N_{Poly}\f$}]: {\red long doc missing}
\item[\texttt{distance}]: {\red long doc missing}
\item[\texttt{timeOrigin}]: {\red long doc missing}
\item[\texttt{origin}]:  Typically one should see here e.g. a JPL identifier, eventually orbital parameters, etc...". for example, one might see in that string: <br/>
origin = 'JPL Horizons - DE405,JUP230'
<br/>
In any case, the observing system of ALMA or VLA should feel free to put in there whatever string they think best describes the information.
\item[\texttt{\f$N_{Poly}\f$}]: {\red long doc missing}
\item[\texttt{radVel}]: {\red long doc missing}
\end{description}
\endgroup

 \newpage

\subsection{ExecBlock Table}

 
 Characteristics of the Execution block.

\begingroup
%
% define shortcuts for dimensions

%\newcommand{\numObservingLog}{\f$N_{Obse}\f$}
%\newcommand{\numAntenna}{\f$N_{Ante}\f$}

\par\noindent\begin{longtable} {|p{45mm}|p{45mm}|p{65mm}|}
\hline \multicolumn{3}{|c|}{\textbf{ExecBlock}} \\
\hline\hline
Name & Type (Shape) & Comment \\
\hline \endfirsthead
\hline \multicolumn{3}{|c|}{\textbf{ExecBlock} -- continued from previous page} \\
\hline \hline
Name & Type (Shape) & Comment \\
\hline \endhead
\hline \endfoot


\hline \multicolumn{3}{|l|}{\em Key} \\
\hline 

\texttt{execBlockId} & \texttt{Tag} &  identifies a unique row in ExecBlock Table. \\
\hline \multicolumn{3}{|l|}{\em Required Data} \\
\hline
\texttt{startTime} & \texttt{ArrayTime} &
 the start time of the execution block. \\
\texttt{endTime} & \texttt{ArrayTime} &
 the end time of the execution block. \\
\texttt{execBlockNum} & \texttt{int} &
 indicates the position of the execution block in the project (sequential numbering starting at 1). \\
\texttt{execBlockUID} & \texttt{EntityRef} &
 the archive's UID of the execution block. \\
\texttt{projectUID} & \texttt{EntityRef} &
 the archive's UID of the project. \\
\texttt{configName} & \texttt{string} &
 the name of the array's configuration. \\
\texttt{telescopeName} & \texttt{string} &
 the name of the telescope. \\
\texttt{observerName} & \texttt{string} &
 the name of the observer. \\
\texttt{\f$N_{Obse}\f$} (\f$N_{Obse}\f$)& \texttt{int} &
 the number of elements in the (array) attribute observingLog.  \\
\texttt{observingLog} & \texttt{string [numObservingLog] } &
  logs of the observation during this execution block. \\
\texttt{sessionReference} & \texttt{EntityRef} &
 the observing session reference. \\
\texttt{baseRangeMin} & \texttt{Length} &
 the length of the shortest baseline. \\
\texttt{baseRangeMax} & \texttt{Length} &
 the length of the longest baseline. \\
\texttt{baseRmsMinor} & \texttt{Length} &
 the minor axis of the representative ellipse of baseline lengths. \\
\texttt{baseRmsMajor} & \texttt{Length} &
 the major axis of the representative ellipse of baseline lengths. \\
\texttt{basePa} & \texttt{Angle} &
 the baselines position angle. \\
\texttt{aborted} & \texttt{bool} &
 the execution block has been aborted (true) or has completed (false). \\
\texttt{\f$N_{Ante}\f$} (\f$N_{Ante}\f$)& \texttt{int} &
 the number of antennas. \\
\texttt{antennaId} & \texttt{Tag [numAntenna] } &
 refers to the relevant rows in AntennaTable. \\
\texttt{sBSummaryId} & \texttt{Tag} &
 refers to a unique row  in SBSummaryTable. \\

\hline \multicolumn{3}{|l|}{\em Optional Data} \\
\hline
\texttt{releaseDate}  & \texttt{ArrayTime} &
 the date when the data go to the public domain. \\
\texttt{schedulerMode}  & \texttt{string} &
 the mode of scheduling. \\
\texttt{siteAltitude}  & \texttt{Length} &
 the altitude of the site. \\
\texttt{siteLongitude}  & \texttt{Angle} &
 the longitude of the site. \\
\texttt{siteLatitude}  & \texttt{Angle} &
 the latitude of the site. \\
\texttt{observingScript}  & \texttt{string} &
 The text of the observation script.  \\
\texttt{observingScriptUID}  & \texttt{EntityRef} &
 A reference to the Entity which contains the observing script.  \\
\texttt{scaleId}  & \texttt{Tag} &
 refers to a unique row in the table Scale.  \\
\hline
\end{longtable}
  
~\par\noindent{\bf Column Descriptions:}

\begin{description}
\item[\texttt{execBlockId}]: Identifies the row in the ExecBlock Table. 
\item[\texttt{startTime}]: Scheduled time of start of data taking. 
\item[\texttt{endTime}]: Scheduled time of end of data taking. 
\item[\texttt{execBlockNum}]: Number of the ExecBlock: in ALMA Execution     blocks in a project are consecutively numbered starting from 1. 
\item[\texttt{execBlockUID}]: Archive UID of the ExecBlock (the container     of the data set). 
\item[\texttt{projectUID}]: The archive UID of the Project. 
\item[\texttt{configName}]: Name of the array baseline configuration. 
\item[\texttt{telescopeName}]:  Name of the telescope (e.g. 'ALMA') 
\item[\texttt{observerName}]: Name of the observer. 
\item[\texttt{\f$N_{Obse}\f$}]: {\red long doc missing}
\item[\texttt{observingLog}]: Logs of observations (information     entered at execution time by the Operator). 
\item[\texttt{sessionReference}]: This is useful for grouping     execblocks. Data capture know the session reference so this information     is easily available. 
\item[\texttt{baseRangeMin}]: Length of the minimum baseline.      For Alma this is expected to be filled from the unprojected baselines available in the array being     used in this ExecBlock. 
\item[\texttt{baseRangeMax}]: Length of the maximum baseline. For Alma this is expected to     be filled from the unprojected baselines available in the array being     used in this ExecBlock. 
\item[\texttt{baseRmsMinor}]: Minor axis of the representative ellipse of     baseline lengths. For Alma this is expected to be filled from the     unprojected baselines available in the array being used in this     ExecBlock. 
\item[\texttt{baseRmsMajor}]: Major axis of the representative ellipse of     baseline lengths. For Alma this is expected to be filled from the     unprojected baselines available in the array being used in this     ExecBlock. 
\item[\texttt{basePa}]: Position angle of the major axis on the     representative ellipse of baseline lengths. For Alma this is expected to     be filled from the unprojected baselines available in the array being     used in this ExecBlock. 
\item[\texttt{aborted}]: Set when the execution was aborted. 
\item[\texttt{\f$N_{Ante}\f$}]: Number of antennas used in the ExecBlock. 
\item[\texttt{antennaId}]: Antenna Table identifier. 
\item[\texttt{sBSummaryId}]: SBSummary Table identifier. 
\item[\texttt{releaseDate}]: The date when the data will become public. 
\item[\texttt{schedulerMode}]: Mode of the Scheduling when this data was     taken (Dynamic, Interactive, ...) 
\item[\texttt{siteAltitude}]: Latitude of the site (array center). 
\item[\texttt{siteLongitude}]: Longitude of the site (array center). 
\item[\texttt{siteLatitude}]: Latitude of the site (array center). 
\item[\texttt{observingScript}]: {\red long doc missing}
\item[\texttt{observingScriptUID}]: {\red long doc missing}
\item[\texttt{scaleId}]: {\red long doc missing}
\end{description}
\endgroup

 \newpage

\subsection{Feed Table}

 
 Contains characteristics of the feeds.

\begingroup
%
% define shortcuts for dimensions

%\newcommand{\numReceptor}{\f$N_{Rece}\f$}
%\newcommand{\numChan}{\f$N_{Chan}\f$}

\par\noindent\begin{longtable} {|p{45mm}|p{45mm}|p{65mm}|}
\hline \multicolumn{3}{|c|}{\textbf{Feed}} \\
\hline\hline
Name & Type (Shape) & Comment \\
\hline \endfirsthead
\hline \multicolumn{3}{|c|}{\textbf{Feed} -- continued from previous page} \\
\hline \hline
Name & Type (Shape) & Comment \\
\hline \endhead
\hline \endfoot


\hline \multicolumn{3}{|l|}{\em Key} \\
\hline 

\texttt{antennaId} & \texttt{Tag} &  refers to a unique row in AntennaTable. \\
\texttt{spectralWindowId} & \texttt{Tag} &  refers to a unique row in SpectralWindowTable. \\
\texttt{timeInterval} & \texttt{ArrayTimeInterval} &  the time interval of validity of the content of the row. \\
\texttt{feedId} & \texttt{int} &  identifies a collection of rows in the table. \\
\hline \multicolumn{3}{|l|}{\em Required Data} \\
\hline
\texttt{\f$N_{Rece}\f$} (\f$N_{Rece}\f$)& \texttt{int} &
 the number of receptors. \\
\texttt{beamOffset} & \texttt{double [numReceptor]  [2] } &
 the offsets of the beam (one pair per receptor). \\
\texttt{focusReference} & \texttt{Length [numReceptor]  [3] } &
 the references for the focus position (one triple per receptor). \\
\texttt{polarizationTypes} & \texttt{PolarizationType [numReceptor] } &
 identifies the polarization types (one value per receptor). \\
\texttt{polResponse} & \texttt{Complex [numReceptor]  [numReceptor] } &
 the polarization response (one value per pair of receptors). \\
\texttt{receptorAngle} & \texttt{Angle [numReceptor] } &
 the receptors angles (one value per receptor). \\
\texttt{receiverId} & \texttt{int [numReceptor] } &
 refers to one or more collections of rows in ReceiverTable. \\

\hline \multicolumn{3}{|l|}{\em Optional Data} \\
\hline
\texttt{feedNum}  & \texttt{int} &
 the feed number to be used for multi-feed receivers. \\
\texttt{illumOffset}  & \texttt{Length [2] } &
 the illumination offset. \\
\texttt{position}  & \texttt{Length [3] } &
 the position of the feed. \\
\texttt{skyCoupling}  & \texttt{float} &
 the sky coupling is the coupling efficiency to the sky of the WVR radiometer's. Note that in general one expects to see whether \\b no sky coupling efficiency recorded or \\b only \\b one of the two forms  scalar (skyCoupling) or array (skyCouplingSpectrum, numChan).  \\
\texttt{\f$N_{Chan}\f$} (\f$N_{Chan}\f$) & \texttt{int} &
 the size of skyCouplingSpectrum. This attribute must be present when the (array) attribute skyCouplingSpectrum is present since it defines its number of elements. The value of this attribute must be equal to the value of numChan in the row of the SpectralWindow table refered to by spectralWindowId.  \\
\texttt{skyCouplingSpectrum}  & \texttt{float [numChan] } &
 the sky coupling is the coupling efficiency to the sky of the WVR radiometer's. This column differs from the skyCoupling column because it contains one value for each of the individual channels of that spectralWindow. See the documentation of numChan for the size and the presence of this attribute. Note that in general one expects to see whether \\b no sky coupling efficiency recorded or \\b only \\b one of the two forms  scalar (skyCoupling) or array (skyCouplingSpectrum, numChan).  \\
\hline
\end{longtable}
  
~\par\noindent{\bf Column Descriptions:}

\begin{description}
\item[\texttt{antennaId}]: Antenna Table identifier. 
\item[\texttt{spectralWindowId}]: SpectralWindow Table identifier. 
\item[\texttt{timeInterval}]: Time Interval of validity of the feed information. 
\item[\texttt{feedId}]: Feed Table identifier. 
\item[\texttt{\f$N_{Rece}\f$}]: The number of receptors for which the result is given. 
\item[\texttt{beamOffset}]: Offset of the beam. 
\item[\texttt{focusReference}]: Reference for the focus position. 
\item[\texttt{polarizationTypes}]: The polarization types of the receptors being used. 
\item[\texttt{polResponse}]: The polarization response of the receptors. 
\item[\texttt{receptorAngle}]: Position angle for X polarization direction. 
\item[\texttt{receiverId}]: Points to the receivers corresponding to this feed. 
\item[\texttt{feedNum}]: Feed number is to be used for multi-feed     receivers (there are none in ALMA; so \\texttt{feedNum} is always one). 
\item[\texttt{illumOffset}]: Illumination offset for this feed,        measured in linear distance from the center of the primary reflector.       \\MPosition{REFLECTOR}  
\item[\texttt{position}]: The position of the feed. 
\item[\texttt{skyCoupling}]: {\red long doc missing}
\item[\texttt{\f$N_{Chan}\f$}]: {\red long doc missing}
\item[\texttt{skyCouplingSpectrum}]: {\red long doc missing}
\end{description}
\endgroup

 \newpage

\subsection{Field Table}

 
 The field position for each source.

\begingroup
%
% define shortcuts for dimensions

%\newcommand{\numPoly}{\f$N_{Poly}\f$}

\par\noindent\begin{longtable} {|p{45mm}|p{45mm}|p{65mm}|}
\hline \multicolumn{3}{|c|}{\textbf{Field}} \\
\hline\hline
Name & Type (Shape) & Comment \\
\hline \endfirsthead
\hline \multicolumn{3}{|c|}{\textbf{Field} -- continued from previous page} \\
\hline \hline
Name & Type (Shape) & Comment \\
\hline \endhead
\hline \endfoot


\hline \multicolumn{3}{|l|}{\em Key} \\
\hline 

\texttt{fieldId} & \texttt{Tag} &  identifies a unique row in the table. \\
\hline \multicolumn{3}{|l|}{\em Required Data} \\
\hline
\texttt{fieldName} & \texttt{string} &
 the name of the field. \\
\texttt{\f$N_{Poly}\f$} (\f$N_{Poly}\f$)& \texttt{int} &
 number of coefficients of the polynomials. \\
\texttt{delayDir} & \texttt{Angle [numPoly]  [2] } &
 the delay tracking direction. \\
\texttt{phaseDir} & \texttt{Angle [numPoly]  [2] } &
 the phase tracking direction. \\
\texttt{referenceDir} & \texttt{Angle [numPoly]  [2] } &
 the reference direction. \\

\hline \multicolumn{3}{|l|}{\em Optional Data} \\
\hline
\texttt{time}  & \texttt{ArrayTime} &
 value used as the origin for the polynomials. \\
\texttt{code}  & \texttt{string} &
 describes the function of the field. \\
\texttt{directionCode}  & \texttt{DirectionReferenceCode} &
 the direction reference code of the field. \\
\texttt{directionEquinox}  & \texttt{ArrayTime} &
 the direction reference equinox of the field. \\
\texttt{assocNature}  & \texttt{string} &
 identifies the nature of the association with the row refered to by fieldId. \\
\texttt{ephemerisId}  & \texttt{int} &
 refers to a collection of rows in the EphemerisTable.  \\
\texttt{sourceId}  & \texttt{int} &
 refers to a collection of rows in SourceTable. \\
\texttt{assocFieldId}  & \texttt{Tag} &
 Associated Field ID \\
\hline
\end{longtable}
  
~\par\noindent{\bf Column Descriptions:}

\begin{description}
\item[\texttt{fieldId}]: Field Table identifier. 
\item[\texttt{fieldName}]: Name of this Field (usually a representative     source, or one of several fields in a mosaic). 
\item[\texttt{\f$N_{Poly}\f$}]: Number of coefficients used for polynomial     expansion of tracked directions. 
\item[\texttt{delayDir}]: Direction in the sky for which delays due to Earth motion    are compensated in real time    \\MDirection{directionCode}{directionEquinox}{-} 
\item[\texttt{phaseDir}]: Direction in the sky for which phases due to     Earth motion are tracked in real time      \\MDirection{directionCode}{directionEquinox}{-} 
\item[\texttt{referenceDir}]: Direction of reference.   \\\\- In Interferometry this is the correlated field center (common pointing    direction for all antennas)   \\\\- In single dish this is the reference direction     \\MDirection{directionCode}{directionEquinox}{-} 
\item[\texttt{time}]: Used as an origin for expansion polynomials for tracked directions. 
\item[\texttt{code}]: Used to identify the function of this field (target,     calibration, etc.). Purely informative. 
\item[\texttt{directionCode}]: The common reference code for    field directions in the row, if not \\texttt{J2000}. 
\item[\texttt{directionEquinox}]: The common reference equinox for      field directions in the row,   if required by \\texttt{directionCode}. 
\item[\texttt{assocNature}]: Gives the meaning of Associated Field rows. 
\item[\texttt{ephemerisId}]: {\red long doc missing}
\item[\texttt{sourceId}]: Source Table identifier. 
\item[\texttt{assocFieldId}]: Refers to a unique associate row in the table. 
\end{description}
\endgroup

 \newpage

\subsection{Flag Table}

 
 This table is used for flagging visibility data and is used in addition to the Binary Data Format flags produced by the correlator software.

\begingroup
%
% define shortcuts for dimensions

%\newcommand{\numAntenna}{\f$N_{Ante}\f$}
%\newcommand{\numPolarizationType}{\f$N_{Pola}\f$}
%\newcommand{\numSpectralWindow}{\f$N_{Spec}\f$}
%\newcommand{\numPairedAntenna}{\f$N_{Pair}\f$}

\par\noindent\begin{longtable} {|p{45mm}|p{45mm}|p{65mm}|}
\hline \multicolumn{3}{|c|}{\textbf{Flag}} \\
\hline\hline
Name & Type (Shape) & Comment \\
\hline \endfirsthead
\hline \multicolumn{3}{|c|}{\textbf{Flag} -- continued from previous page} \\
\hline \hline
Name & Type (Shape) & Comment \\
\hline \endhead
\hline \endfoot


\hline \multicolumn{3}{|l|}{\em Key} \\
\hline 

\texttt{flagId} & \texttt{Tag} &  identifies a unique row in the table.  \\
\hline \multicolumn{3}{|l|}{\em Required Data} \\
\hline
\texttt{startTime} & \texttt{ArrayTime} &
 the start time of the flagging period.  \\
\texttt{endTime} & \texttt{ArrayTime} &
 the end time of the flagging period.  \\
\texttt{reason} & \texttt{string} &
 Extensible list of flagging conditions.  \\
\texttt{\f$N_{Ante}\f$} (\f$N_{Ante}\f$)& \texttt{int} &
 The number of antennas to which the flagging refers.By convention numAntenna== 0 means that the flag applies to all the existing antennas, in such a case the array antennaId can be left empty.  \\
\texttt{antennaId} & \texttt{Tag [numAntenna] } &
 An array of Tag which refers to a collection of rows in the Antenna table. The flag applies to the antennas described in these rows. It is an error to have different elements with a same value in this array.  \\

\hline \multicolumn{3}{|l|}{\em Optional Data} \\
\hline
\texttt{\f$N_{Pola}\f$} (\f$N_{Pola}\f$) & \texttt{int} &
 The number of polarization types , i.e. the size of the attribute polarizationType. By convention numPolarizationType == 0 means that the flag applies to all the defined polarization types. \\b Remark : numPolarizationType and polarizationType, both optional, must be both present or both absent in one same row of the table, except if numPolarizationType==0 in which case all the defined polarization types are involved in the flagging.  \\
\texttt{\f$N_{Spec}\f$} (\f$N_{Spec}\f$) & \texttt{int} &
 The number of spectral windows targeted by the flagging. By convention numSpectralWindow == 0 means that the flag applies to all the existing spectral windows. \\b Remark : numSpectralWindow and spectralWindow, both optional, must be both present or both absent in one same row of the table, except if numSpectralWindow==0, in which case all the declared spectral windows are involved in the flagging.  \\
\texttt{\f$N_{Pair}\f$} (\f$N_{Pair}\f$) & \texttt{int} &
 The number of antennas to be paired with to form the flagged baselines. By convention, numPairedAntenna == 0 means that the flag applies to all baselines built on the antennas declared in the attribute antennaId. \\b Remark: numPairedAntenna and pairedAntenna, both optional, must be both present or both absent except if numPairedAntenna==0 in which case one has to consider all the baselines defined upon the antennas announced in  antennaId.  \\
\texttt{polarizationType}  & \texttt{PolarizationType [numPolarizationType] } &
 An array of values of type PolarizationType. It specifies the polarization types where the flagging applies. It is an error to have different elements with a same value in this array.   \\
\texttt{pairedAntennaId}  & \texttt{Tag [numPairedAntenna] } &
 An array of Tag which refers to rows in the Antenna table. These rows contain the description of the antennas which are paired to form the flagged baselines. It is an error to have distinct elements with a same value in this array.  \\
\texttt{spectralWindowId}  & \texttt{Tag [numSpectralWindow] } &
 An array of Tag  which refers to a collection of rows in the SpectralWindow table. The flag applies to the spectral windows described in these rows. It is an error to have different elements with a same value in this array.    \\
\hline
\end{longtable}
  
~\par\noindent{\bf Column Descriptions:}

\begin{description}
\item[\texttt{flagId}]: {\red long doc missing}
\item[\texttt{startTime}]: {\red long doc missing}
\item[\texttt{endTime}]: {\red long doc missing}
\item[\texttt{reason}]: {\red long doc missing}
\item[\texttt{\f$N_{Ante}\f$}]: {\red long doc missing}
\item[\texttt{antennaId}]: {\red long doc missing}
\item[\texttt{\f$N_{Pola}\f$}]: {\red long doc missing}
\item[\texttt{\f$N_{Spec}\f$}]: {\red long doc missing}
\item[\texttt{\f$N_{Pair}\f$}]: {\red long doc missing}
\item[\texttt{polarizationType}]: {\red long doc missing}
\item[\texttt{pairedAntennaId}]: {\red long doc missing}
\item[\texttt{spectralWindowId}]: {\red long doc missing}
\end{description}
\endgroup

 \newpage


 \newpage

\subsection{Focus Table}

 
 Contains the focus information.

\begingroup
%
% define shortcuts for dimensions


\par\noindent\begin{longtable} {|p{45mm}|p{45mm}|p{65mm}|}
\hline \multicolumn{3}{|c|}{\textbf{Focus}} \\
\hline\hline
Name & Type (Shape) & Comment \\
\hline \endfirsthead
\hline \multicolumn{3}{|c|}{\textbf{Focus} -- continued from previous page} \\
\hline \hline
Name & Type (Shape) & Comment \\
\hline \endhead
\hline \endfoot


\hline \multicolumn{3}{|l|}{\em Key} \\
\hline 

\texttt{antennaId} & \texttt{Tag} &  refers to a unique row in AntennaTable. \\
\texttt{timeInterval} & \texttt{ArrayTimeInterval} &  time interval for which the row's content is valid. \\
\hline \multicolumn{3}{|l|}{\em Required Data} \\
\hline
\texttt{focusTracking} & \texttt{bool} &
 the focus motions have been tracked (true) or not tracked (false). \\
\texttt{focusOffset} & \texttt{Length [3] } &
 focus offset relative to the tracked position (a triple ). \\
\texttt{focusRotationOffset} & \texttt{Angle [2] } &
 focus rotation offset relative to the tracked position (tip, tilt).  \\
\texttt{focusModelId} & \texttt{int} &
 refers to a collection of rows in FocusModelTable. \\

\hline \multicolumn{3}{|l|}{\em Optional Data} \\
\hline
\texttt{measuredFocusPosition}  & \texttt{Length [3] } &
 the measured focus position. \\
\texttt{measuredFocusRotation}  & \texttt{Angle [2] } &
 the measured focus rotation (tip, tilt).  \\
\hline
\end{longtable}
  
~\par\noindent{\bf Column Descriptions:}

\begin{description}
\item[\texttt{antennaId}]: Antenna Table identifier. 
\item[\texttt{timeInterval}]: Time Interval of validity of the focus information. 
\item[\texttt{focusTracking}]: Set if the subreflector was tracking the focus motions. 
\item[\texttt{focusOffset}]: Focus offset introduced relative to the tracked position   \\MPositionOffset{REFLECTOR}{Virtual} 
\item[\texttt{focusRotationOffset}]: {\red long doc missing}
\item[\texttt{focusModelId}]: Specifies which Focus Model was used (FocusModel Table). 
\item[\texttt{measuredFocusPosition}]: Measured Focus position.   \\MPosition{REFLECTOR} 
\item[\texttt{measuredFocusRotation}]: {\red long doc missing}
\end{description}
\endgroup

 \newpage

\subsection{FocusModel Table}

 
 Contains the focus model data (function of elevation and temperature).

\begingroup
%
% define shortcuts for dimensions

%\newcommand{\numCoeff}{\f$N_{Coef}\f$}

\par\noindent\begin{longtable} {|p{45mm}|p{45mm}|p{65mm}|}
\hline \multicolumn{3}{|c|}{\textbf{FocusModel}} \\
\hline\hline
Name & Type (Shape) & Comment \\
\hline \endfirsthead
\hline \multicolumn{3}{|c|}{\textbf{FocusModel} -- continued from previous page} \\
\hline \hline
Name & Type (Shape) & Comment \\
\hline \endhead
\hline \endfoot


\hline \multicolumn{3}{|l|}{\em Key} \\
\hline 

\texttt{antennaId} & \texttt{Tag} &  refers to a unique row in AntennaTable. \\
\texttt{focusModelId} & \texttt{int} &  refers to a collection of rows in the table. \\
\hline \multicolumn{3}{|l|}{\em Required Data} \\
\hline
\texttt{polarizationType} & \texttt{PolarizationType} &
 identifies the polarization type. \\
\texttt{receiverBand} & \texttt{ReceiverBand} &
 identifies the receiver band. \\
\texttt{\f$N_{Coef}\f$} (\f$N_{Coef}\f$)& \texttt{int} &
 the number of coefficients. \\
\texttt{coeffName} & \texttt{string [numCoeff] } &
 the names of the coefficients (one string per coefficient). \\
\texttt{coeffFormula} & \texttt{string [numCoeff] } &
 textual representations of the fitted functions (one string per coefficient). \\
\texttt{coeffVal} & \texttt{float [numCoeff] } &
 the values of the coefficients used (one value per coefficient). \\
\texttt{assocNature} & \texttt{string} &
 nature of the association with the row refered to by associatedFocusModelId. \\
\texttt{assocFocusModelId} & \texttt{int} &
 refers to a collection of rows in the table. \\

\hline
\end{longtable}
  
~\par\noindent{\bf Column Descriptions:}

\begin{description}
\item[\texttt{antennaId}]: Antenna Table identifier. 
\item[\texttt{focusModelId}]: Identifies the focus model. 
\item[\texttt{polarizationType}]: Polarization component for which the focus model is valid. 
\item[\texttt{receiverBand}]: The name of the front-end frequency band being used. 
\item[\texttt{\f$N_{Coef}\f$}]: The number of coefficients in     the analytical form of the model. 
\item[\texttt{coeffName}]: Given names of coefficients. 
\item[\texttt{coeffFormula}]: Analytical formula:     explains the function fitted (e.g. \\$\\cos(el)\\$ or \\$\\ln(T)\\$). 
\item[\texttt{coeffVal}]: The values of the coefficients used. 
\item[\texttt{assocNature}]: Nature of     associated focus model,e.g., receiver-specific, local, ... 
\item[\texttt{assocFocusModelId}]: Associates another focus model     used in addition. Used for the auxiliary pointing model (e.g. the local     pointing model). The actual associated row is obtained by     associating the current \\texttt{antennaId} with     \\texttt{associatedFocusModelId} to form the key. 
\end{description}
\endgroup

 \newpage

\subsection{FreqOffset Table}

 
 Frequency offset information. Contains an additional antenna-based frequency  offset relative to the frequencies in the Spectral Windows. Useful for such  thing as Doppler tracking.

\begingroup
%
% define shortcuts for dimensions


\par\noindent\begin{longtable} {|p{45mm}|p{45mm}|p{65mm}|}
\hline \multicolumn{3}{|c|}{\textbf{FreqOffset}} \\
\hline\hline
Name & Type (Shape) & Comment \\
\hline \endfirsthead
\hline \multicolumn{3}{|c|}{\textbf{FreqOffset} -- continued from previous page} \\
\hline \hline
Name & Type (Shape) & Comment \\
\hline \endhead
\hline \endfoot


\hline \multicolumn{3}{|l|}{\em Key} \\
\hline 

\texttt{antennaId} & \texttt{Tag} &  refers to a unique row in AntennaTable. \\
\texttt{spectralWindowId} & \texttt{Tag} &  refers to a unique row in SpectralWindowTable. \\
\texttt{timeInterval} & \texttt{ArrayTimeInterval} &  the time interval of validity of the row's content. \\
\texttt{feedId} & \texttt{int} &  refers to a collection of rows in FeedTable. \\
\hline \multicolumn{3}{|l|}{\em Required Data} \\
\hline
\texttt{offset} & \texttt{Frequency} &
 frequency offset. \\

\hline
\end{longtable}
  
~\par\noindent{\bf Column Descriptions:}

\begin{description}
\item[\texttt{antennaId}]: Antenna Table identifier. 
\item[\texttt{spectralWindowId}]: SpectralWindow Table identifier. 
\item[\texttt{timeInterval}]: Time Interval of validity of the frequency offset information. 
\item[\texttt{feedId}]: Specifies which feed was used in the Feed Table. 
\item[\texttt{offset}]: Frequency offset to be added to the frequency set in the    spectral window Table. 
\end{description}
\endgroup

 \newpage

\subsection{GainTracking Table}

 
 Gain tracking information. Contains variable control parameters   affecting the signal coming from a receiver in an antenna. 

\begingroup
%
% define shortcuts for dimensions

%\newcommand{\numReceptor}{\f$N_{Rece}\f$}
%\newcommand{\numAttFreq}{\f$N_{AttF}\f$}

\par\noindent\begin{longtable} {|p{45mm}|p{45mm}|p{65mm}|}
\hline \multicolumn{3}{|c|}{\textbf{GainTracking}} \\
\hline\hline
Name & Type (Shape) & Comment \\
\hline \endfirsthead
\hline \multicolumn{3}{|c|}{\textbf{GainTracking} -- continued from previous page} \\
\hline \hline
Name & Type (Shape) & Comment \\
\hline \endhead
\hline \endfoot


\hline \multicolumn{3}{|l|}{\em Key} \\
\hline 

\texttt{antennaId} & \texttt{Tag} &  refers to a unique row in AntennaTable. \\
\texttt{spectralWindowId} & \texttt{Tag} &  refers to a unique row in SpectralWindowTable. \\
\texttt{timeInterval} & \texttt{ArrayTimeInterval} &  time interval for which the row's content is valid. \\
\texttt{feedId} & \texttt{int} &  refers to a unique row in Feed Table \\
\hline \multicolumn{3}{|l|}{\em Required Data} \\
\hline
\texttt{\f$N_{Rece}\f$} (\f$N_{Rece}\f$)& \texttt{int} &
 the number of receptors. \\
\texttt{attenuator} & \texttt{float [numReceptor] } &
 the nominal value of the attenuator (one value per receptor). \\
\texttt{polarizationType} & \texttt{PolarizationType [numReceptor] } &
 describes the polarizations of the receptors (one value per receptor). \\

\hline \multicolumn{3}{|l|}{\em Optional Data} \\
\hline
\texttt{samplingLevel}  & \texttt{float} &
 the sampling level. \\
\texttt{\f$N_{AttF}\f$} (\f$N_{AttF}\f$) & \texttt{int} &
 the sizes of attSpectrum and attFreq. \\
\texttt{attFreq}  & \texttt{double [numAttFreq] } &
 the attenuator frequencies. \\
\texttt{attSpectrum}  & \texttt{Complex [numAttFreq] } &
 the attenuator's measured spectrum. \\
\hline
\end{longtable}
  
~\par\noindent{\bf Column Descriptions:}

\begin{description}
\item[\texttt{antennaId}]: Antenna Table identifier. 
\item[\texttt{spectralWindowId}]: SpectralWindow Table identifier. 
\item[\texttt{timeInterval}]: Time Interval of validity of the feed   information. 
\item[\texttt{feedId}]: Specifies which feed was used in the Feed Table. 
\item[\texttt{\f$N_{Rece}\f$}]: The number of receptors. 
\item[\texttt{attenuator}]: Gain due to the hardware attenuation selected   for the Spectral window in this antenna. This is the nominal value of the attenuator. 
\item[\texttt{polarizationType}]: The polarization types of the receptors being used. 
\item[\texttt{samplingLevel}]: {Correlator sampling level.} Cannot   change for ALMA... 
\item[\texttt{\f$N_{AttF}\f$}]: Number of frequency points in      \\texttt{attSpectrum} 
\item[\texttt{attFreq}]: Frequencies for the values in   \\texttt{attSpectrum}. 
\item[\texttt{attSpectrum}]: Gain due to the hardware     attenuation selected for the Spectral window in this antenna. This is     the actual calibrated spectrum measured in the lab (complex values) as a     function of frequency. 
\end{description}
\endgroup

 \newpage


 \newpage

\subsection{Holography Table}

 
 Used for Single-Dish holography with a fixed transmitter.

\begingroup
%
% define shortcuts for dimensions

%\newcommand{\numCorr}{\f$N_{Corr}\f$}

\par\noindent\begin{longtable} {|p{45mm}|p{45mm}|p{65mm}|}
\hline \multicolumn{3}{|c|}{\textbf{Holography}} \\
\hline\hline
Name & Type (Shape) & Comment \\
\hline \endfirsthead
\hline \multicolumn{3}{|c|}{\textbf{Holography} -- continued from previous page} \\
\hline \hline
Name & Type (Shape) & Comment \\
\hline \endhead
\hline \endfoot


\hline \multicolumn{3}{|l|}{\em Key} \\
\hline 

\texttt{holographyId} & \texttt{Tag} &  identifies a unique row in the table. \\
\hline \multicolumn{3}{|l|}{\em Required Data} \\
\hline
\texttt{distance} & \texttt{Length} &
 the distance to transmitter. \\
\texttt{focus} & \texttt{Length} &
 displacement of the feed from the primary nominal focus. \\
\texttt{\f$N_{Corr}\f$} (\f$N_{Corr}\f$)& \texttt{int} &
 the number of stored correlations. \\
\texttt{type} & \texttt{HolographyChannelType [numCorr] } &
 identifies the types of the correlation signals. \\

\hline
\end{longtable}
  
~\par\noindent{\bf Column Descriptions:}

\begin{description}
\item[\texttt{holographyId}]: Identifies the row in the Holography Table. 
\item[\texttt{distance}]: Distance from intersection of mount axes to transmitter. 
\item[\texttt{focus}]: Displacement of signal feed from the primary nominal focus, used to compensate for the finite distance of transmitter. 
\item[\texttt{\f$N_{Corr}\f$}]: Number of correlations  stored (3 autocorrelations, 3 correlations from the 3 receptors (signal feed, reference feed, quadrature signal feed). 
\item[\texttt{type}]: Identifies each of the numCorr correlation signals. 
\end{description}
\endgroup

 \newpage


 \newpage

\subsection{Pointing Table}

 
 Antenna pointing information.

\begingroup
%
% define shortcuts for dimensions

%\newcommand{\numSample}{\f$N_{Samp}\f$}
%\newcommand{\numTerm}{\f$N_{Term}\f$}

\par\noindent\begin{longtable} {|p{45mm}|p{45mm}|p{65mm}|}
\hline \multicolumn{3}{|c|}{\textbf{Pointing}} \\
\hline\hline
Name & Type (Shape) & Comment \\
\hline \endfirsthead
\hline \multicolumn{3}{|c|}{\textbf{Pointing} -- continued from previous page} \\
\hline \hline
Name & Type (Shape) & Comment \\
\hline \endhead
\hline \endfoot


\hline \multicolumn{3}{|l|}{\em Key} \\
\hline 

\texttt{antennaId} & \texttt{Tag} &  refers to a unique row in AntennaTable. \\
\texttt{timeInterval} & \texttt{ArrayTimeInterval} &  the time interval of validity of the row's content. \\
\hline \multicolumn{3}{|l|}{\em Required Data} \\
\hline
\texttt{\f$N_{Samp}\f$} (\f$N_{Samp}\f$)& \texttt{int} &
 the number of time samples. \\
\texttt{encoder} & \texttt{Angle [numSample]  [2] } &
 Encoder values \\
\texttt{pointingTracking} & \texttt{bool} &
 the antenna was in tracking mode (true) or not (false). \\
\texttt{usePolynomials} & \texttt{bool} &
 use polynomials expansions (true) or not (false). \\
\texttt{timeOrigin} & \texttt{ArrayTime} &
 the value used as origin in the polynomials expansions. \\
\texttt{\f$N_{Term}\f$} (\f$N_{Term}\f$)& \texttt{int} &
 the number of terms of the polynomials. \\
\texttt{pointingDirection} & \texttt{Angle [numTerm]  [2] } &
 the commanded pointing direction. \\
\texttt{target} & \texttt{Angle [numTerm]  [2] } &
 the direction of the target. \\
\texttt{offset} & \texttt{Angle [numTerm]  [2] } &
 Horizon mapping offsets \\
\texttt{pointingModelId} & \texttt{int} &
 refers to a collection of rows in PointingModelTable. \\

\hline \multicolumn{3}{|l|}{\em Optional Data} \\
\hline
\texttt{overTheTop}  & \texttt{bool} &
 pointing ar elevations larger than 90 degrees (true) or lower (false). \\
\texttt{sourceOffset}  & \texttt{Angle [numTerm]  [2] } &
 sources offsets (one pair per term of the polynomial). \\
\texttt{sourceOffsetReferenceCode}  & \texttt{DirectionReferenceCode} &
 the  direction reference code associated to the source offset. \\
\texttt{sourceOffsetEquinox}  & \texttt{ArrayTime} &
 the equinox information (if needed by sourceReferenceCode). \\
\texttt{sampledTimeInterval}  & \texttt{ArrayTimeInterval [numSample] } &
 an array of ArrayTimeInterval which must be given explicitly as soon as the data are irregularily sampled.   \\
\texttt{atmosphericCorrection}  & \texttt{Angle [numTerm]  [2] } &
 This is the correction applied to the commanded position to take into account refraction and any other atmospheric effects. This term will always be zero if there is no atmosphere. For ALMA this is the atmospheric refraction correction and will result in a correction in just the elevation axis.  \\
\hline
\end{longtable}
  
~\par\noindent{\bf Column Descriptions:}

\begin{description}
\item[\texttt{antennaId}]: Antenna Table identifier. 
\item[\texttt{timeInterval}]: Time Interval of  validity of the feed information. 
\item[\texttt{\f$N_{Samp}\f$}]: The number of time samples for   \\texttt{encoder}. The sampling intervals divide \\texttt{timeInterval} into   \\numSample\\ contiguous intervals of equal duration. This is also used for   the other direction columns (\\texttt{offset}, \\texttt{pointingDirection},   \\texttt{target}, \\texttt{sourceOffset}) when \\texttt{usePolynomials} is   \\texttt{false}: in that case \\numTerm = \\numSample . 
\item[\texttt{encoder}]: The values measured from the antenna. They may be   however affected by metrology, if applied. Note that for ALMA this column   will contain positions obtained using the AZ\\_POSN\\_RSP and EL\\_POSN\\_RSP   monitor points of the ACU and not the GET\\_AZ\\_ENC and GET\\_EL\\_ENC monitor   points (as these do not include the metrology corrections). It is agreed   that the the vendor pointing model will never be applied.   \\MDirection{AZEL}{NOW}{Antenna.position} 
\item[\texttt{pointingTracking}]: Indicates that the antenna is in tracking mode. 
\item[\texttt{usePolynomials}]: See \\texttt{numSample} and \\texttt{numTerm}. 
\item[\texttt{timeOrigin}]: The time origin for polynomial expansions of   \\texttt{pointingDirection}, \\texttt{target}, \\texttt{offset}, and \\texttt{sourceOffset}. Equal to the midpoint of \\texttt{timeInterval} if \\numTerm=1. 
\item[\texttt{\f$N_{Term}\f$}]: \\begin{itemize}      \\item If \\texttt{usePolynomials} is \\texttt{false}: \\numTerm =      \\numSample, and the values in the direction columns (\\texttt{offset},      \\texttt{pointingDirection}, \\texttt{target}, \\texttt{sourceOffset})      correspond to the same sampling intervals used for \\texttt{encoder}.     \\item If \\texttt{usePolynomials} is \\texttt{true}: \\numTerm is the number      of terms in the polynomial expansion for the direction columns      (\\texttt{offset}, \\texttt{pointingDirection}, \\texttt{target},      \\texttt{sourceOffset}) . The time origin for these polynomials is      \\texttt{timeOrigin}.      \\end{itemize} 
\item[\texttt{pointingDirection}]:  This is the commanded direction of the   antenna. It is obtained by adding the target and offset columns, and then   applying the pointing model referenced by PointingModelId. The pointing   model can be the composition of the absolute pointing model and of a local   pointing model. In that case their coefficients will both be in the   PointingModel table. 
\item[\texttt{target}]:  This is the field center direction (as given in   the Field Table), possibly affected by the optional {antenna-based   \\texttt{sourceOffset}}. This column is in horizontal coordinates.  \\MDirection{AZEL}{NOW}{Antenna.position} 
\item[\texttt{offset}]: Additional offsets in horizontal coordinates   (usually meant for measuring the pointing corrections, mapping the antenna   beam, ...).   \\MDirectionOffset{AZEL}{NOW}{Antenna.position}{target} 
\item[\texttt{pointingModelId}]: Link to the pointing model applied. 
\item[\texttt{overTheTop}]:  The antenna is pointing at elevations larger   than 90 degrees. Deprecated for ALMA, should not happen. 
\item[\texttt{sourceOffset}]: Optionally, the antenna-based mapping   offsets in the field. These are in the equatorial system, and used, for   instance, in on-the-fly mapping {when the antennas are driven   independently across the field. 
\item[\texttt{sourceOffsetReferenceCode}]: Source offset   reference code, defaults to J2000. 
\item[\texttt{sourceOffsetEquinox}]: Source offset equinox,     if needed by \\texttt{sourceOffsetReference}. 
\item[\texttt{sampledTimeInterval}]: {\red long doc missing}
\item[\texttt{atmosphericCorrection}]: {\red long doc missing}
\end{description}
\endgroup

 \newpage

\subsection{PointingModel Table}

 
 The pointing models used to point the antennas.

\begingroup
%
% define shortcuts for dimensions

%\newcommand{\numCoeff}{\f$N_{Coef}\f$}

\par\noindent\begin{longtable} {|p{45mm}|p{45mm}|p{65mm}|}
\hline \multicolumn{3}{|c|}{\textbf{PointingModel}} \\
\hline\hline
Name & Type (Shape) & Comment \\
\hline \endfirsthead
\hline \multicolumn{3}{|c|}{\textbf{PointingModel} -- continued from previous page} \\
\hline \hline
Name & Type (Shape) & Comment \\
\hline \endhead
\hline \endfoot


\hline \multicolumn{3}{|l|}{\em Key} \\
\hline 

\texttt{antennaId} & \texttt{Tag} &  refers to a unique row in AntennaTable. \\
\texttt{pointingModelId} & \texttt{int} &  pointingModel identifier \\
\hline \multicolumn{3}{|l|}{\em Required Data} \\
\hline
\texttt{\f$N_{Coef}\f$} (\f$N_{Coef}\f$)& \texttt{int} &
 the number of coefficients used in the analytical form of the model. \\
\texttt{coeffName} & \texttt{string [numCoeff] } &
 the names of the coefficients. \\
\texttt{coeffVal} & \texttt{float [numCoeff] } &
 the values of the coefficients. \\
\texttt{polarizationType} & \texttt{PolarizationType} &
 identifies the polarization type. \\
\texttt{receiverBand} & \texttt{ReceiverBand} &
 identifies the receiver band. \\
\texttt{assocNature} & \texttt{string} &
 nature of the association with the row refered to by associatedPointingModelId. \\
\texttt{assocPointingModelId} & \texttt{int} &
 refers to a collection of rows in the table. \\

\hline \multicolumn{3}{|l|}{\em Optional Data} \\
\hline
\texttt{coeffFormula}  & \texttt{string [numCoeff] } &
 the fitted functions \\
\hline
\end{longtable}
  
~\par\noindent{\bf Column Descriptions:}

\begin{description}
\item[\texttt{antennaId}]: Antenna Table identifier. 
\item[\texttt{pointingModelId}]: Identifies the pointing model used. 
\item[\texttt{\f$N_{Coef}\f$}]: The number of coefficients in the analytical form of the model. 
\item[\texttt{coeffName}]: Standard names used by tpoint. 
\item[\texttt{coeffVal}]: The values of the coefficients used. 
\item[\texttt{polarizationType}]: Polarization component for which the pointing  model is valid. 
\item[\texttt{receiverBand}]: The name of the front-end frequency band being used. 
\item[\texttt{assocNature}]: Nature   of the associated pointing model: e.g. local pointing model,   receiver-specific. 
\item[\texttt{assocPointingModelId}]: Associates another pointing     model used in addition. Used for the auxiliary pointing model     (e.g. local pointing model, ...). The actual associated row is obtained     by associating the current \\texttt{antennaId} with     \\texttt{associatedPointingModelId} to form the key. 
\item[\texttt{coeffFormula}]: Analytical formulae. This is not     needed for ALMA as we use tpoint generic coefficients. 
\end{description}
\endgroup

 \newpage

\subsection{Polarization Table}

 
 Polarization information.

\begingroup
%
% define shortcuts for dimensions

%\newcommand{\numCorr}{\f$N_{Corr}\f$}

\par\noindent\begin{longtable} {|p{45mm}|p{45mm}|p{65mm}|}
\hline \multicolumn{3}{|c|}{\textbf{Polarization}} \\
\hline\hline
Name & Type (Shape) & Comment \\
\hline \endfirsthead
\hline \multicolumn{3}{|c|}{\textbf{Polarization} -- continued from previous page} \\
\hline \hline
Name & Type (Shape) & Comment \\
\hline \endhead
\hline \endfoot


\hline \multicolumn{3}{|l|}{\em Key} \\
\hline 

\texttt{polarizationId} & \texttt{Tag} &  Polarization Table identifier \\
\hline \multicolumn{3}{|l|}{\em Required Data} \\
\hline
\texttt{\f$N_{Corr}\f$} (\f$N_{Corr}\f$)& \texttt{int} &
 Number of correlation products \\
\texttt{corrType} & \texttt{StokesParameter [numCorr] } &
 Correlation type \\
\texttt{corrProduct} & \texttt{PolarizationType [numCorr]  [2] } &
 Correlation product. \\

\hline
\end{longtable}
  
~\par\noindent{\bf Column Descriptions:}

\begin{description}
\item[\texttt{polarizationId}]: Polarization Table identifier. 
\item[\texttt{\f$N_{Corr}\f$}]: The number of correlation products. This could be     1 (e.g.  XX or YY), 2 (e.g. XX and YY), {3 (full polarization for     auto-correlation)}, or 4 (full polarization for cross-correlation). 
\item[\texttt{corrType}]: For each correlation product this indicates the   Stokes type as defined in the Stokes parameters enumeration. 
\item[\texttt{corrProduct}]: For each correlation product, a pair of   integers, specifying the receptors from which the signal originated. The   polarization of each receptor is defined in the \\texttt{polarizationType}   column in the Feed table.  An example would be (0,0), (0,1), (1,0), (1,1)   to specify all possible correlation prodicts between two receptors. 
\end{description}
\endgroup

 \newpage

\subsection{Processor Table}

 
 Processor characteristics. This table holds summary information for the  back-end processing devices used to generate the basic science  data.

\begingroup
%
% define shortcuts for dimensions


\par\noindent\begin{longtable} {|p{45mm}|p{45mm}|p{65mm}|}
\hline \multicolumn{3}{|c|}{\textbf{Processor}} \\
\hline\hline
Name & Type (Shape) & Comment \\
\hline \endfirsthead
\hline \multicolumn{3}{|c|}{\textbf{Processor} -- continued from previous page} \\
\hline \hline
Name & Type (Shape) & Comment \\
\hline \endhead
\hline \endfoot


\hline \multicolumn{3}{|l|}{\em Key} \\
\hline 

\texttt{processorId} & \texttt{Tag} &  Processor identifier \\
\hline \multicolumn{3}{|l|}{\em Required Data} \\
\hline
\texttt{modeId} & \texttt{Tag} &
 refers to a unique row in CorrelatorModeTable or SquareLawDetectorTable or AlmaRadiometerTable. \\
\texttt{processorType} & \texttt{ProcessorType} &
 identifies the generic type of the processor. \\
\texttt{processorSubType} & \texttt{ProcessorSubType} &
 identifies the type of processor refered to by modeId. \\

\hline
\end{longtable}
  
~\par\noindent{\bf Column Descriptions:}

\begin{description}
\item[\texttt{processorId}]: Processor Table identifier. 
\item[\texttt{modeId}]: Processor table identifier. 
\item[\texttt{processorType}]: The generic type of processor used. 
\item[\texttt{processorSubType}]: Identifies the SDM table containing the  processor-dependent    information. 
\end{description}
\endgroup

 \newpage

\subsection{Receiver Table}

 
 Receiver properties.

\begingroup
%
% define shortcuts for dimensions

%\newcommand{\numLO}{\f$N_{LO}\f$}

\par\noindent\begin{longtable} {|p{45mm}|p{45mm}|p{65mm}|}
\hline \multicolumn{3}{|c|}{\textbf{Receiver}} \\
\hline\hline
Name & Type (Shape) & Comment \\
\hline \endfirsthead
\hline \multicolumn{3}{|c|}{\textbf{Receiver} -- continued from previous page} \\
\hline \hline
Name & Type (Shape) & Comment \\
\hline \endhead
\hline \endfoot


\hline \multicolumn{3}{|l|}{\em Key} \\
\hline 

\texttt{receiverId} & \texttt{int} &  Receiver identifier \\
\texttt{spectralWindowId} & \texttt{Tag} &  refers to a unique row in SpectralwindowTable. \\
\texttt{timeInterval} & \texttt{ArrayTimeInterval} &  time interval for which the content is valid. \\
\hline \multicolumn{3}{|l|}{\em Required Data} \\
\hline
\texttt{name} & \texttt{string} &
 the name of the frontend. \\
\texttt{\f$N_{LO}\f$} (\f$N_{LO}\f$)& \texttt{int} &
 the number of frequencies of the local oscillator. \\
\texttt{frequencyBand} & \texttt{ReceiverBand} &
 identifies the band of frequencies. \\
\texttt{freqLO} & \texttt{Frequency [numLO] } &
 the frequencies of the local oscillator. \\
\texttt{receiverSideband} & \texttt{ReceiverSideband} &
 the receiver sideband used. \\
\texttt{sidebandLO} & \texttt{NetSideband [numLO] } &
 the sideband conversions. \\

\hline
\end{longtable}
  
~\par\noindent{\bf Column Descriptions:}

\begin{description}
\item[\texttt{receiverId}]: Receiver  Table identifier. Note that this is always       zero for ALMA. 
\item[\texttt{spectralWindowId}]: SpectralWindow Table identifier. 
\item[\texttt{timeInterval}]: Time Interval ofy validity of the receiver information. 
\item[\texttt{name}]: Name of the frontend. 
\item[\texttt{\f$N_{LO}\f$}]: The number of frequency changes in the receiver chain. 
\item[\texttt{frequencyBand}]: The name of this frequency band (bands 1 to 10 for ALMA)         These correspond to receiver cartridges in the ALMA dewars. 
\item[\texttt{freqLO}]: Frequencies of the Local Oscillators in the receiving chain. 
\item[\texttt{receiverSideband}]: The receiver sideband used. 
\item[\texttt{sidebandLO}]: The sideband conversion for each of the local oscillators.       Used to check the frequency plan. 
\end{description}
\endgroup

 \newpage

\subsection{SBSummary Table}

 
 Characteristics of the Scheduling Block that has been executed. Much of the  data here is reproduced from the Scheduling block itself.

\begingroup
%
% define shortcuts for dimensions

%\newcommand{\numObservingMode}{\f$N_{Obse}\f$}
%\newcommand{\numberRepeats}{\f$N_{berR}\f$}
%\newcommand{\numScienceGoal}{\f$N_{Scie}\f$}
%\newcommand{\numWeatherConstraint}{\f$N_{Weat}\f$}

\par\noindent\begin{longtable} {|p{45mm}|p{45mm}|p{65mm}|}
\hline \multicolumn{3}{|c|}{\textbf{SBSummary}} \\
\hline\hline
Name & Type (Shape) & Comment \\
\hline \endfirsthead
\hline \multicolumn{3}{|c|}{\textbf{SBSummary} -- continued from previous page} \\
\hline \hline
Name & Type (Shape) & Comment \\
\hline \endhead
\hline \endfoot


\hline \multicolumn{3}{|l|}{\em Key} \\
\hline 

\texttt{sBSummaryId} & \texttt{Tag} &  refers to a unique row in the table. \\
\hline \multicolumn{3}{|l|}{\em Required Data} \\
\hline
\texttt{sbSummaryUID} & \texttt{EntityRef} &
 the scheduling block archive's UID. \\
\texttt{projectUID} & \texttt{EntityRef} &
 the projet archive's UID. \\
\texttt{obsUnitSetUID} & \texttt{EntityRef} &
 the observing unit set archive's UID. \\
\texttt{frequency} & \texttt{double} &
 a representative frequency. \\
\texttt{frequencyBand} & \texttt{ReceiverBand} &
 the frequency band. \\
\texttt{sbType} & \texttt{SBType} &
 the type of scheduling block. \\
\texttt{sbDuration} & \texttt{Interval} &
 the duration of the scheduling block. \\
\texttt{\f$N_{Obse}\f$} (\f$N_{Obse}\f$)& \texttt{int} &
 the number of observing modes. \\
\texttt{observingMode} & \texttt{string [numObservingMode] } &
 the observing modes. \\
\texttt{\f$N_{berR}\f$} (\f$N_{berR}\f$)& \texttt{int} &
 the number of repeats. \\
\texttt{\f$N_{Scie}\f$} (\f$N_{Scie}\f$)& \texttt{int} &
 the number of scientific goals. \\
\texttt{scienceGoal} & \texttt{string [numScienceGoal] } &
 the scientific goals. \\
\texttt{\f$N_{Weat}\f$} (\f$N_{Weat}\f$)& \texttt{int} &
 the number of weather constraints. \\
\texttt{weatherConstraint} & \texttt{string [numWeatherConstraint] } &
 the weather constraints. \\

\hline \multicolumn{3}{|l|}{\em Optional Data} \\
\hline
\texttt{centerDirection}  & \texttt{Angle [2] } &
 the representative target direction. \\
\texttt{centerDirectionCode}  & \texttt{DirectionReferenceCode} &
 identifies the direction reference frame associated with centerDirection. \\
\texttt{centerDirectionEquinox}  & \texttt{ArrayTime} &
 the equinox associated to centerDirectionReferenceCode (if needed). \\
\hline
\end{longtable}
  
~\par\noindent{\bf Column Descriptions:}

\begin{description}
\item[\texttt{sBSummaryId}]: Identifies each row of the SBSummary  table. 
\item[\texttt{sbSummaryUID}]: Archive UID of the scheduling block. 
\item[\texttt{projectUID}]: Archive UID of the project. 
\item[\texttt{obsUnitSetUID}]: Archive UID of the observing unit set. 
\item[\texttt{frequency}]: Representative frequency. 
\item[\texttt{frequencyBand}]: Frequency band used for the SB. 
\item[\texttt{sbType}]: Type of Scheduling Block (e.g. science, calibration, ...) 
\item[\texttt{sbDuration}]: The duration of this SB (as planned). 
\item[\texttt{\f$N_{Obse}\f$}]: The number of observing modes. 
\item[\texttt{observingMode}]: Standard observing mode, e.g. singleField,   mosaic. One SB cannot have more than one standard observing mode. Examples:   Single Field Interferometry, Pointed Mosaic, ... 
\item[\texttt{\f$N_{berR}\f$}]: Number of executions scheduled for this SB. 
\item[\texttt{\f$N_{Scie}\f$}]: Number of science goals. 
\item[\texttt{scienceGoal}]: The science goals. 
\item[\texttt{\f$N_{Weat}\f$}]: The number of weather constraints. 
\item[\texttt{weatherConstraint}]: Weather scheduling constraints, as specified in the Observing Tool. 
\item[\texttt{centerDirection}]: Direction of the    representative target. 
\item[\texttt{centerDirectionCode}]:  The center direction reference    code, if not \\texttt{J2000}. 
\item[\texttt{centerDirectionEquinox}]: The center direction reference    equinox, if required by \\texttt{centerDirectionCode}. 
\end{description}
\endgroup

 \newpage

\subsection{Scale Table}

 
 Specifies the time scales and units used in the dataset. 

\begingroup
%
% define shortcuts for dimensions


\par\noindent\begin{longtable} {|p{45mm}|p{45mm}|p{65mm}|}
\hline \multicolumn{3}{|c|}{\textbf{Scale}} \\
\hline\hline
Name & Type (Shape) & Comment \\
\hline \endfirsthead
\hline \multicolumn{3}{|c|}{\textbf{Scale} -- continued from previous page} \\
\hline \hline
Name & Type (Shape) & Comment \\
\hline \endhead
\hline \endfoot


\hline \multicolumn{3}{|l|}{\em Key} \\
\hline 

\texttt{scaleId} & \texttt{Tag} &  identifies a unique row in the table Scale.  \\
\hline \multicolumn{3}{|l|}{\em Required Data} \\
\hline
\texttt{timeScale} & \texttt{TimeScale} &
 a TimeScale value.  \\
\texttt{crossDataScale} & \texttt{DataScale} &
 the unit of the cross data in the BDF.  \\
\texttt{autoDataScale} & \texttt{DataScale} &
 the unit of the  auto data  in the BDF.  \\
\texttt{weightType} & \texttt{WeightType} &
 a  type of weighting.  \\

\hline
\end{longtable}
  
~\par\noindent{\bf Column Descriptions:}

\begin{description}
\item[\texttt{scaleId}]: {\red long doc missing}
\item[\texttt{timeScale}]: {\red long doc missing}
\item[\texttt{crossDataScale}]: {\red long doc missing}
\item[\texttt{autoDataScale}]: {\red long doc missing}
\item[\texttt{weightType}]: {\red long doc missing}
\end{description}
\endgroup

 \newpage

\subsection{Scan Table}

 
 A summary of information for each scan.

\begingroup
%
% define shortcuts for dimensions

%\newcommand{\numIntent}{\f$N_{Inte}\f$}
%\newcommand{\numSubscan}{\f$N_{Subs}\f$}
%\newcommand{\numField}{\f$N_{Fiel}\f$}

\par\noindent\begin{longtable} {|p{45mm}|p{45mm}|p{65mm}|}
\hline \multicolumn{3}{|c|}{\textbf{Scan}} \\
\hline\hline
Name & Type (Shape) & Comment \\
\hline \endfirsthead
\hline \multicolumn{3}{|c|}{\textbf{Scan} -- continued from previous page} \\
\hline \hline
Name & Type (Shape) & Comment \\
\hline \endhead
\hline \endfoot


\hline \multicolumn{3}{|l|}{\em Key} \\
\hline 

\texttt{execBlockId} & \texttt{Tag} &  refers to a unique row in ExecBlockTable. \\
\texttt{scanNumber} & \texttt{int} &  the scan number. \\
\hline \multicolumn{3}{|l|}{\em Required Data} \\
\hline
\texttt{startTime} & \texttt{ArrayTime} &
 the actual start time of the scan. \\
\texttt{endTime} & \texttt{ArrayTime} &
 the actual end time of the scan. \\
\texttt{\f$N_{Inte}\f$} (\f$N_{Inte}\f$)& \texttt{int} &
 the number of intents for this scan. \\
\texttt{\f$N_{Subs}\f$} (\f$N_{Subs}\f$)& \texttt{int} &
 the number of subscans contained by this scan. \\
\texttt{scanIntent} & \texttt{ScanIntent [numIntent] } &
 identifies the intents of this scan. \\
\texttt{calDataType} & \texttt{CalDataOrigin [numIntent] } &
 identifies the calibration data types (one value per intent). \\
\texttt{calibrationOnLine} & \texttt{bool [numIntent] } &
 the online calibration was required (true) or not (false) (one value per intent). \\

\hline \multicolumn{3}{|l|}{\em Optional Data} \\
\hline
\texttt{calibrationFunction}  & \texttt{CalibrationFunction [numIntent] } &
 identifies the calibration functions (one value per intent). \\
\texttt{calibrationSet}  & \texttt{CalibrationSet [numIntent] } &
 attaches this scan to a calibration set (one value per intent). \\
\texttt{calPattern}  & \texttt{AntennaMotionPattern [numIntent] } &
 identifies the antenna motion patterns used for the calibration. \\
\texttt{\f$N_{Fiel}\f$} (\f$N_{Fiel}\f$) & \texttt{int} &
 the number of fields observed. \\
\texttt{fieldName}  & \texttt{string [numField] } &
 the names of the observed fields (one value per field). \\
\texttt{sourceName}  & \texttt{string} &
 the name of the observed source. \\
\hline
\end{longtable}
  
~\par\noindent{\bf Column Descriptions:}

\begin{description}
\item[\texttt{execBlockId}]: ExecBlock Table identifier. 
\item[\texttt{scanNumber}]: The scan number. This increments starting from 1 in each Execution Block. 
\item[\texttt{startTime}]: The time when this scan actually started acquiring data. 
\item[\texttt{endTime}]: The time when this scan actually finished  acquiring data. 
\item[\texttt{\f$N_{Inte}\f$}]: The scan intents for this scan. 
\item[\texttt{\f$N_{Subs}\f$}]: The number of subscans used to achieve this scan. 
\item[\texttt{scanIntent}]: The scan intents for this scan. 
\item[\texttt{calDataType}]: Calibration data type used for data reduction. 
\item[\texttt{calibrationOnLine}]: True if on-line calibration was required. 
\item[\texttt{calibrationFunction}]: Function of the scan in   the calibration set. This is used in on-line calibration. 
\item[\texttt{calibrationSet}]: Identifies that the scan is   part of a set of scans aimed at a particular calibration. 
\item[\texttt{calPattern}]: Identifies the antenna motion   pattern used for calibration; data reduction may used this information. 
\item[\texttt{\f$N_{Fiel}\f$}]: The number of fields observed during this scan. 
\item[\texttt{fieldName}]: The names of of the fields observed during this scan. 
\item[\texttt{sourceName}]: The name of the source observed (optional). 
\end{description}
\endgroup

 \newpage


 \newpage

\subsection{Source Table}

 
 Summary of astromomical source information.

\begingroup
%
% define shortcuts for dimensions

%\newcommand{\numLines}{\f$N_{Line}\f$}
%\newcommand{\numFreq}{\f$N_{Freq}\f$}
%\newcommand{\numStokes}{\f$N_{Stok}\f$}

\par\noindent\begin{longtable} {|p{45mm}|p{45mm}|p{65mm}|}
\hline \multicolumn{3}{|c|}{\textbf{Source}} \\
\hline\hline
Name & Type (Shape) & Comment \\
\hline \endfirsthead
\hline \multicolumn{3}{|c|}{\textbf{Source} -- continued from previous page} \\
\hline \hline
Name & Type (Shape) & Comment \\
\hline \endhead
\hline \endfoot


\hline \multicolumn{3}{|l|}{\em Key} \\
\hline 

\texttt{sourceId} & \texttt{int} &  identifies a collection of rows in the table. \\
\texttt{timeInterval} & \texttt{ArrayTimeInterval} &   the time interval of validity of the row's content. \\
\texttt{spectralWindowId} & \texttt{Tag} &  refers to a unique row in SpectralWindowTable. \\
\hline \multicolumn{3}{|l|}{\em Required Data} \\
\hline
\texttt{code} & \texttt{string} &
 indicates the nature of the source. \\
\texttt{direction} & \texttt{Angle [2] } &
 the direction of the source. \\
\texttt{properMotion} & \texttt{AngularRate [2] } &
 the proper motion of the source. \\
\texttt{sourceName} & \texttt{string} &
 the name of the source. \\

\hline \multicolumn{3}{|l|}{\em Optional Data} \\
\hline
\texttt{directionCode}  & \texttt{DirectionReferenceCode} &
 identifies the direction reference frame associated to direction. \\
\texttt{directionEquinox}  & \texttt{ArrayTime} &
 the equinox associated to the direction reference frame (if required). \\
\texttt{calibrationGroup}  & \texttt{int} &
 the calibration group number. \\
\texttt{catalog}  & \texttt{string} &
 the name of the catalog. \\
\texttt{deltaVel}  & \texttt{Speed} &
 the velocity resolution. \\
\texttt{position}  & \texttt{Length [3] } &
 the position of the source. \\
\texttt{\f$N_{Line}\f$} (\f$N_{Line}\f$) & \texttt{int} &
 the number of line transitions. \\
\texttt{transition}  & \texttt{string [numLines] } &
 the names of the transitions. \\
\texttt{restFrequency}  & \texttt{Frequency [numLines] } &
 the rest frequencies (one value per transition line). \\
\texttt{sysVel}  & \texttt{Speed [numLines] } &
 the systemic velocity. \\
\texttt{rangeVel}  & \texttt{Speed [2] } &
 the velocity range. \\
\texttt{sourceModel}  & \texttt{SourceModel} &
 identifies the source model. \\
\texttt{frequencyRefCode}  & \texttt{FrequencyReferenceCode} &
 the frequency reference code. \\
\texttt{\f$N_{Freq}\f$} (\f$N_{Freq}\f$) & \texttt{int} &
 the number of frequencies. \\
\texttt{\f$N_{Stok}\f$} (\f$N_{Stok}\f$) & \texttt{int} &
 the number of Stokes parameters. \\
\texttt{frequency}  & \texttt{Frequency [numFreq] } &
 the array of frequencies (one value per frequency). \\
\texttt{frequencyInterval}  & \texttt{Frequency [numFreq] } &
 an array of frequency intervals (one value per interval). \\
\texttt{stokesParameter}  & \texttt{StokesParameter [numStokes] } &
 the array of Stokes parameters (one value per parameter). \\
\texttt{flux}  & \texttt{Flux [numFreq]  [numStokes] } &
 the array of flux densities expressed in Jansky (Jy). \\
\texttt{fluxErr}  & \texttt{Flux [numFreq]  [numStokes] } &
 the array of uncertainties on flux densities. \\
\texttt{positionAngle}  & \texttt{Angle [numFreq] } &
 the major axis position angles (one value per frequency). \\
\texttt{positionAngleErr}  & \texttt{Angle [numFreq] } &
 the uncertainties on major axis position angles. \\
\texttt{size}  & \texttt{Angle [numFreq]  [2] } &
 the sizes of source (one pair of values per frequency). \\
\texttt{sizeErr}  & \texttt{Angle [numFreq]  [2] } &
 the uncertainties on the source sizes (one pair of value per frequency). \\
\texttt{velRefCode}  & \texttt{RadialVelocityReferenceCode} &
 the velocity reference code for velocities: sysVel, rangeVel, deltaVel.  \\
\hline
\end{longtable}
  
~\par\noindent{\bf Column Descriptions:}

\begin{description}
\item[\texttt{sourceId}]: Identifies each row of the Source  table. 
\item[\texttt{timeInterval}]: Time Interval of  validity of the source information. 
\item[\texttt{spectralWindowId}]: SpectralWindow Table identifier. 
\item[\texttt{code}]: Source code, e.g., Bandpass calibrator. 
\item[\texttt{direction}]: The source direction, usually EQ J2000.  \\MDirection{directionCode}{directionEquinox}{-}  
\item[\texttt{properMotion}]: Source proper motion (change of directiuon with time). 
\item[\texttt{sourceName}]: The source name (please respect UAI/IAU standard conventions). 
\item[\texttt{directionCode}]: The source direction reference code,    if not \\texttt{J2000}. 
\item[\texttt{directionEquinox}]: The source direction reference equinox ,    if required by \\texttt{directionCode}. 
\item[\texttt{calibrationGroup}]: Used to group sources for calibration purposes    (kept for further use). 
\item[\texttt{catalog}]: Gives the origin for source information if taken from a catalog. 
\item[\texttt{deltaVel}]: Velocity resolution required for this source (from observer    input; kept for further use). 
\item[\texttt{position}]: Source position (when 3-dimentional information is available,    like a local transmitter).    {\\MBaseline{ITRF}} 
\item[\texttt{\f$N_{Line}\f$}]: Number of line transitions for which information is given. 
\item[\texttt{transition}]: The line of each spectral line transition. 
\item[\texttt{restFrequency}]: The rest frequency for each transition considered.   {\\MFrequency{LABREST}} 
\item[\texttt{sysVel}]: The source radial velocity in each of the line transitions.   {\\MRadialVelocity{LSR}} 
\item[\texttt{rangeVel}]: Range of radial velocities present in the source (for spectroscopy). 
\item[\texttt{sourceModel}]: A model for source. 
\item[\texttt{frequencyRefCode}]:  The reference code for \\texttt{frequency}    (if not \\texttt{LSRK}) 
\item[\texttt{\f$N_{Freq}\f$}]: The number of frequencies for which parameters are provided. 
\item[\texttt{\f$N_{Stok}\f$}]: The number of Stokes parameters that are provided at each freqeuncy. 
\item[\texttt{frequency}]: The list of frequencies for the  flux densities   {\\MFrequency{LSRK}} 
\item[\texttt{frequencyInterval}]: The list of frequency intervals over which flux densities    were measured. 
\item[\texttt{stokesParameter}]: The array of Stokes parameters (one value per parameter). 
\item[\texttt{flux}]: The values of fluxes provided in Jansky (Jy). 
\item[\texttt{fluxErr}]: The statistical uncertainties on the fluxes. 
\item[\texttt{positionAngle}]: Position angles of the major axes for the elliptical sources. 
\item[\texttt{positionAngleErr}]: Statistical uncertainties for  the major axis position angles. 
\item[\texttt{size}]: Source sizes measured along the major and minor axis. 
\item[\texttt{sizeErr}]: Statistical uncertainties on the source sizes. 
\item[\texttt{velRefCode}]: {\red long doc missing}
\end{description}
\endgroup

 \newpage

\subsection{SpectralWindow Table}

 
 Spectral window description. The convention in ALMA is to describe the  frequency axis in the topocentric reference frame. If this is not  the case (for instance if active Doppler tracking is implemented) then  \\texttt{measFreqRef} should be set accordingly.

\begingroup
%
% define shortcuts for dimensions

%\newcommand{\numChan}{\f$N_{Chan}\f$}
%\newcommand{\numAssocValues}{\f$N_{Asso}\f$}

\par\noindent\begin{longtable} {|p{45mm}|p{45mm}|p{65mm}|}
\hline \multicolumn{3}{|c|}{\textbf{SpectralWindow}} \\
\hline\hline
Name & Type (Shape) & Comment \\
\hline \endfirsthead
\hline \multicolumn{3}{|c|}{\textbf{SpectralWindow} -- continued from previous page} \\
\hline \hline
Name & Type (Shape) & Comment \\
\hline \endhead
\hline \endfoot


\hline \multicolumn{3}{|l|}{\em Key} \\
\hline 

\texttt{spectralWindowId} & \texttt{Tag} &  identifies a unique row in the table. \\
\hline \multicolumn{3}{|l|}{\em Required Data} \\
\hline
\texttt{basebandName} & \texttt{BasebandName} &
 identifies the baseband. \\
\texttt{netSideband} & \texttt{NetSideband} &
 identifies the net sideband. \\
\texttt{\f$N_{Chan}\f$} (\f$N_{Chan}\f$)& \texttt{int} &
 the number of frequency channels. \\
\texttt{refFreq} & \texttt{Frequency} &
 the reference frequency. \\
\texttt{sidebandProcessingMode} & \texttt{SidebandProcessingMode} &
 identifies the sideband processing mode. \\
\texttt{totBandwidth} & \texttt{Frequency} &
 the total bandwidth. \\
\texttt{windowFunction} & \texttt{WindowFunction} &
 identifies the window function. \\

\hline \multicolumn{3}{|l|}{\em Optional Data} \\
\hline
\texttt{chanFreqStart}  & \texttt{Frequency} &
 the frequency of the first channel. \\
\texttt{chanFreqStep}  & \texttt{Frequency} &
 the increment between two successive frequencies. \\
\texttt{chanFreqArray}  & \texttt{Frequency [numChan] } &
 the frequencies defined as an array (\\texttt{numChan} values). \\
\texttt{chanWidth}  & \texttt{Frequency} &
 the width of the frequency channel (supposedly constant). \\
\texttt{chanWidthArray}  & \texttt{Frequency [numChan] } &
 Array of channel widths \\
\texttt{correlationBit}  & \texttt{CorrelationBit} &
 identifies the number of bits used in the signal representation. \\
\texttt{effectiveBw}  & \texttt{Frequency} &
 the effective noise bandwidth. \\
\texttt{effectiveBwArray}  & \texttt{Frequency [numChan] } &
 array of effective bandwidths (one value per channel). \\
\texttt{freqGroup}  & \texttt{int} &
 the frequency group number. \\
\texttt{freqGroupName}  & \texttt{string} &
 the frequency group name. \\
\texttt{lineArray}  & \texttt{bool [numChan] } &
 indicates lines (true) versus baselines (false). \\
\texttt{measFreqRef}  & \texttt{FrequencyReferenceCode} &
 the reference frame of the frequencies. \\
\texttt{name}  & \texttt{string} &
 a name for this spectral window. \\
\texttt{oversampling}  & \texttt{bool} &
 data are "oversampled" (true) or not (false). \\
\texttt{quantization}  & \texttt{bool} &
 a quantization correction has been applied (true) or not applied (false). \\
\texttt{refChan}  & \texttt{double} &
 the reference channel "number". \\
\texttt{resolution}  & \texttt{Frequency} &
 the half power frequency resolution (supposedly constant for all the channels). \\
\texttt{resolutionArray}  & \texttt{Frequency [numChan] } &
 the frequency resolutions (possibly variable )(one value per channel). \\
\texttt{\f$N_{Asso}\f$} (\f$N_{Asso}\f$) & \texttt{int} &
 the number of associated values. \\
\texttt{assocNature}  & \texttt{SpectralResolutionType [numAssocValues] } &
 the natures of the associations with the rows refered to by assocSpectralWindowId. \\
\texttt{assocSpectralWindowId}  & \texttt{Tag [numAssocValues] } &
 refers to a collection of associated rows in the table. \\
\texttt{imageSpectralWindowId}  & \texttt{Tag} &
 refers to a unique row in the table (image sideband description). \\
\texttt{dopplerId}  & \texttt{int} &
 refers to a collection of rows in DopplerTable. \\
\hline
\end{longtable}
  
~\par\noindent{\bf Column Descriptions:}

\begin{description}
\item[\texttt{spectralWindowId}]: Identifies each row of the SpectralWindow  table. 
\item[\texttt{basebandName}]: Name of the baseband this spectral window is connected to. 
\item[\texttt{netSideband}]: Equivalent sideband of spectrum frequency   axis. Can be USB, LSB but also DSB , or NOSB (in the case of direct   amplification). 
\item[\texttt{\f$N_{Chan}\f$}]: The number of frequency channels. 
\item[\texttt{refFreq}]: This is the reference frequency of the spectrum,  it corresponds to the channel number \\texttt{refChan} in the case of equidistant channels. This is normally set to the zero-frequency edge of the spectral window, for correlation data, for compatibility with the Measurement Set.  \\MFrequency{TOPO} 
\item[\texttt{sidebandProcessingMode}]: In the case of single side band   data, provides information on how side bands were separated, or how image   sideband data was eliminated. 
\item[\texttt{totBandwidth}]: The total bandwidth for the spectral window. 
\item[\texttt{windowFunction}]: Indicates which window function was   applied to the time-domain data before performing Fourier transform, for   correlator data. Applying a window spectrum provides apodization of the   spectral response to a monochromatic line. 
\item[\texttt{chanFreqStart}]: The central frequency for the first    channel of the spectrum. Used  for regularily spaced   frequencies. Either the couple (chanFreqStart, chanFreqStep) or   chanFreqArray must be present.    \\MFrequency{TOPO}  
\item[\texttt{chanFreqStep}]: The channel frequency   separation. Used for regularily spaced frequencies. Either the couple   (chanFreqStart, chanFreqStep) or chanFreqArray must be present.    {\\MFrequency{TOPO} 
\item[\texttt{chanFreqArray}]: The central frequency for each   channel of the spectrum. Present for irregularily spaced   frequencies. Either the couple (chanFreqStart, chanFreqStep) or   chanFreqArray must be present.    {\\MFrequency{TOPO}  
\item[\texttt{chanWidth}]: The channel width (at half   power) of all channels. Used an array of identical channels.  Either   \\texttt{chanWidth} or \\texttt{chanWidthArray} must be present. 
\item[\texttt{chanWidthArray}]: The channel width at half power. Either \\texttt{chanWidth} or \\texttt{chanWidthArray} must be present. 
\item[\texttt{correlationBit}]: Indicates how many bits were used to   represent the signal to be correlated. 
\item[\texttt{effectiveBw}]: The effective noise   bandwidth of all channels in the spectrum (that is the white noise   bandwidth that produces the same output noise fluctuations). Either   \\texttt{effectiveBw} or \\texttt{effectiveBwArray} must be present. 
\item[\texttt{effectiveBwArray}]: The array of effective   noise bandwidth for each channel in the spectrum (that is the white noise   bandwidth that produces the same output noise fluctuations). Either   \\texttt{effectiveBw} or \\texttt{effectiveBwArray} must be present. 
\item[\texttt{freqGroup}]: Frequency group number. Used to associate   spectral windows for calibration purposes. 
\item[\texttt{freqGroupName}]: Name of the frequency group. Used to   associate spectral windows for calibration purposes. 
\item[\texttt{lineArray}]: Indicates line vs. baseline. For   channel averaged data, indicates the channels with line signal as opposed   to the ones with baseline data. Used for single-dish spectral line   pointing. 
\item[\texttt{measFreqRef}]: The reference frame for frequencies (if not   topocentric). 
\item[\texttt{name}]: Name of this spectral window. 
\item[\texttt{oversampling}]: Oversampling is true if the data is sampled   at more than the Nyquist rate (improving signal to noise). 
\item[\texttt{quantization}]: True if the quantization correction has been   applied to the data. 
\item[\texttt{refChan}]: This is the channel number   corresponding to the frequency \\texttt{refFreq}. It does not need to be an integer (for instance the   reference frequency may fall between two channels if a half-channel is   introduced in the correlator software). 
\item[\texttt{resolution}]: The half-power width of the   frequency channels. Used for identical channels. Either   \\texttt{resolution} or \\texttt{resolutionArray} must be   present. 
\item[\texttt{resolutionArray}]: The array of half-power   widths of the frequency channels. Either \\texttt{resolution} or   \\texttt{resolutionArray} must be present. 
\item[\texttt{\f$N_{Asso}\f$}]: Number of associated spectral windows. 
\item[\texttt{assocNature}]: Nature of association between spectral   windows. In some cases two spectral windows are generated with the same   data (full resolution and channel averaged); the association mechanism   links these together. 
\item[\texttt{assocSpectralWindowId}]: Associates another focus model     used in addition. Used for the auxiliary pointing model (e.g. local     pointing model, ...). The actual associated row is obtained by     associating the current \\texttt{antennaId} with     \\texttt{associatedFocusModelId} to form the key. 
\item[\texttt{imageSpectralWindowId}]: The row in the SpectralWindow table which contains the description of the image sideband. 
\item[\texttt{dopplerId}]: Doppler Table identifier. 
\end{description}
\endgroup

 \newpage

\subsection{SquareLawDetector Table}

 
 Processor information for total power detectors.

\begingroup
%
% define shortcuts for dimensions

%\newcommand{\numBand}{\f$N_{Band}\f$}

\par\noindent\begin{longtable} {|p{45mm}|p{45mm}|p{65mm}|}
\hline \multicolumn{3}{|c|}{\textbf{SquareLawDetector}} \\
\hline\hline
Name & Type (Shape) & Comment \\
\hline \endfirsthead
\hline \multicolumn{3}{|c|}{\textbf{SquareLawDetector} -- continued from previous page} \\
\hline \hline
Name & Type (Shape) & Comment \\
\hline \endhead
\hline \endfoot


\hline \multicolumn{3}{|l|}{\em Key} \\
\hline 

\texttt{squareLawDetectorId} & \texttt{Tag} &  identifies a unique row in the table. \\
\hline \multicolumn{3}{|l|}{\em Required Data} \\
\hline
\texttt{\f$N_{Band}\f$} (\f$N_{Band}\f$)& \texttt{int} &
 the number of bands. \\
\texttt{bandType} & \texttt{DetectorBandType} &
 identifies the band type. \\

\hline
\end{longtable}
  
~\par\noindent{\bf Column Descriptions:}

\begin{description}
\item[\texttt{squareLawDetectorId}]: Identifies each row of the Square law detector   table. 
\item[\texttt{\f$N_{Band}\f$}]: The number of band detectors. 
\item[\texttt{bandType}]: Band type, e.g. BASEBAND (for Alma there is also a detector          higher in the chain, covering all basebands). 
\end{description}
\endgroup

 \newpage

\subsection{State Table}

 
 State information.

\begingroup
%
% define shortcuts for dimensions


\par\noindent\begin{longtable} {|p{45mm}|p{45mm}|p{65mm}|}
\hline \multicolumn{3}{|c|}{\textbf{State}} \\
\hline\hline
Name & Type (Shape) & Comment \\
\hline \endfirsthead
\hline \multicolumn{3}{|c|}{\textbf{State} -- continued from previous page} \\
\hline \hline
Name & Type (Shape) & Comment \\
\hline \endhead
\hline \endfoot


\hline \multicolumn{3}{|l|}{\em Key} \\
\hline 

\texttt{stateId} & \texttt{Tag} &  identifies a unique row in the table. \\
\hline \multicolumn{3}{|l|}{\em Required Data} \\
\hline
\texttt{calDeviceName} & \texttt{CalibrationDevice} &
 the name of the calibration device. \\
\texttt{sig} & \texttt{bool} &
 data to be used for signal measurement (true) or not (false). \\
\texttt{ref} & \texttt{bool} &
 data to be used for reference measurement (true) or not (false). \\
\texttt{onSky} & \texttt{bool} &
 the beam is on the sky (true) or not (false). \\

\hline \multicolumn{3}{|l|}{\em Optional Data} \\
\hline
\texttt{weight}  & \texttt{float} &
 used for combining data. \\
\hline
\end{longtable}
  
~\par\noindent{\bf Column Descriptions:}

\begin{description}
\item[\texttt{stateId}]: Identifies each row of the State  table. 
\item[\texttt{calDeviceName}]: Name of the calibration device (must be   there {if} onSky is false). 
\item[\texttt{sig}]: Data will be used for signal measurement. 
\item[\texttt{ref}]: Data will be used for reference measurement. 
\item[\texttt{onSky}]: True if the receiver is getting radiation from the sky. 
\item[\texttt{weight}]: Weight to be used for combining data (useful when   there are more than one  signal or reference states). 
\end{description}
\endgroup

 \newpage

\subsection{Station Table}

 
 Antenna station information.

\begingroup
%
% define shortcuts for dimensions


\par\noindent\begin{longtable} {|p{45mm}|p{45mm}|p{65mm}|}
\hline \multicolumn{3}{|c|}{\textbf{Station}} \\
\hline\hline
Name & Type (Shape) & Comment \\
\hline \endfirsthead
\hline \multicolumn{3}{|c|}{\textbf{Station} -- continued from previous page} \\
\hline \hline
Name & Type (Shape) & Comment \\
\hline \endhead
\hline \endfoot


\hline \multicolumn{3}{|l|}{\em Key} \\
\hline 

\texttt{stationId} & \texttt{Tag} &  Station identifier. \\
\hline \multicolumn{3}{|l|}{\em Required Data} \\
\hline
\texttt{name} & \texttt{string} &
 the name of the station. \\
\texttt{position} & \texttt{Length [3] } &
 the position of the station. \\
\texttt{type} & \texttt{StationType} &
 identifies the type of the station. \\

\hline \multicolumn{3}{|l|}{\em Optional Data} \\
\hline
\texttt{time}  & \texttt{ArrayTime} &
 the time of position measurement.  \\
\hline
\end{longtable}
  
~\par\noindent{\bf Column Descriptions:}

\begin{description}
\item[\texttt{stationId}]: Identifies each row of the Station  table. 
\item[\texttt{name}]: The name of the station. 
\item[\texttt{position}]: The position of the station in  geocentric frame of reference      (X, Y, Z coordinates). {\\MPosition{ITRF} }  
\item[\texttt{type}]: The station type, e.g. ANTENNA or WEATHER. 
\item[\texttt{time}]: {\red long doc missing}
\end{description}
\endgroup

 \newpage

\subsection{Subscan Table}

 
 Subscan-based information.

\begingroup
%
% define shortcuts for dimensions

%\newcommand{\numIntegration}{\f$N_{Inte}\f$}
%\newcommand{\numSubintegration}{\f$N_{Subi}\f$}

\par\noindent\begin{longtable} {|p{45mm}|p{45mm}|p{65mm}|}
\hline \multicolumn{3}{|c|}{\textbf{Subscan}} \\
\hline\hline
Name & Type (Shape) & Comment \\
\hline \endfirsthead
\hline \multicolumn{3}{|c|}{\textbf{Subscan} -- continued from previous page} \\
\hline \hline
Name & Type (Shape) & Comment \\
\hline \endhead
\hline \endfoot


\hline \multicolumn{3}{|l|}{\em Key} \\
\hline 

\texttt{execBlockId} & \texttt{Tag} &  identifies a unique row in ExecBlockTable. \\
\texttt{scanNumber} & \texttt{int} &  the number of the scan this subscan belongs to. \\
\texttt{subscanNumber} & \texttt{int} &  the subscan number. \\
\hline \multicolumn{3}{|l|}{\em Required Data} \\
\hline
\texttt{startTime} & \texttt{ArrayTime} &
 the actual start time of the subscan. \\
\texttt{endTime} & \texttt{ArrayTime} &
 the actual end time of the subscan. \\
\texttt{fieldName} & \texttt{string} &
 the name of the observed field. \\
\texttt{subscanIntent} & \texttt{SubscanIntent} &
 the intent of the subscan. \\
\texttt{\f$N_{Inte}\f$} (\f$N_{Inte}\f$)& \texttt{int} &
 the number of integrations during the scan. \\
\texttt{\f$N_{Subi}\f$} (\f$N_{Subi}\f$)& \texttt{int [numIntegration] } &
 the number of subintegrations for each integration. \\

\hline \multicolumn{3}{|l|}{\em Optional Data} \\
\hline
\texttt{subscanMode}  & \texttt{SwitchingMode} &
 identifies the data acquisition mode during the subscan. \\
\texttt{correlatorCalibration}  & \texttt{CorrelatorCalibration} &
 identifies the correlator calibration during the subscan. \\
\hline
\end{longtable}
  
~\par\noindent{\bf Column Descriptions:}

\begin{description}
\item[\texttt{execBlockId}]: ExecBlock Table identifier. 
\item[\texttt{scanNumber}]: The scan number. This increments starting from 1 in each Execution Block. 
\item[\texttt{subscanNumber}]: The subscans in a scan are numbered incrementally starting from 1 in each new scan. 
\item[\texttt{startTime}]: The time when the subscan started collecting science data. 
\item[\texttt{endTime}]: The time when the subscan finished collecting science data. 
\item[\texttt{fieldName}]: The name of the field observed during this subscan. 
\item[\texttt{subscanIntent}]: The intent for this subscan. 
\item[\texttt{\f$N_{Inte}\f$}]: The number of integrations in this subscan. 
\item[\texttt{\f$N_{Subi}\f$}]: The number of sub-integrations in each one of the integrations of this subscan. 
\item[\texttt{subscanMode}]: Describe the mode in which data was taken. Can be e.g. nutator switching, frequency switching, or no switching. 
\item[\texttt{correlatorCalibration}]: Indicates whether this subscan is used for correlator internal calibration. 
\end{description}
\endgroup

 \newpage

\subsection{SwitchCycle Table}

 
 Cycle information in switching modes. Describe each step in a switching cycle.

\begingroup
%
% define shortcuts for dimensions

%\newcommand{\numStep}{\f$N_{Step}\f$}

\par\noindent\begin{longtable} {|p{45mm}|p{45mm}|p{65mm}|}
\hline \multicolumn{3}{|c|}{\textbf{SwitchCycle}} \\
\hline\hline
Name & Type (Shape) & Comment \\
\hline \endfirsthead
\hline \multicolumn{3}{|c|}{\textbf{SwitchCycle} -- continued from previous page} \\
\hline \hline
Name & Type (Shape) & Comment \\
\hline \endhead
\hline \endfoot


\hline \multicolumn{3}{|l|}{\em Key} \\
\hline 

\texttt{switchCycleId} & \texttt{Tag} &  identifies a unique row in the table. \\
\hline \multicolumn{3}{|l|}{\em Required Data} \\
\hline
\texttt{\f$N_{Step}\f$} (\f$N_{Step}\f$)& \texttt{int} &
 the number of steps. \\
\texttt{weightArray} & \texttt{float [numStep] } &
 the weights (one value per step). \\
\texttt{dirOffsetArray} & \texttt{Angle [numStep]  [2] } &
 the pointing direction offsets (one pair per step). \\
\texttt{freqOffsetArray} & \texttt{Frequency [numStep] } &
 the frequencies offsets (one value per step). \\
\texttt{stepDurationArray} & \texttt{Interval [numStep] } &
 the duration of the steps (one value per steps). \\

\hline \multicolumn{3}{|l|}{\em Optional Data} \\
\hline
\texttt{directionCode}  & \texttt{DirectionReferenceCode} &
 the reference frame associated to dirOffsetArray.t \\
\texttt{directionEquinox}  & \texttt{ArrayTime} &
 the equinox associated to directionCode (if required). \\
\hline
\end{longtable}
  
~\par\noindent{\bf Column Descriptions:}

\begin{description}
\item[\texttt{switchCycleId}]: Identifies each row of the Switch Cycle  table. 
\item[\texttt{\f$N_{Step}\f$}]: The number of steps in the swtching cycle. 
\item[\texttt{weightArray}]: Specify a weight to be used for each step of the switching cycle, when combining data (useful to differenciate signal and reference steps). 
\item[\texttt{dirOffsetArray}]: Gives the pointing direction offsets in each step of the switching cycle \\MDirection{AZEL}{NOW}{Antenna.position}  
\item[\texttt{freqOffsetArray}]: Gives the frequency offsets in each step of the switching cycle. 
\item[\texttt{stepDurationArray}]: The durations of the steps in the switching cycle. 
\item[\texttt{directionCode}]: The reference frame associated to \\textt{dirOffsetArray}. 
\item[\texttt{directionEquinox}]: The source direction reference equinox ,    if required by \\texttt{directionCode} (e.g. being other than    \\texttt{AZEL} or \\texttt{J2000}) 
\end{description}
\endgroup

 \newpage

\subsection{SysCal Table}

 
 System calibration. Gives information on the conversion  of data to temperature scale. This table is reduced to follow  the contents of the Measurement Set SysCal table. Use only spectral  values (use a single channel spectral window for single numbers).   \\texttt{numChan} can be found in the SpectralWindow Table.  The contents of this table are used to scale the data in the filler.

\begingroup
%
% define shortcuts for dimensions

%\newcommand{\numReceptor}{\f$N_{Rece}\f$}
%\newcommand{\numChan}{\f$N_{Chan}\f$}

\par\noindent\begin{longtable} {|p{45mm}|p{45mm}|p{65mm}|}
\hline \multicolumn{3}{|c|}{\textbf{SysCal}} \\
\hline\hline
Name & Type (Shape) & Comment \\
\hline \endfirsthead
\hline \multicolumn{3}{|c|}{\textbf{SysCal} -- continued from previous page} \\
\hline \hline
Name & Type (Shape) & Comment \\
\hline \endhead
\hline \endfoot


\hline \multicolumn{3}{|l|}{\em Key} \\
\hline 

\texttt{antennaId} & \texttt{Tag} &  refers to a unique row  in AntennaTable. \\
\texttt{spectralWindowId} & \texttt{Tag} &  refers to a unique row in SpectralWindowTable. \\
\texttt{timeInterval} & \texttt{ArrayTimeInterval} &  time interval for which the row's content is valid. \\
\texttt{feedId} & \texttt{int} &  refers to a collection of rows in FeedTable. \\
\hline \multicolumn{3}{|l|}{\em Required Data} \\
\hline
\texttt{\f$N_{Rece}\f$} (\f$N_{Rece}\f$)& \texttt{int} &
 the number of receptors. \\
\texttt{\f$N_{Chan}\f$} (\f$N_{Chan}\f$)& \texttt{int} &
 the number of frequency channels. \\

\hline \multicolumn{3}{|l|}{\em Optional Data} \\
\hline
\texttt{tcalFlag}  & \texttt{bool} &
 the calibration temperature flag. \\
\texttt{tcalSpectrum}  & \texttt{Temperature [numReceptor]  [numChan] } &
 the calibration temperatures (one value per receptor per channel). \\
\texttt{trxFlag}  & \texttt{bool} &
 the receiver temperature flag. \\
\texttt{trxSpectrum}  & \texttt{Temperature [numReceptor]  [numChan] } &
 the receiver temperatures (one value per receptor per channel). \\
\texttt{tskyFlag}  & \texttt{bool} &
 the sky temperature flag. \\
\texttt{tskySpectrum}  & \texttt{Temperature [numReceptor]  [numChan] } &
 the sky temperatures (one value per receptor per channel). \\
\texttt{tsysFlag}  & \texttt{bool} &
 the system temperature flag. \\
\texttt{tsysSpectrum}  & \texttt{Temperature [numReceptor]  [numChan] } &
 the system temperatures (one value per receptor per channel). \\
\texttt{tantFlag}  & \texttt{bool} &
 the tant flag. \\
\texttt{tantSpectrum}  & \texttt{float [numReceptor]  [numChan] } &
 the Tant spectrum (one value per receptor per channel). \\
\texttt{tantTsysFlag}  & \texttt{bool} &
 the Tant/Tsys flag. \\
\texttt{tantTsysSpectrum}  & \texttt{float [numReceptor]  [numChan] } &
 the Tant/Tsys spectrum(one value per receptor per channel) . \\
\texttt{phaseDiffFlag}  & \texttt{bool} &
 the phase difference flag. \\
\texttt{phaseDiffSpectrum}  & \texttt{float [numReceptor]  [numChan] } &
 the phase difference spectrum (one value per receptor per channel). \\
\hline
\end{longtable}
  
~\par\noindent{\bf Column Descriptions:}

\begin{description}
\item[\texttt{antennaId}]: Specifies which antenna was used in the Antenna Table. 
\item[\texttt{spectralWindowId}]: SpectralWindow Table identifier. 
\item[\texttt{timeInterval}]: Time Interval of  validity of the SysCal information 
\item[\texttt{feedId}]: Specifies which feed was used in the Feed Table. 
\item[\texttt{\f$N_{Rece}\f$}]: The number of receptors. 
\item[\texttt{\f$N_{Chan}\f$}]: The number of frequency channels. 
\item[\texttt{tcalFlag}]: Calibration temperature flag. 
\item[\texttt{tcalSpectrum}]: Calibration temperature for each channel. 
\item[\texttt{trxFlag}]: Receiver temperature flag. 
\item[\texttt{trxSpectrum}]: Receiver temperature for each channel. 
\item[\texttt{tskyFlag}]: Sky temperature flag. 
\item[\texttt{tskySpectrum}]: Sky temperature for each channel 
\item[\texttt{tsysFlag}]: System temperature flag. 
\item[\texttt{tsysSpectrum}]: System temperature for each channel. 
\item[\texttt{tantFlag}]: Tant Flag (MS compatibility). 
\item[\texttt{tantSpectrum}]: Tant Spectrum (MS compatibility). 
\item[\texttt{tantTsysFlag}]: Tant/Tsys Flag (MS compatibility). 
\item[\texttt{tantTsysSpectrum}]: Tant/Tsys Spectrum (MS compatibility). 
\item[\texttt{phaseDiffFlag}]: Phase Difference Flag  (MS compatibility) 
\item[\texttt{phaseDiffSpectrum}]: Phase Difference Spectrum  (MS compatibility) 
\end{description}
\endgroup

 \newpage

\subsection{SysPower Table}

 
 This table is intended to store power measurements based on a synchronous power detector as  used at the EVLA. While the table is intended to be general enough for use with other arrays, it is  deeply entwined with the EVLA data acquisition scheme. The units of \\f\$P_{diff}\\f\$ and \\f\$P_{sum}\\f\$ are such that the system temperature, \\f\$T_{sys}\\f\$, is :
\\f[
T_{sys}=\\frac{P_{sum}}{P_{diff}} . \\frac{T_{cal}}{2} 
\\f]
where \\f\$T_{cal}\\f\$ is the temperature of the noise tube.


\begingroup
%
% define shortcuts for dimensions

%\newcommand{\numReceptor}{\f$N_{Rece}\f$}

\par\noindent\begin{longtable} {|p{45mm}|p{45mm}|p{65mm}|}
\hline \multicolumn{3}{|c|}{\textbf{SysPower}} \\
\hline\hline
Name & Type (Shape) & Comment \\
\hline \endfirsthead
\hline \multicolumn{3}{|c|}{\textbf{SysPower} -- continued from previous page} \\
\hline \hline
Name & Type (Shape) & Comment \\
\hline \endhead
\hline \endfoot


\hline \multicolumn{3}{|l|}{\em Key} \\
\hline 

\texttt{antennaId} & \texttt{Tag} &    \\
\texttt{spectralWindowId} & \texttt{Tag} &    \\
\texttt{feedId} & \texttt{int} &    \\
\texttt{timeInterval} & \texttt{ArrayTimeInterval} &  time interval over which the content of the row was measured.  \\
\hline \multicolumn{3}{|l|}{\em Required Data} \\
\hline
\texttt{\f$N_{Rece}\f$} (\f$N_{Rece}\f$)& \texttt{int} &
 The number of receptors.  \\

\hline \multicolumn{3}{|l|}{\em Optional Data} \\
\hline
\texttt{switchedPowerDifference}  & \texttt{float [numReceptor] } &
 the switched power difference \\f\\$P_{diff}\\f\\$ defined by \\f\\$P_{diff} = G*(P_{on}-P_{off})\\f\\$.  \\
\texttt{switchedPowerSum}  & \texttt{float [numReceptor] } &
 the switched power sum \\f\\$P_{sum}\\f\\$ defined by \\f\\$P_{sum}=G(P_{on} + P_{off})\\f\\$.  \\
\texttt{requantizerGain}  & \texttt{float [numReceptor] } &
 refers to the gain inserted after the synchronous power detector. For WIDAR, it is the requantizer gain (\\f\\$G\\f\\$).  \\
\hline
\end{longtable}
  
~\par\noindent{\bf Column Descriptions:}

\begin{description}
\item[\texttt{antennaId}]: {\red long doc missing}
\item[\texttt{spectralWindowId}]: {\red long doc missing}
\item[\texttt{feedId}]: {\red long doc missing}
\item[\texttt{timeInterval}]: {\red long doc missing}
\item[\texttt{\f$N_{Rece}\f$}]: {\red long doc missing}
\item[\texttt{switchedPowerDifference}]: {\red long doc missing}
\item[\texttt{switchedPowerSum}]: {\red long doc missing}
\item[\texttt{requantizerGain}]: {\red long doc missing}
\end{description}
\endgroup

 \newpage


 \newpage

\subsection{WVMCal Table}

 
 Coefficients to use water vapour monitor information to correct for  pathlength variations. This contains the coefficients actually used, while  CalWVR contains the coefficients derived from TelCal calibration.

\begingroup
%
% define shortcuts for dimensions

%\newcommand{\numInputAntenna}{\f$N_{Inpu}\f$}
%\newcommand{\numChan}{\f$N_{Chan}\f$}
%\newcommand{\numPoly}{\f$N_{Poly}\f$}

\par\noindent\begin{longtable} {|p{45mm}|p{45mm}|p{65mm}|}
\hline \multicolumn{3}{|c|}{\textbf{WVMCal}} \\
\hline\hline
Name & Type (Shape) & Comment \\
\hline \endfirsthead
\hline \multicolumn{3}{|c|}{\textbf{WVMCal} -- continued from previous page} \\
\hline \hline
Name & Type (Shape) & Comment \\
\hline \endhead
\hline \endfoot


\hline \multicolumn{3}{|l|}{\em Key} \\
\hline 

\texttt{antennaId} & \texttt{Tag} &  refers to a unique row in AntennaTable. \\
\texttt{spectralWindowId} & \texttt{Tag} &  refers to a unique row in SpectralWindowTable. \\
\texttt{timeInterval} & \texttt{ArrayTimeInterval} &  the time interval for which the row's content is valid. \\
\hline \multicolumn{3}{|l|}{\em Required Data} \\
\hline
\texttt{wvrMethod} & \texttt{WVRMethod} &
 identifies the calibration method. \\
\texttt{polyFreqLimits} & \texttt{Frequency [2] } &
 the range of frequencies in which the computation is performed. \\
\texttt{\f$N_{Inpu}\f$} (\f$N_{Inpu}\f$)& \texttt{int} &
 The number of antennas used for the calculations.  \\
\texttt{\f$N_{Chan}\f$} (\f$N_{Chan}\f$)& \texttt{int} &
 the number of WVR channels. \\
\texttt{\f$N_{Poly}\f$} (\f$N_{Poly}\f$)& \texttt{int} &
 the number of coefficients used in the polynomial expansions. \\
\texttt{pathCoeff} & \texttt{float [numInputAntenna]  [numChan]  [numPoly] } &
 the pathlengths coefficients (one value per antenna per chan per coefficient). \\
\texttt{refTemp} & \texttt{Temperature [numInputAntenna]  [numChan] } &
 the reference temperatures (one value par antenna per channel). \\
\texttt{inputAntennaId} & \texttt{Tag [numInputAntenna] } &
 Refers to row(s) in the Antenna table describing the antenna(s) used for the calculations. It is a 1D array expected to have numInputAntenna elements.  \\

\hline
\end{longtable}
  
~\par\noindent{\bf Column Descriptions:}

\begin{description}
\item[\texttt{antennaId}]: Antenna Table identifier. 
\item[\texttt{spectralWindowId}]: SpectralWindow Table identifier. 
\item[\texttt{timeInterval}]: Time Interval of  validity of the Water Vapour  information. 
\item[\texttt{wvrMethod}]: Calibration method (see CalWVR table). 
\item[\texttt{polyFreqLimits}]: Frequency limits of the frequency   interval for which the path length coefficients are computed. 
\item[\texttt{\f$N_{Inpu}\f$}]: {\red long doc missing}
\item[\texttt{\f$N_{Chan}\f$}]: The number of WVR channels. 
\item[\texttt{\f$N_{Poly}\f$}]: Number of coefficients in polynomial expansion on frequency. 
\item[\texttt{pathCoeff}]: Pathlength coefficients (see CalWVR table). 
\item[\texttt{refTemp}]: Reference temperatures  (see CalWVR table). 
\item[\texttt{inputAntennaId}]: {\red long doc missing}
\end{description}
\endgroup

 \newpage

\subsection{Weather Table}

 
 Weather station information.

\begingroup
%
% define shortcuts for dimensions

%\newcommand{\numLayer}{\f$N_{Laye}\f$}
%\newcommand{\numWVR}{\f$N_{WVR}\f$}

\par\noindent\begin{longtable} {|p{45mm}|p{45mm}|p{65mm}|}
\hline \multicolumn{3}{|c|}{\textbf{Weather}} \\
\hline\hline
Name & Type (Shape) & Comment \\
\hline \endfirsthead
\hline \multicolumn{3}{|c|}{\textbf{Weather} -- continued from previous page} \\
\hline \hline
Name & Type (Shape) & Comment \\
\hline \endhead
\hline \endfoot


\hline \multicolumn{3}{|l|}{\em Key} \\
\hline 

\texttt{stationId} & \texttt{Tag} &  refers to a unique row in StationTable. \\
\texttt{timeInterval} & \texttt{ArrayTimeInterval} &  the time interval for which the row's content is valid. \\
\hline \multicolumn{3}{|l|}{\em Required Data} \\
\hline

\hline \multicolumn{3}{|l|}{\em Optional Data} \\
\hline
\texttt{pressure}  & \texttt{Pressure} &
 the ambient pressure. \\
\texttt{relHumidity}  & \texttt{Humidity} &
 the relative humidity. \\
\texttt{temperature}  & \texttt{Temperature} &
 the ambient temperature. \\
\texttt{windDirection}  & \texttt{Angle} &
 the wind direction. \\
\texttt{windSpeed}  & \texttt{Speed} &
 the wind speed. \\
\texttt{windMax}  & \texttt{Speed} &
 the maximum wind speed \\
\texttt{dewPoint}  & \texttt{Temperature} &
 the dew point's value. \\
\texttt{\f$N_{Laye}\f$} (\f$N_{Laye}\f$) & \texttt{int} &
 NLayer the number of layers in the temperature profile.  \\
\texttt{layerHeight}  & \texttt{Length [numLayer] } &
 the height of each layer for the temperature profile.  \\
\texttt{temperatureProfile}  & \texttt{Temperature [numLayer] } &
 the temperature on the atmosphere at each height.  \\
\texttt{cloudMonitor}  & \texttt{Temperature} &
 the temperature of the cloud monitor.  \\
\texttt{\f$N_{WVR}\f$} (\f$N_{WVR}\f$) & \texttt{int} &
 the number of WVR channels.  \\
\texttt{wvrTemp}  & \texttt{Temperature [numWVR] } &
 the observed temperature in each WVR channel.  \\
\texttt{water}  & \texttt{double} &
 the water precipitable content.  \\
\hline
\end{longtable}
  
~\par\noindent{\bf Column Descriptions:}

\begin{description}
\item[\texttt{stationId}]: Station Table identifier.
\item[\texttt{timeInterval}]: Time Interval of  validity of the weather information. 
\item[\texttt{pressure}]: Ambient pressure  (observatory ground level). 
\item[\texttt{relHumidity}]: Relative humidity (related to dewpoint temperature). 
\item[\texttt{temperature}]: Ambient temperature. 
\item[\texttt{windDirection}]: Wind direction (counted East from North). 
\item[\texttt{windSpeed}]: Average wind speed in interval. 
\item[\texttt{windMax}]: Maximal wind speed in time interval. 
\item[\texttt{dewPoint}]: Dewpoint temperature (related to humidity) 
\item[\texttt{\f$N_{Laye}\f$}]: {\red long doc missing}
\item[\texttt{layerHeight}]: {\red long doc missing}
\item[\texttt{temperatureProfile}]: {\red long doc missing}
\item[\texttt{cloudMonitor}]: {\red long doc missing}
\item[\texttt{\f$N_{WVR}\f$}]: {\red long doc missing}
\item[\texttt{wvrTemp}]: {\red long doc missing}
\item[\texttt{water}]: {\red long doc missing}
\end{description}
\endgroup

